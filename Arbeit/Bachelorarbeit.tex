\documentclass[11pt,a4paper]{book} % Basisdokumentenklasse

\usepackage{amsmath,amssymb,amsthm}
\usepackage{pgf, tikz}
\usetikzlibrary{decorations.pathreplacing}
\usetikzlibrary{positioning,shapes,shadows,arrows}
\usepackage{xspace}
\usepackage{typearea}
\usepackage{lmodern}
\usepackage{stmaryrd}
\usepackage{fancyhdr}
\usepackage{listings}
\usepackage[T1]{fontenc}
\usepackage[utf8]{inputenc}
\usepackage[ngerman, english]{babel}
\usepackage[inner=3cm,outer=2cm,top=3cm,bottom=2.5cm]{geometry}
\usepackage{textcomp}
\usepackage[hyperfootnotes=false,bookmarksopen=true]{hyperref}
\usepackage{cleveref}
\usepackage{algorithm}
\usepackage{verbatim}
\usepackage{fancyvrb}
\usepackage{algpseudocode}
\usepackage{cite}
\usepackage{enumerate}
\usepackage{multicol}
\usepackage{graphicx}
\usepackage{oldgerm}
\usepackage{subfigure}
\usepackage{setspace}
\usetikzlibrary{arrows,shapes,positioning,automata}
\usepackage{booktabs}
\usepackage{tabularx}
\usepackage{multirow}
\usepackage{todonotes}
\usepackage{wrapfig}

\usepackage{mdframed}
\usepackage[section]{placeins}
\usetikzlibrary{patterns,snakes}

\allowdisplaybreaks

%%%%%%%%%%%%%%%%%%%%%%%%%%%%%%%%%%%%%%%%%%%%%%%%%%%%%%%%
\pagestyle{fancyplain}

\lhead[\fancyplain{}{\thepage}]{\fancyplain{}{\let\uppercase\relax\sl\rightmark}}
\chead{}
\rhead[\fancyplain{}{\let\uppercase\relax\sl\leftmark}]{\fancyplain{}{\thepage}}
\cfoot{}
\pagenumbering{arabic}
\makeatletter
\def\cleardoublepage{\clearpage\if@twoside\ifodd\c@page\else
	\hbox{}
	\thispagestyle{empty}
	\newpage
	\if@twocolumn\hbox{}\newpage\fi\fi\fi}
\makeatother

\lstset{
	backgroundcolor=\color{white},	% choose the background color
	basicstyle=\ttfamily,			% the size of the fonts that are used for the code
	breakatwhitespace=false,		% sets if automatic breaks should only happen at whitespace
	breaklines=true,				% sets automatic line breaking
	captionpos=b,					% sets the caption-position to bottom
	commentstyle=\color{green!60!black},% comment style
	deletekeywords={...},			% if you want to delete keywords from the given language
	escapeinside={\%*}{*)},			% if you want to add LaTeX within your code
	frame=none,					% adds a frame around the code
	keepspaces=false,				% keeps spaces in text, useful for keeping indentation of code
	% (possibly needs columns=flexible)
	keywordstyle=\color{blue},		% keyword style
	lineskip=0.1pt,
	%linewidth=0.7\linewidth,
	frame=lines,
	morekeywords={*,...},			% if you want to add more keywords to the set
	numbers=left,				% where to put the line-numbers; possible values are (none, left, right)
	numbersep=5pt,				% how far the line-numbers are from the code
	numberstyle=\tiny\color{gray},		% the style that is used for the line-numbers
	rulecolor=\color{black},			% if not set, the frame-color may change on line-breaks within not-black text
	showspaces=false,				% show spaces using underscores; overrides 'showstringspaces'
	showstringspaces=false,			% underline spaces within strings only
	showtabs=false,				% show tabs within strings adding particular underscores
	stepnumber=1,				% the step between two line-numbers. If it's 1, each line will be numbered
	stringstyle=\color{mymauve},		% string literal style
	tabsize=2,					% sets default tabsize to 2 spaces
	%title=\lstname	,				% show the filename of files included with \lstinputlisting; also try caption instead of title
	xleftmargin=2cm,
	xrightmargin=2cm
} 


%%%%%%%%%%%%%%%%%%%%%%%%%%%%%%%%%%%%%%%%%%%%%%%%%%%%%%%%
\newcommand{\labelname}[1]{% \labelname{<stuff>}
	\def\@currentlabelname{#1}}%

\newcommand{\code}[1]{\textit{#1}}
\newcommand{\tool}[1]{\textsf{#1}}
\newcommand{\aprove}{\tool{AProVE}\xspace}
\newcommand{\HRule}[1]{\rule{\linewidth}{#1}}
\newcommand{\stem}{\textit{STEM}\xspace}
\newcommand{\loopt}{\textit{LOOP}\xspace}
\newcommand{\guardmatrix}{\textit{Guard Matrix}\xspace}
\newcommand{\guardconstants}{\textit{Guard Constants}\xspace}
\newcommand{\updatematrix}{\textit{Update Matrix}\xspace}
\newcommand{\updateconstants}{\textit{Update Constants}\xspace}
\newcommand{\iterationmatrix}{\textit{Iteration Matrix}\xspace}
\newcommand{\iterationconstants}{\textit{Iteration Constants}\xspace}
\newcommand{\startterm}{\textit{start term}\xspace}
\newcommand{\gna}{\textit{geometric nontermination argument}\xspace}
\newcommand{\gnanal}{\textit{geometric nontermination analysis}\xspace}
\newcommand{\rpntree}{\textit{Reverse Polish Notation Tree}\xspace}
\newcommand{\smtfactory}{\code{SMTFactory}\xspace}
\newcommand{\its}{\tool{int-TRS}\xspace}
\newcommand{\llvm}{\tool{llvm}\xspace}
\newcommand{\nonterm}{\textit{non-termination}\xspace}
\newcommand{\seg}{\textit{Symbolic Execution Graph}\xspace}
\newcommand{\stdLinInt}{standard linear integer form\xspace}
\newcommand{\stdG}{standard guard form\xspace}
\newcommand{\strG}{strict guard form\xspace}
\newcommand{\solver}{\textit{SMT}-\textit{solver}\xspace}
\newcommand{\domc}{\textit{Domain Criteria}\xspace}
\newcommand{\initc}{\textit{Initiation Criteria}\xspace}
\newcommand{\pointc}{\textit{Point Criteria}\xspace}
\newcommand{\rayc}{\textit{Ray Criteria}\xspace}
\newcommand{\addass}{\textit{Additional assertion}\xspace}
\newcommand{\qfnia}{\textit{quantifier free non-linear integer arithmetic}\xspace}
\newcommand{\lasso}{\tool{lasso}\xspace}

\newtheorem{definition}{Definition}[section]
\newtheorem{corollary}{Corollary}[section]
\newtheorem{example}{Example}
\newtheorem{satz}{Sentence}[chapter]
\newtheoremstyle{example}{5pt}{5pt}{}{}{\bfseries}{.}{0.5em}{}

\tikzstyle{arrow} = [->]
\tikzstyle{thickarrow}=[arrow, thick]
\tikzstyle{dashedarrow} = [thickarrow, >=stealth, dashed, line width = 1pt ]
\tikzstyle{stdNode} = [shape = rectangle, draw, rounded corners]
\tikzstyle{aproveNode} = [shape = rectangle, draw, rounded corners, minimum width = 2cm]
\tikzstyle{objDia}=[rectangle, draw=black, rounded corners, text centered, anchor=north, rectangle split, rectangle split parts=2]
\tikzstyle{class}=[rectangle, draw=black, rounded corners, text centered, anchor=north, rectangle split, rectangle split parts=3]
\tikzstyle{explNode} = [align = center, minimum height = 10pt,	font = \bfseries, fill= green!20]
\tikzstyle{considered}=[thickarrow, color = blue]
\tikzstyle{neglected} = [thickarrow, color = red]
\tikzstyle{query} = [thickarrow, color = green!50!black]

\newcommand{\definitionautorefname}{definition}
\newcommand{\exampleautorefname}{example}
\newcommand{\listingautorefname}{figure}
\newcommand{\algorithmautorefname}{algorithm}
\renewcommand{\sectionautorefname}{Section}

\begin{document}

% Einrücken von Absätzen verhindern und 1.5 Zeilen Absatzabstand
\setlength{\parindent}{0pt}
\setlength{\parskip}{1.5ex plus0.5ex minus0.5ex}

%\begin{titlepage}
%	\begin{center}
%		\huge \textbf{\textsf{Geometric Non-Termination Arguments for Integer Programs}} \\
%		\vspace{2cm}
%		\LARGE\textbf{\textsc{Bachelor-Thesis}}\\
%		\vspace{1cm}
%		\normalsize
%		vorgelegt am: \today \\
%		\vspace{2.5cm}
%		\large \textbf{at the Lehr- und Forschungsgebiet Informatik 2}\\
%		\large \textbf{Rheinisch Westfälische Technische Hochschule Aachen}\\
%		\vspace{3cm}
%	\end{center}
%	\normalsize{
%		\begin{tabular}{ll}
%			Name: & {Timo Bergerbusch} \\
%			Matrikelnummer: & {344408} \\
%			Studiengang: & Informatik\\
%			Studienjahrgang: & 2017\\
%			Erstgutachter: & {Jera Hensel} \\
%			Zweitgutachter: & {Prof. Dr. Noll} \\
%		\end{tabular}\\
%	}
%\end{titlepage}

\title{ \normalsize \textsc{Bachelor Thesis}
	\\ [2.0cm]
	\HRule{0.5pt} \\
	\LARGE \textbf{\uppercase{Geometric Non-Termination Arguments for Integer Programs}}
	\HRule{2pt} \\ [0.5cm]
	\normalsize \today \vspace*{5\baselineskip}}

\date{}

\author{
	Timo Bergerbusch\\ 
	Rheinisch Westfälisch Technische Hochschule Aachen \\
	Lehr- und Forschungsgebiet Informatik 2  \\ \\
	First Supervisor: Prof. Dr. Jürgen Giesl \\
	Second Supervisor: apl. Prof. Dr. Thomas Noll \\
	Advisor: Jera Hensel
	}

\maketitle % English cover

%\clearpage
{\bf\Large Acknowledgement} \\ [1em] 

First, I would like to thank Prof. Dr. Jürgen Giesl for giving me the opportunity to work on a ongoing and relevant topic.
Secondly I would like to thank apl. Prof. Dr. Thomas Noll for agreeing to be the second supervisor of my thesis.

Thirdly I would like to thank Jera Hensel, who supervised me during my thesis. I want to thank her for the many patient answers she gave my no matter how obvious the solution was. She not only answered my questions, but also got proactive herself and helped me creating better results by pointing out my failures and encouraging me during the whole process. Also I want to thank her for the possibility to write the underlying program the way I wanted to without any restrictions or limits regarding the way of approaching the topic. 

Also I want to thank my girlfriend Nadine Vinkelau and all my friends, who encouraged me during my whole studies and not only accepting that I often was short on time, but also strengthen my back during the whole process. Especially I want to thank my good friend Tobias Räwer, who explained many topics to me over and over again throughout the whole bachelor to help me pass my exams without demanding anything in return. Thanks to his selfless behaviour I got this far within only three years.

Finally I want to thank my parents for giving me the possibility to fulfil my desire to study at a worldwide known university. Without the financial support I would not have had this opportunity. \\ \\

\paragraph{Erklärung} Ich versichere hiermit, dass ich die vorliegende Arbeit selbstständig verfasst und keine
anderen als die angegebenen Quellen und Hilfsmittel benutzt sowie Zitate kenntlich
gemacht habe.\newline \\
Aachen, den \today

\begin{tabular}{lp{2em}l} 
	\hspace{4cm} \\\cline{1-1}\cline{3-3} 
	Timo Bergerbusch
\end{tabular}


%\clearpage
\newpage
{\bf\Large Abstract} \\ [1em] 

% motivation
The topic of program termination analysis undergoes a significant importance increase owed to the expansion of software usage throughout everyday life. Since fixing problems caused by software bugs leads to an overhead of support, the initial guarantee of correctness can save time spent on fixing these problems. Therefore the research of automated assisting during the engineering of large programs is a growing field.
% problem statement
One major point of a correct program is determined by its termination, which means that it reaches a final state after finitely many steps. Even though such a tool can never provide soundness in every condition since it would have to solve the \textit{halting problem}, which is proven by Turing to be undecidable, a variety of tools addressing this problem exist, for example \aprove. \aprove tries to prove (non-)termination for as many programs possible, although not all programs can be handled.\newline \\
% approach
In this thesis we extend the possibilities of proving \nonterm using \aprove by a special set of programs based on the approach described in \cite{leike2014geometric} by Jan Leike and Matthias Heizmann. Altering the underlying structure from linear loop programs to integer transition systems (ITS) we prove \nonterm using a \textit{\gna (GNA)} derived from the program itself. By the usage of linear algebra and \solver we are able to prove the existence of a GNA, which results in a proof of \nonterm of the integer transition system. Using this technique as an additional approach in \aprove increases its power power to prove \nonterm. 
%We restrict ourselves to only have linear updates of the variables in order to be able to apply the underlying approach, which provides correctness and soundness for this particular set of programs.\newline
\\
% results
%TODO: rewrite sentence
As a result we will see that the implemented technique provides the desired mechanism of proving \nonterm for the considered programs under certain limitations, which are reasoned by the complexity of integer transition systems in general.

 One restriction is the existence of a start term within the system, which is mandatory to apply the definition of a GNA. Further the handling of newly introduced variables within the systems are only very basic, since correctness of using division on integers is not generally given. \newline
 %%%%%%%
\\
% conclusion
In summary we can say that the use of GNAs is a promising approach to prove \nonterm of \itss, as it is for linear lasso programs. The restriction to only use linear updates and its consequences regarding modern programs need further investigation to evaluate the applicability in real industrial software.
 

\tableofcontents

% Ab erstem Kapitel Seiten arabisch zählen
\setcounter{page}{1}
\pagenumbering{arabic}

\chapter{Introduction}

\section{Motivation}
The topic of verification and termination analysis of software increases in importance with the development of new programs. Even though that for Touring Complete programming languages the \textit{halting problem} is undecidable, and therefore no complete and sound method can exist, a verity of approaches to determine termination are researched and still being developed. These approaches can determine termination on programs, which match certain criteria in form of structure, composition or using only a closed set of operations for example only linear updates of variables. \newline
Given a tool, which can provide a sound and in many scenarios applicable mechanism to prove termination, an optimized framework could analyse written code and find bugs before the actual release of the software \cite{verschaetse1993automatic}. Contemplating that automatic verification can be applied to termination proved software the estimated annual US Economy loses of \$60 billion each year in costs associated software maintenance could be reduced significantly \cite{zhivich2009real}. \newline

\section{\emph{AProVE}}
\label{sec:aprove}
One promising approach is the tool \aprove (\underline{A}utomated \underline{Pro}gram \underline{V}erification \underline{E}nvironment) developed at the RWTH Aachen by the Lehr- und Forschungsgebiet Informatik 2. The \emph{AProVE}-tool (further only called \aprove) for automatic termination and complexity proving works with different programming languages of major language paradigms like \tool{Java} (object oriented), \tool{Haskell} (functional), \tool{Prolog} (logical) as well as \tool{rewrite systems}.
\begin{figure}[H]
	\centering
	\begin{tikzpicture}
		\node[aproveNode] at (0,0) (java) {Java};
		\node[aproveNode]at (0,-.75) (c) {C};
		\node[aproveNode] at (0,-1.5) (haskell) {Haskell};
		\node[aproveNode] at (0,-2.25) (prolog) {Prolog};
		\node[shape = circle, draw,align = center] at (3, -1.125) (seg) {Symbolic\\ Execution\\ Tree};
		\node[stdNode] at (6,-.75) (its) {\its};
		\node[stdNode, minimum width = 3cm] at (10,-.25) (comp) {Complexity};
		\node[stdNode, minimum width = 3cm] at (10,-1) (term) {Termination};
		\node[stdNode, minimum width = 3cm] at (10,-1.75) (nterm) {Non-Termination};
		
		\draw[thickarrow] (java) edge (seg);
		\draw[thickarrow] (c) edge (seg);
		\draw[thickarrow] (haskell) edge (seg);
		\draw[thickarrow] (prolog) edge (seg);
		
		\draw[thickarrow] (seg) edge (its);
		\draw[thickarrow] (seg) edge (nterm.west);
		
		\draw[thickarrow] (its) edge (term);
		\draw[thickarrow] (its) edge (comp);
		\draw[thickarrow] (its) edge (nterm);
		
		\draw [thick,decoration={brace,mirror},decorate] (-1,-2.75) -- (3.9,-2.75) 
		node [pos=0.5,anchor=north,yshift=-0.3cm] {\footnotesize Frontends}; 
		
		\draw [thick,decoration={brace,mirror},decorate] (5.2,-2.75) -- (11.5,-2.75) 
		node [pos=0.5,anchor=north,yshift=-0.3cm] {\footnotesize Backend}; 
	\end{tikzpicture}
	\caption{Schematic partition of the derivation process of \aprove adapted from \cite{giesl2017analyzing}}
	\label{fig:aprove-graph}
\end{figure}
\aprove is able to unify different languages into one structure by converting programs of specific languages like \tool{C} into \textit{Low Level Virtual Machine(\llvm)}-code using the tool \tool{Clang} \footnote{further information: \url{https://clang.llvm.org/}}. Among others these \llvm-programs can be converted into a so called \seg. This graph represents all possible computations of the input program. If this graph contains \lasso's, which are strongly connected components (SCC) and the corresponding path from the root to the SCC, \aprove derives so called (integer) term rewrite systems (further only called \its)\footnote{a mathematical definition can be found within \cite{fuhs2009proving}}. By adding conditions to the \its rules the solution space gets restricted and therefore the \its under-approximates. From that it is proven that the non-termination of (at least) one \its implies non-termination of the program. A more detailed description of the process is stated in \cite{hensel2017aprove} \newline
The conversion of different languages into \its and subsequently applying various different approaches is what makes this tool strong in meanings of proofing \cite{giesl2017analyzing}.

\section{Overview}
\label{sec:overview}
This paper provides the introduction to the topic of termination analysis. We focus on the very basic steps, because of the huge variety of possible approaches and related methods. Any further knowledge about termination analysis techniques and how they are applied within \aprove can be found in the related papers \cite{giesl2017analyzing}, \cite{giesl2006aprove}, \cite{giesl2003aprove}.\newline
Within \Cref{chapter:preliminaries} some preliminaries used within the paper are defined to create a well-defined base for any further argumentation and derivation. It covers the most essential definition of the this thesis, which is the nontermination, topics of basic knowledge about \textit{Integer Term Rewrite Systems} and it's within this approach considered subset based on it's structure. Also the definition of the \gna, which builds the main constituent, and any strongly related matrices are defined. Also we define a tree-structure, which we use to handle arithmetical terms containing variables. Last we take a glimpse at the topic of \tool{SMT}-solving and declare the essential parts used within the implementation of the approach.\newline
The main chapter, which is \Cref{chapter:geo-non-term}, deals with the derivation of the \stem part, for constant or variable terms, the derivation of the \loopt with all it's matrices and finally the derivation of the \tool{SMT}-Problem, which provides a \gna if it exists. \newline
At the end, we want to take a look at the usability of the approach itself. Also we want to point out possible adaptations and improvements of the implementation of this approach.
\chapter{Preliminaries}
\label{ch:preliminaries}

%\begin{itemize}
%	\item what is the general form of the considered Programs
%		\begin{itemize}
%			\item single-loop
%			\item linear updates
%			\item (single guard)
%			\item What is considered \emph{STEM}
%			\item What is considered \emph{LOOP}
%		\end{itemize}
%	\item The different Matrices
%	\item the theorem
%	\item the Reverse-Polish-Notation-Tree
%	\item SMT-Problem
%\end{itemize}
In order to be able to explain the solution approach we have to declare to, which programs are considered within the Geometric Nontermination. Furthermore we have to define a few structures we work on.

\section{Geometric Nontermination Argument (GNA)}
Adapted from Jan Leikes and Matthias Heizmanns paper \textit{Geometric Nontermination Arguments} \cite{leike2014geometric} I will define the considered programs, define the \stem and \loopt and finally state the definition of Geometric Nontermination Arguments.

\subsection{Considered Programs}
The considered programs in the Geometric Nontermination are not bound to a special programming language. The paper works on so called Linear-Lasso Programs, which in fact %TODO: really?
are also used within \aprove to derive the so called (int-)TRS. Because of the, within the \hyperref[sec:aprove]{introduction} stated, conversion of the language into \textit{llvm}-code and further analysis the applicability of Geometric Nontermination Arguments are not bound to any program language. \newline %TODO: llvm? 
In order to define the specific conditions under which we can use the approach, we take the language \tool{Java} as an example.
\subsection{Structure}
The structure of the considered programs is quite simple. They contain an optional declaration of the used variables and a \code{while}-loop. Even though \tool{Java} would not accept this the conversion to \tool{llvm} would still be sound. An example of a fulfilling \tool{Java} program is shown in \autoref{fig:structure-example-java}. 
\begin{itemize}
	\item The \stem: \newline
		The initialization and optional declaration of variables used within the \code{while}-loop. In the example line 3 and 4 are considered the \stem. Also only $b$ is declared.
	\item The guard: \newline
		The guard of the \code{while}-loop is essential to restrict $a$  as we will see in %TODO: ref
		. With the restriction of $a+b\ge 4 $ we can prove termination for $a < 3$ without further analysis, and also to prove termination assume that $a \ge 3$.
	\item The linear Updates: \newline
		The updates of the variables within the \code{while}-loop are the most essential part for termination, since their value determine if the guard still holds. The approach works with only linear updates of the variables, so for every variable $v_i$ where $1\le i\le n$ we can have a $f(v_i)=a_1*v_1+...+a_n*v_n$ with $n \in \mathbb{N}$. Note since we work on int-TRS it is sufficient for $n$ to be in $\mathbb{N}$. 	
\end{itemize} 

\begin{figure}[h]
	\begin{lstlisting}[language = java, escapechar = !, linewidth=0.6\linewidth]
	int main(){
		
		int a;!\tikz[remember picture] \node [] (a) {};!
		int b=1;!\tikz[remember picture] \node [] (b) {};!
		
		while(a+b>=4){! \tikz[remember picture] \node [] (c) {}; !
			a=3*a+b;!\tikz[remember picture] \node [] (d) {}; !
			b=2*b;!\tikz[remember picture] \node [] (e) {}; !
		}
	}		
	\end{lstlisting}
	\begin{tikzpicture}[remember picture, overlay, 
		every edge/.append style = { ->, thick, >=stealth, dashed, line width = 1pt },
		every node/.append style = { align = center, minimum height = 10pt,	font = \bfseries, fill= green!20},
	text width = 2.5cm ]
		\node[above right = .2cm and 3.1 cm of a] (A) {the \textit{STEM}};
		\draw[rounded corners=5pt] (A.west) edge (a.east);
		\draw (A.west) edge (b.east);
		
		\node[below = of A, right = of c] (B) {the guard};
		\draw (B.west) edge (c.east);
		
		\node[right = 4cm of d, below = of B] (C) {the linear update};
		\draw (C.west) edge (d.east);
		\draw (C.west) edge (e.east);
	\end{tikzpicture}
%	\lstinputlisting{src/listings/preliminaries-consideredPrograms-example-java.java}		
	\caption{A \tool{Java} program fulfilling the conditions to be applicable}
	\label{fig:structure-example-java}
\end{figure}
The guard and linear updates together form the so called \loopt. 

After ...
%TODO: was geschieht mit dem c programm normalerweise?
%TODO: die 2 statt 1 erklären
we finally receive the equivalent \tool{int-TRS} shown in \autoref{fig:structure-example-TRS}. As we can see the original program can be recognized quite easily. The first rule in line 1 denotes the \stem, while the second line equals the loop \loopt.

\begin{figure}[H]
	\begin{lstlisting}[linewidth=1.4\textwidth, escapechar = !]
!$\overbrace{f_1	     -> f_2(1+3*c,2)   :|: c>2 \&\& 8<3*c}^{\text{\stem}}$!
!$\underbrace{\underbrace{f_2(a,b) \rightarrow f_2(3*a+b,2*b)}_{\text{linear update}} :|: \underbrace{3*a>29 \& \& a+b>11 \& \& 31<3*a+b \& \& 3<2*b}_{\text{guards}}}_{\text{\loopt}}$!
	\end{lstlisting}
	\caption{The \tool{int-TRS} corresponding to the \tool{Java} program in \autoref{fig:structure-example-java}}
	\label{fig:structure-example-TRS}
\end{figure}
 Neglecting the conditional terms for now the declaration of $b$ is set in line 1 obviously to 2, because of the one circle the GRAPH has to compute in order to find a loop. The definition of $a$ is more difficult and will be shown within REF %TODO: ref to derive STEM
 . 
Also the update within line 2 is the same as in \autoref{fig:structure-example-java} line 7 and 8. \newline

\begin{definition}[Guard Matrix, Guard Constants]
	Let $n \in \mathbb{N}$ be the number of distinct variables, $v_i$ $1 \le i \le n$ the $i$-th distinct variable name , $m \in \mathbb{N}$ be the number of guards, $r_i$ $1 \le i \le m$ the $i$-th guard, $a_{i,j} \in \mathbb{N}$ $1\le i \le n$, $1 \le j \le m$ the factor of $v_i$ in $g_j$ and $c_i$ be the constant term within $r_j$. \newline
		
	Then the \guardmatrix $G \in \mathbb{N}^{m\times n}$ is defined as $G_{i,j}=a_{i,j} $ and \guardconstants $g \in \mathbb{N}^m$ are defined as $g_i = c_i$.
\end{definition}
\begin{example}
	The corresponding \guardmatrix to \autoref{fig:structure-example-TRS} is $G = \begin{pmatrix} 3 & 0 \\ 1 & 1 \\ 3 & 1 \\ 0 & 2 \end{pmatrix}$ and the \guardconstants is $g= \begin{pmatrix} 29 \\ 11 \\ 31 \\ 3 \end{pmatrix}$
\end{example}



\section{Reverse-Polish-Notation-Tree}

\section{\emph{SMT}-Problem}

\chapter{Geometric Non-Termination}

Now that all preliminaries are stated we can start looking how the approach works within \aprove. To find a \gna and so prove nontermination we use \aprove to generate an \its of a given program. %TODO: write how? ref: prelim
Based on the calculated \its we derive the \stem, the \loopt and then generate an \tool{SMT}-Problem using \autoref{def:gna} and compute a \gna, which would be a prove of nontermination, or state that no \gna can exist, which does not infer termination nor nontermination.

\section{Derivation of the \emph{STEM}}
\label{sec:stem}
The derivation of the \stem is the first step to do in order to derive a \gna. As described in \autoref{sec:structure} the \stem defines the variables before iterating through the \loopt.  Owned to the fact, that \aprove has to find the a loop within the generated symbolic execution graph %TODO: cite
one iteration through the \loopt will be calculated. Obviously this does not falsify the result. If it does not terminate i will still not terminate after one iteration and if it terminates after $n$ iterations and we compute one it will still terminate after $n-1$ iterations. \newline
Within the derivation of the \stem we distinguish between two cases discussed in the following sections.

\subsection{Constant \stem}
\label{sec:stem-const}

\newsavebox{\stemexone}% Box to store smallmatrix content
\savebox{\stemexone}{$\begin{pmatrix}10\\2\end{pmatrix}$}
\begin{wrapfigure}{r}{0.6\textwidth}
	\begin{lstlisting}[escapechar=!]
	!$f_1 \rightarrow f_2(10,2) :|: TRUE $!
	\end{lstlisting}	
	\caption{An example of a constant \its rule to derive the \stem. The \stem in this case would be \usebox{\stemexone} }
	\label{lst:stem-cons}
\end{wrapfigure}

The constant stem is the easiest case to derive the \stem from. It has the form: 
%\vspace{-1em}
\begin{figure}[H]
	$f_x \rightarrow f_y(c_1,\dots c_n) :|: TRUE$
\end{figure} 
%\vspace{-2em}
An example of a constant \stem is shown in \autoref{lst:stem-cons}. 
The values of $x$ can be directly read from the right hand side and need no further calculations.

\subsection{Variable \stem}
\label{sec:stem-var}
The more complex case is given if the start function symbol has the following form:
\begin{figure}[H]
	\hspace{2cm}
	$f_x \rightarrow f_y(v_1, \dots v_n) :|: cond$
\end{figure}
%\vspace{-1em}
where $v_i$ $1 \le i \le n$ is either a constant term like in \autoref{sec:stem-const} \underline{or} a variable defined by the $cond$ term. An example for such a \stem is shown in \autoref{lst:stem-var}. In order to derive terms in $\mathbb{Z}$ an \tool{SMT}-Problem needs to be solved. We can compute the \guardmatrix, \guardconstants, \updatematrix and \updateconstants of the start function symbol and use the \smtfactory, which we will explain in %TODO: ref
, to create the assertions leading to either an assignment of $x$ to a value or to a unsatisfiable core. Such a core would state, that the \code{while}-Loop would not hold after any assignment and therefore prove termination.
%TODO: im code: wenn unsat dann termination

\newsavebox{\stemextwo}% Box to store smallmatrix content
\savebox{\stemextwo}{$\begin{pmatrix}1+3*3\\2\end{pmatrix}$}
\newsavebox{\stemextwosecond}% Box to store smallmatrix content
\savebox{\stemextwosecond}{$\begin{pmatrix}10\\2\end{pmatrix}$}
\begin{figure}[H]
	\begin{lstlisting}[escapechar=!]
	!$f_1 \rightarrow f_2(1 + 3 * v, 2) :|: v > 2\text{ \&\& }8 < 3 * v $!
	\end{lstlisting}	
	\caption{An example of a variable \its rule to derive the \stem. In order to derive it an $v$ fulfilling the conditions need to be found using an \tool{SMT}-Solver. Since $v=3$ is the first number in $\mathbb{Z}$ that satisfies the guards the \stem would be \usebox{\stemextwo}$=$\usebox{\stemextwosecond} }
	\label{lst:stem-var}
\end{figure}

\section{Derivation of the \emph{LOOP}}

The derivation of the \loopt is pretty straight forward applying \autoref{def:guard}, \autoref{def:update} to a looping rule and then computing \iterationmatrix and \iterationconstants using \autoref{def:iteration}. \newline
Let $f_x$ be the starting function symbol given by the \its and $r_i$ be a rule, with $f_x

\subsection{The Update Matrix}

\subsection{The Guard Matrix}

\subsection{The Iteration Matrix}

\section{Derivation of the \emph{SMT}-Problem}

\subsection{The Domain Criteria}

\subsection{The Initiation Criteria}

\subsection{The Point Criteria}

\subsection{The Ray Criteria}

\section{Verification of the Geometric Non-Termination Argument}
	 

\chapter{Evaluation and Benchmarks}
\label{chapter:eval}
In this chapter we want to take a look at the implementation and evaluate if the approach is useful in terms of applicable cases or if the approach works only on uncommon preconditions. \newline
Also we want to take a look at the benchmarks of the implementation in computational efficiency. \newline
Further we want to outline possibilities to extend and improve the implementation and and current problems in cases where an efficient solution is not quite obvious.

\section{Evaluation of the Approach}
The approach provides a sound solution to specific types of programs. Given a tool like \aprove which provides an \its, the further computation that has to be done can be solved quite efficiently. Using a state-of-the-art \solver and the definition of $\lambda_i$ to be the $i$-th eigenvalue, the problem can also be resolved efficiently for given $\mu_i$. If the $\mu_i$ are not given, the problem is undecidable, which makes it still useful within \aprove but not as strong as before. \newline
Since the original approach does not mention equalities within the guards, the substitution of newly introduced variables as handled within \Cref{sec:derivation-guard} is necessary in order to apply the definition. Such a substitution is very costly in computation time and also highly error prone.

\section{Benchmarks}
\label{sec:benchmarks}
In order to discuss the possibilities of improvement regarding the implementation, we want to take a look at the performance. For that we compared \aprove with all its techniques to \aprove only running the \gnanal. The set of example programs is taken from the Termination category of the SV-COMP \cite{SVComp}.\newline
Detailed results of all programs can be found in \cite{Benchmark}. Not all of the programs are non-terminating and also the set includes non-terminating programs, which do not meet the stated preliminaries. An overview is given below.

\begin{figure}[H]
	\centering
	\begin{tabular}{rlrl}
		number of programs 	& 847 & $\varnothing$-time for termination & +0.0078 sec \\
		applicable programs & 53 & $\varnothing$-time for \nonterm & -1.2675 sec \\
		false pos./neg. & 0 & $\varnothing$-time save overall & 0.76 sec \\
	\end{tabular}	
\end{figure}

Regarding the average duration it takes \aprove to prove these programs, which is 4.513 seconds, an average time saving of 0.76 seconds, which is 17 percent, is observed. Keeping in mind that the \gnanal has the objective of proving \nonterm and that \aprove uses approaches in parallel the save can be even bigger. \aprove stops if any approach returns a proof. So if more than one proof is found, all but the fastest proof do not affect the average saving of time. \newline

\begin{figure}[H]
	\begin{lstlisting}[language = java]
	int main(void){
		int a = __VERIFIER_nondet_int();
		int b = __VERIFIER_nondet_int();
		int c = __VERIFIER_nondet_int();
		
		while (a+b+c >= 4) {
			a = b;
			b = a+b;
			c = b-1;
		}
		return 0;
	}
	\end{lstlisting}
	\caption{A program \aprove cannot prove \nonterm for without the \gnanal}
	\label{fig:program-newproof}
\end{figure}
Also there are programs, like the program shown in \Cref{fig:program-newproof}, which \aprove cannot prove \nonterm for without using the \gnanal. Including this technique to the set of approaches \aprove is able to prove \nonterm. So the set of provable programs gets extended and improves \aprove.


\section{Possible Improvement of the Implementation}
The implementation of the approach is fully functional under the circumstances mentioned, like for example the defined structure in \Cref{sec:structure}. Nevertheless also this implementation has certain cases in which it does not perform as efficient as it could, or where it can be improved in terms of applicability.
So here we state the possible improvements of the implementation to make it applicable to more cases and therefore stronger or more efficient.
\subsection{Choosing a Logical Fragment}
As already stated in \Cref{sec:derivation-smt} the difficulty of the $\mu_i$ leads to a shift into undecidability, since variable multiplication on integers is non-linear. Also mentioned in \Cref{sec:derivation-smt} there are a bunch of approaches which lead to semi-decidability and therefore to the possibility to still use the variable multiplication within the problem if the $\mu_i$ can be restricted to a finite domain.\newline
A possible improvement could be an iteration approach over different values of the $\mu_i$. The number of problems that have to be solved would blow up but the problems themselves would be decidable.\newline
For the examples of the International Competition on Software Verification, we estimate the method that we chose more suitable. In general, a reliable case study could give an indication of the advantages and disadvantages of the two approaches.
%A reliable case study of a large set of examples could underline the necessity of the iteration since we wouldn't be able to derive a \gna using the \qfnia. It could also lead to the overhead of computational cost using the iterative method, which can be useful if the problem does not have any time restrictions of the deciding process, but is not suitable within the competitions \aprove participates, like the \textit{International Competition of Termination Tools} or \textit{International Competition on Software Verification} \cite{aproveWebsite} \cite{TermComp} \cite{SVComp}. \newline
%The \textit{Termination Competition 2017}, which is organized by the \textit{International Competition of Termination Tools}, for example has a time limit of 300 seconds and only allows 4 core usage, which makes an iterative method very costly \cite{wiki2017termComp}.

\subsection{The \stem derivation}
Within this thesis we considered the \stem to be derived from the \its without any dependence to the \loopt. Based on a specific \stem we try to derive a \gna. A stronger approach would be if the \stem would included in the derivation of a \gna. So we combine the set of assertions from \Cref{sec:stem-var} and add them to the \solver of \Cref{sec:derivation-smt}. Not restricting the \gna to one particular \stem we would consider any \stem that fulfils the conditions. This could prove \nonterm for programs, which have a \gna for a \stem that is not the possible first \stem given from the solver as used in this thesis.

\subsection{\its program structure}
\label{sec:structure-improvement}
As stated in \Cref{sec:structure} we restricted our implementation to \itss of the form 
\begin{figure}[H]
	\begin{lstlisting}[escapechar=!]
	!$f_x \qquad\qquad \>\> \rightarrow f_y (v_1, \dots v_n) \> :|: cond_1$!
	!$f_y(v_1, \dots v_n) \> \rightarrow  f_y \>(v^\prime_1,\dots v^\prime_n)  :|: cond_2$!
	\end{lstlisting}
\end{figure}
This obviously is a restriction, because \itss of the form
\begin{figure}[H]
	\begin{lstlisting}[escapechar=!]
	!$f_x \qquad\qquad \>\ \rightarrow f_y (v_1, \dots v_n) \>\>\> :|: cond_1$!
	!$f_{y_1}(v_1, \dots v_n) \> \rightarrow  f_{y_2} \>(v^\prime_1,\dots v^\prime_n)  :|: cond_2$!
			!$\vdots$!
	!$f_{y_k}(v_1, \dots v_n) \> \rightarrow  f_{y_1} \>(v^\prime_1,\dots v^\prime_n)  :|: cond_{k+1}$!
	\end{lstlisting}
\end{figure}
could possibly be considered by compressing the rules into one rule using multiple equalities and concatenation of the conditional terms. Analysing such \itss would make the implementation much stronger.
\\
Another possible variation of the considered \itss could be of the form
\begin{figure}[H]
	\begin{lstlisting}[escapechar=!]
	!$f_x \qquad\qquad \>\>\>\> \rightarrow f_y (v_1, \dots v_n) \>\> :|: cond_1$!
	!$f_{y_1}(v_1, \dots v_n) \> \rightarrow  f_{y_2} (v^\prime_1,\dots v^\prime_m)  :|: cond_2$!
	!$f_{y_2}(v_1, \dots v_m)  \rightarrow  f_{y_1} (v^\prime_1,\dots v^\prime_n) \> :|: cond_3$!
	\end{lstlisting}
\end{figure}
where $m \ne n$ but the values $v^\prime_i$, $1 \le i \le m$, are computed as linear updates of the values $v_j$, $1 \le j \le n$.

These are only two extensions of the considered structure, which would be also recommended to implement in order to create a more universally applicable method.

\subsection{\rpntree}
In \Cref{sec:rpntree} we defined the \rpntree, on which we base the arithmetic computations and statements. \aprove also uses a tree structure to handles such statements with. We chose \rpntrees structure because of two reasons:
\begin{enumerate}
	\item The structure \aprove uses is much more complex but also much more powerful, which made programming a lot more difficult. Parsing it into a tree which can only contain elements expected to be in such expressions not only works equivalently, but it also prevents errors if \aprove's structure gets extended or changed. The \code{RPNTreeParser} handles the conversion and therefore can be seen as an adapter which filters every \its that must not occur in geometric non-termination analysis as stated in this thesis.
	\item Many algorithms are difficult to implement if not programmed recursively. Since extending the existing classes was no option, and inheritance would not work because of restricted visibility, creating my own structure was a simple workaround.
\end{enumerate}
The examples we tested have been small enough to not create any problems with the conversion and possibly less efficient methods, but if applied to huge problems a converting of the approach to work on the structure \aprove uses would be the better way.
\chapter{Related Work}
\label{chapter:related-work}

The research of Jan Leike and Matthias Heizmann within the field of geometric \nonterm is the base of various different approaches. These two researchers from the Australian National University and University of Freiburg wrote the \gna-Paper this paper relies on. Their definition of the \gna approach and their proof of soundness is the base of all computation explained throughout \Cref{chapter:preliminaries} and \Cref{chapter:geo-non-term}  \cite{leike2014geometric}. \newline
In this context also \tool{Ultimate LassoRanker} should be mentioned. It is a tool for termination and \nonterm arguments for linear lasso programs by Jan Leike and Matthias Heizmann also containing \gna's  \cite{LassoRanker}.\newline
Further A. Tiwari considered linear loop programs not over the naturals but over the real numbers. For such programs he proved that termination in fact is decidable if the condition of only strict guards is met. This means, that every condition is of the form $\varphi < c$ or $\varphi > c$. \cite{tiwari2004termination}. \newline
Another interesting related work is written by R. Rebiha et al., which contains the relaxation of the eigenvalues to not range over the naturals but to range over the real numbers \cite{rebiha2014characterization}. Also J. Ouaknine defined an approach to decide termination of integer lasso programs in those cases where the corresponding \updatematrix is diagonalizable \cite{ouaknine2014termination}. \newline
An alternative \nonterm approach is given by the work of C. David et al., a technique also applicable to non-deterministic programs. It uses a constraint-based synthesis of recurrence sets, which are defined by A. Tiwari in \cite{tiwari2004termination}. It also supports second-order theory for bitvectors. This approach can be used to find non-terminating lassos that do not have a \gna. The downside of this extension is the problem of solving an $\exists\forall\exists$-constraint \cite{david2015unrestricted} \cite{leike2014geometric}.

Last but not least the often mentioned tool \aprove is to be mentioned. \aprove already uses two different \nonterm approaches. \cite{tacas} 
As stated in \Cref{sec:benchmarks} the approach of this thesis not only leads to a saving of time for applicable programs, it also extends \aprove's power to prove \nonterm. 

\addcontentsline{toc}{chapter}{Literaturverzeichnis}

%Bib
\bibliographystyle{alpha}
\bibliography{src/bib/Literatur}

% Begin Anhang
\appendix
%\input{src/tex/appendix_docu}

\end{document}
