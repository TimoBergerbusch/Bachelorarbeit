\chapter{Related Work}
\label{chapter:related-work}

The first related source that should be mentioned is the research of Jan Leike and Matthias Heizmann. These two researchers from the Australian National University and University of Freiburg wrote the \gna-Paper this paper relies on. Their definition and proof of the \gna in combination with the derivation on \lasso-programs is the base of all computation explained throughout \Cref{chapter:preliminaries} and \Cref{chapter:geo-non-term}. \cite{leike2014geometric} \newline
In that context also \tool{Ultimate LassoRanker} should be mentioned. A tool for termination and nontermination arguments for linear lasso programs by Jan Leike and Matthias Heizmann also containing \gna's. \cite{LassoRanker}\newline
Further A. Tiwari considered linear loop programs not over the naturals, but over the reals. For such programs he proofed that they in fact are decidable in terms of termination if they met the condition of only strict guards. \cite{tiwari2004termination} \newline
An other interesting related work is written by R. Rebiha et al., which contains the generalization of the eigenvalues not only to be natural but also can be within the reals. \cite{rebiha2014characterization} Also J. Ouaknine generalized from the basic approach by mentioning integer lasso programs, where the corresponding \updatematrix is diagonalizable. \cite{ouaknine2014termination} \newline
An extension of the \gna approach can be found within the work of C. David et al. providing a technique also applicable to non deterministic programs. It uses a constraint-based synthesis of recurrence sets, which are apart from others defined by A. Tiwari in \cite{tiwari2004termination}. Also it can work with second order theory for bit vectors. These can be used to find nonterminating lassos, which do not have a \gna. The downside of such an extension is the problem of solving an $\exists\forall\exists$-constraint. \cite{david2015unrestricted} \cite{leike2014geometric}

Last but not least the often mentioned tool \aprove is to be mentioned. Using a verity of techniques including lasso's within a range of proofing attempts this tool provides a series of promising techniques implemented. 

%TODO: vergleich aprove vorher