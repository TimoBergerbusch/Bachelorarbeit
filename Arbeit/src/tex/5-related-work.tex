\chapter{Related Work}
\label{chapter:related-work}

The research of Jan Leike and Matthias Heizmann within the field of geometric \nonterm is the base of various different approaches. These two researchers from the Australian National University and University of Freiburg wrote the \gna-Paper this paper relies on. Their definition of the \gna approach and their proof of soundness is the base of all computation explained throughout \Cref{chapter:preliminaries} and \Cref{chapter:geo-non-term}  \cite{leike2014geometric}. \newline
In this context also \tool{Ultimate LassoRanker} should be mentioned. It is a tool for termination and \nonterm arguments for linear lasso programs by Jan Leike and Matthias Heizmann also containing \gna's  \cite{LassoRanker}.\newline
Further A. Tiwari considered linear loop programs not over the naturals but over the real numbers. For such programs he proved that termination in fact is decidable if the condition of only strict guards is met. This means, that every condition is of the form $\varphi < c$ or $\varphi > c$. \cite{tiwari2004termination}. \newline
Another interesting related work is written by R. Rebiha et al., which contains the relaxation of the eigenvalues to not range over the naturals but to range over the real numbers \cite{rebiha2014characterization}. Also J. Ouaknine defined an approach to decide termination of integer lasso programs in those cases where the corresponding \updatematrix is diagonalizable \cite{ouaknine2014termination}. \newline
An alternative \nonterm approach is given by the work of C. David et al., a technique also applicable to non-deterministic programs. It uses a constraint-based synthesis of recurrence sets, which are defined by A. Tiwari in \cite{tiwari2004termination}. It also supports second-order theory for bitvectors. This approach can be used to find non-terminating lassos that do not have a \gna. The downside of this extension is the problem of solving an $\exists\forall\exists$-constraint \cite{david2015unrestricted} \cite{leike2014geometric}.

Last but not least the often mentioned tool \aprove is to be mentioned. \aprove already uses two different \nonterm approaches. \cite{tacas} 
As stated in \Cref{sec:benchmarks} the approach of this thesis not only leads to a saving of time for applicable programs, it also extends \aprove's power to prove \nonterm. 