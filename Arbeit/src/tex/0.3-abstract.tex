{\bf\Large Abstract} \\ [1em] 

% motivation
The topic of program termination analysis undergoes a significantly importance increase owed to the expansion of software usage throughout everyday life. Since fixing problems caused by software bugs lead to an overhead of support the initial guarantee of correctness can save time spend on fixing these problems. Therefore the research of automated assisting during the engineering of large programs is a growing field.
% problem statement
One major point of a correct program is determined by the termination, which means the resulting in a final state after finitely many steps. Even though such a tool can never provide correctness and soundness in every condition since it would have to solve the \textit{halting problem}, which is proven by Touring to be undecidable, a variety of tools addressing this problematic exist. For example \aprove. Restricting the underlying program to certain circumstances tools like \aprove are able to decide termination.\newline \\
% approach
In this thesis we extend the possibilities of proofing nontermination using \aprove with a special set of programs based on the \text{Geometric Non-Termination} approach of Jan Leike and Matthias Heizmann. Altering the underlying structure from linear loop programs to integer term rewrite systems we prove nontermination using a \gna derived from the program itself. Through the usage of linear algebra and \textit{Satisfiability Modulo Theorie} solvers we are able to prove the existence of a \gna which results in a proof of nontermination of the integer term rewrite system. All this implemented as one additional approach within \aprove leads to a more applicable tool. We restrict ourselves to only have linear updates of the variables in order to be able to apply the underlying approach, which provides correctness and soundness for this particular set of programs.\newline
\\
% results
As a result we will see that the implemented technique provides the desired mechanism of proving nontermination for the considered programs under certain limitations, which are reasoned by the complexity of integer term rewrite systems in general. One restriction is the existence of a start term within the system, which is mandatory to apply the definition of a \gna. Further the handling of newly introduced variables within the systems are only very basic, since correctness of using division on integers is not generally given. \newline
\\
% conclusion
In summary we can derive that \textit{Geometric Nontermination} is a promising topic for integer term rewrite system, like it is for linear lasso programs. Such a termination analysis obviously is only applicable for programs computing mathematical procedures, which are restricted to linear updates. The restriction to only use linear updates and it's consequences regarding in modern programs need further investigation to evaluate the applicability in real industrial software.
