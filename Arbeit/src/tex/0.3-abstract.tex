{\bf\Large Abstract} \\ [1em] 

% motivation
The topic of program termination analysis undergoes a significant importance increase owed to the expansion of software usage throughout everyday life. Since fixing problems caused by software bugs leads to an overhead of support, the initial guarantee of correctness can save time spent on fixing these problems. Therefore the research of automated assisting during the engineering of large programs is a growing field.
% problem statement
One major point of a correct program is determined by its termination, which means that it reaches a final state after finitely many steps. Even though such a tool can never provide soundness in every condition since it would have to solve the \textit{halting problem}, which is proven by Turing to be undecidable, a variety of tools addressing this problem exist, for example \aprove. \aprove tries to prove (non-)termination for as many programs possible, although not all programs can be handled.\newline \\
% approach
In this thesis we extend the possibilities of proving nontermination using \aprove by a special set of programs based on the approach described in \cite{leike2014geometric} by Jan Leike and Matthias Heizmann. Altering the underlying structure from linear loop programs to integer transition systems (ITS) we prove nontermination using a \textit{\gna (GNA)} derived from the program itself. By the usage of linear algebra and \solver we are able to prove the existence of a GNA, which results in a proof of nontermination of the integer transition system. Using this technique as an additional approach in \aprove increases its power power to prove nontermination. 
%We restrict ourselves to only have linear updates of the variables in order to be able to apply the underlying approach, which provides correctness and soundness for this particular set of programs.\newline
\\
% results
%TODO: rewrite sentence
As a result we will see that the implemented technique provides the desired mechanism of proving nontermination for the considered programs under certain limitations, which are reasoned by the complexity of integer transition systems in general.

 One restriction is the existence of a start term within the system, which is mandatory to apply the definition of a GNA. Further the handling of newly introduced variables within the systems are only very basic, since correctness of using division on integers is not generally given. \newline
 %%%%%%%
\\
% conclusion
In summary we can say that the use of GNAs is a promising approach to prove nontermination of \itss, as it is for linear lasso programs. The restriction to only use linear updates and its consequences regarding modern programs need further investigation to evaluate the applicability in real industrial software.
