\chapter{Geometric Non-Termination}
\label{chapter:geo-non-term}
Now that all preliminaries are stated we can start looking how the approach works within \aprove. To find a \gna and so prove nontermination we use \aprove to generate an \its of a given program. %TODO: write how? ref: prelim
Based on the calculated \its we derive the \stem, the \loopt and then generate an \tool{SMT}-Problem using \autoref{def:gna} and compute a \gna, which would be a prove of nontermination, or state that no \gna can exist, which does not infer termination nor nontermination.

\section{Derivation of the \emph{STEM}}
\label{sec:stem}
The derivation of the \stem is the first step to do in order to derive a \gna. As described in \autoref{sec:structure} the \stem defines the variables before iterating through the \loopt.  Owned to the fact, that \aprove has to find the a loop within the generated \seg %TODO: cite
one iteration through the \loopt will be calculated. Obviously this does not falsify the result. If it does not terminate i will still not terminate after one iteration and if it terminates after $n$ iterations and we compute one it will still terminate after $n-1$ iterations. \newline
Within the derivation of the \stem we distinguish between two cases discussed in the following sections.

\subsection{Constant \stem}
\label{sec:stem-const}
The constant stem is the easiest case to derive the \stem from. It has the form: 
%\vspace{-1em}
\begin{figure}[H]
	\centering
	$f_x \rightarrow f_y(c_1,\dots c_n) :|: TRUE$
\end{figure} 
%\vspace{-2em}
An example of a constant \stem is shown in \autoref{lst:stem-cons}. 
The values of $x$ can be directly read from the right hand side and need no further calculations.
\newsavebox{\stemexone}% Box to store smallmatrix content
\savebox{\stemexone}{$\begin{pmatrix}10\\2\end{pmatrix}$}
\begin{figure}[H]
	\begin{lstlisting}[escapechar=!]
	!$f_1 \rightarrow f_2(10,2) :|: TRUE $!
	\end{lstlisting}	
	\caption{An example of a constant \its rule to derive the \stem. The \stem in this case would be \usebox{\stemexone} }
	\label{lst:stem-cons}
\end{figure}

\subsection{Variable \stem}
\label{sec:stem-var}
The more complex case is given if the start function symbol has the following form:
\begin{figure}[H]
	\hspace{2cm}
	$f_x \rightarrow f_y(v_1, \dots v_n) :|: cond$
\end{figure}
%\vspace{-1em}
where $v_i$ $1 \le i \le n$ is either a constant term like in \autoref{sec:stem-const} \underline{or} a variable defined by the $cond$ term. An example for such a \stem is shown in \autoref{lst:stem-var}. In order to derive terms in $\mathbb{Z}$ an \tool{SMT}-Problem needs to be solved. We can compute the \guardmatrix, \guardconstants, \updatematrix and \updateconstants of the start function symbol and use the \smtfactory, which is explained within \autoref{sec:smt-problem}, to create the assertions leading to either an assignment of $x$ to a value or to a unsatisfiable core. Such a core would state, that the \code{while}-Loop would not hold after any assignment and therefore prove termination.
%TODO: im code: wenn unsat dann termination

\newsavebox{\stemextwo}% Box to store smallmatrix content
\savebox{\stemextwo}{$\begin{pmatrix}1+3*3\\2\end{pmatrix}$}
\newsavebox{\stemextwosecond}% Box to store smallmatrix content
\savebox{\stemextwosecond}{$\begin{pmatrix}10\\2\end{pmatrix}$}
\begin{figure}[H]
	\begin{lstlisting}[escapechar=!]
	!$f_1 \rightarrow f_2(1 + 3 * v, 2) :|: v > 2\text{ \&\& }8 < 3 * v $!
	\end{lstlisting}	
	\caption{An example of a variable \its rule to derive the \stem. In order to derive it an $v$ fulfilling the conditions need to be found using an \tool{SMT}-Solver. Since $v=3$ is the first number in $\mathbb{Z}$ that satisfies the guards the \stem would be \usebox{\stemextwo}$=$\usebox{\stemextwosecond} }
	\label{lst:stem-var}
\end{figure}

\section{Derivation of the \emph{LOOP}}
\label{sec:loop}
The derivation of the \loopt is pretty straight forward applying \autoref{def:guard}, \autoref{def:update} to a looping rule and then computing \iterationmatrix and \iterationconstants using \autoref{def:iteration}. \newline
Let $f_x$ be the starting function symbol given by the \its and $r_i$ be a rule, with 
\begin{lstlisting}[escapechar=!]
	!$f_x \rightarrow f_y(v_1, \dots, v_n) :|: cond_1$!
\end{lstlisting} 
then we take the in lexicographical order first rule $r_l$ of the form 
\begin{lstlisting}[escapechar=!]
	!$f_y(v_1, \dots, v_n) \rightarrow f_y(v^\prime_1, \dots, v^\prime_n) :|: cond_2$!
\end{lstlisting}
and compute the \iterationmatrix and \iterationconstants according to $r_l$. 

\subsection{The \updatematrix and \updateconstants}
\label{sec:derivation-update}
The derivation of the \updatematrix and \updateconstants can be achieved by applying the \autoref{def:update} to the given rule $r_l$. For that we create $U$ as the coefficient matrix. The size of $U$ can be determined by adding a column per occurring variable and rows per linear equation of every $v^\prime_i$. To derive the entry's of the matrix we use the \rpntree of the given equation and simply perform recursive searching to derive the factor. 
%Also we can use
The procedure works like the following:
\begin{algorithm}[H]
	\caption{Derivation of a coefficient within an \rpntree}
	\label{algo:coefficient}
	\begin{algorithmic}[1]
		\Function{getCoefficient}{$query$}
			\If{node == query}\Comment{query is the tree}
				\State \Return $1$
			\ElsIf{node does \underline{not} contain query} \Comment{tree does not contain query}
				\State \Return $0$
			\EndIf
			\State
			\If{node represents PLUS} \Comment{Choose the subtree containing the query}
				\If{left side contains $query$}
					\State \Return getCoefficient$(query)$
				\Else
					\State \Return getCoefficient$(query)$
				\EndIf
			\EndIf
			\If{node represents TIMES} \Comment{Retrieve value}
				\If{node.right == query}
					\State \Return node.left.value
				\EndIf				
			\EndIf
		\EndFunction
	\end{algorithmic}
\end{algorithm}

Since we can rely on the usage of the \stdLinInt described in \autoref{sec:its} and therefore we can neglect cases for example that the \textit{left}-child of a \textit{RPNFunctionSymbol} with \textit{arithmeticSymbol} \code{TIMES} is the \textit{RPNVariable} and the \textit{right}-child is the \textit{RPNConstant}.
%TODO: Proof?
An example derivation of a factor using \autoref{algo:coefficient} is shown in \autoref{ex:factor-derivation}.

\begin{figure}[H]
	\begin{tikzpicture}
		\node (Plus) at (0,0) [objDia] {
				\textbf{f1}:RPNFunctionSymbol
				\nodepart{second}arithmeticSymbol: PLUS
			};
		\node (Times1) at (-4, -2 ) [objDia] {
				\textbf{f2}:RPNFunctionSymbol
				\nodepart{second}arithmeticSymbol: TIMES	
		};
		\node (Times2) at (4, -2) [objDia] {
			\textbf{f3}:RPNFunctionSymbol
			\nodepart{second}arithmeticSymbol: TIMES	
		};
		\node (cons1) at (-6, -4) [objDia] {
			\textbf{c1}:RPNConstant
			\nodepart{second}value: 2
		};
		\node (var1) at (-2, -4)[objDia] {
			\textbf{v1}:RPNVariable
			\nodepart{second}varName: x
		};
		\node (cons2) at (2, -4) [objDia] {
			\textbf{c2}:RPNConstant
			\nodepart{second}value: 4
		};
		\node (var2) at (6, -4) [objDia] {
			\textbf{v2}:RPNVariable
			\nodepart{second}varName: y
		};
		\draw[neglected] (Plus.south)  -- ++(0,-0.4) -| (Times1.north) node [pos = 0.4, above, font=\footnotesize]{left};
		\draw[considered] (Plus.south)  -- ++(0,-0.4) -| (Times2.north) node [pos = 0.4, above, font=\footnotesize]{right};
		\draw[thickarrow] (Times1.south)  -- ++(0,-0.5) -| (cons1.north) node [pos = 0.4, above, font=\footnotesize]{left};
		\draw[thickarrow] (Times1.south)  -- ++(0,-0.5) -| (var1.north) node [pos = 0.4, above, font=\footnotesize]{right};
		\draw[considered] (Times2.south)  -- ++(0,-0.5) -| (cons2.north) node [pos = 0.4, above, font=\footnotesize]{left};
		\draw[query] (Times2.south)  -- ++(0,-0.5) -| (var2.north) node [pos = 0.4, above, font=\footnotesize]{right};
	\end{tikzpicture}
	\caption{An example of deriving the coefficient of a given formula and a variable as query. This example uses the \rpntree of \autoref{ex:rpntree} and $y$ as the query. }
	\label{ex:factor-derivation}
\end{figure}

The \color{red}red\color{black}-arrow stands for the neglected left subtree of the root node, which can be neglected because the query is not contained. The \color{blue}blue\color{black}-arrows show the path to the subtree further investigated. The \color{green!50!black}green\color{black}-arrow determines, that the right child node is the query so the left child node has to be the coefficient. Since the underlying update is in \stdLinInt the left subtree has to be a \textit{RPNConstant}.

The \updateconstants can be derived by an simplification of \autoref{algo:coefficient}, since we only have to retrieve the constant term within the tree. The corresponding derivation is given by \autoref{algo:constant-term}.

\begin{algorithm}[H]
	\begin{algorithmic}[1]
		\Function{getConstantTerm}{}
			\If{this is a constant}
				\State \Return this.value
			\EndIf
			\State
			\State $flip \gets 1$
			\If{this represents MINUS}\Comment{flip result in case of prev. negation}
				\State $flip \gets -1$
			\EndIf
			\If{this represents sth. $\ne$ TIMES}
				\State $left \gets left.getConstantTerm()$ \Comment{recursive calls}
				\State $right \gets right.getConstantTerm()*flip$
				\State \Return $left+right$
			\EndIf
		\EndFunction
	\end{algorithmic}
	\caption{Derivation of a constant term within an \rpntree}
	\label{algo:constant-term}
\end{algorithm}
Since a constant $c < 0$ can stored in a constellation shown in \autoref{ex:constant-term-minus} we consider a variable $flip$ to store a sign change occurring for a subtraction. Knowing that the \stdLinInt is used all occurs of a multiplication can be neglected.\newline %TODO: small Proof?
Through the \stdLinInt one of the recursive calls has to be $0$ since only one constant term.

\begin{figure}[H]
	\centering
	\begin{tikzpicture}
		\node (minus) [objDia]{
			\textbf{f1}:RPNFunctionSymbol
			\nodepart{second}arithmeticSymbol: MINUS
		};
		\node[below left = of minus] (var) [objDia]{
			\textbf{v1}:RPNVariable
			\nodepart{second}varName: x
		};
		\node[below right = of minus] (cons) [objDia]{
			\textbf{c1}:RPNConstant
			\nodepart{second}value: 3
		};
		
		\draw[thickarrow] (minus.south)  -- ++(0,-0.5) -| (var.north) node [pos = 0.4, above, font=\footnotesize]{left};
		\draw[thickarrow] (minus.south)  -- ++(0,-0.5) -| (cons.north) node [pos = 0.4, above, font=\footnotesize]{right};
	\end{tikzpicture}
	\caption{A scenario, where the $flip$ of \autoref{algo:constant-term} has to be used. This constellation can not be universally neglected.}
	\label{ex:constant-term-minus}
\end{figure}

Using \autoref{algo:coefficient} and \autoref{algo:constant-term} one can derive the \updatematrix $U \in \mathbb{Z}^{n\times n}$ and \updateconstants $u \in \mathbb{Z}^n$ for a rule $r_j$ of the form
\begin{figure}[H]
	\centering
	$r_j:= f_y(v_1,\dots v_n) \rightarrow f_y(v^\prime_1,\dots v^\prime_n) :|: cond$
\end{figure}  
so that the following holds:
\begin{figure}[H]
	\centering
	$U \times \begin{pmatrix} v_1 \\ \vdots \\ v_n \end{pmatrix} + u = \begin{pmatrix} v^\prime_1 \\ \vdots \\ v^\prime_n \end{pmatrix}$
\end{figure}

\subsection{The \guardmatrix and \guardconstants}
\label{sec:derivation-guard}
The derivation of the \guardmatrix and \guardconstants, whose definition is stated in \autoref{def:guard}, is very similar to \autoref{sec:derivation-update}, but instead of applying the algorithms to the update of the variables the algorithms have to be applied to the guards. The guards are given from the \seg in a standardized form.
\begin{definition}[standard guard form]
	A guard $g$ is in standard guard form iff
		$g := \varphi \circ c$, with
	$\phi$ in \stdLinInt, $a_i,v_i,c \in \mathbb{Z}$, $1 \le i \le n$ and $\circ \in \{ <, >\}$. \newline
	A condition to a rule $cond$ is in standard guard form iff 
	\begin{figure}[H]
		\centering
		$cond = \{ g | g\text{ guard}, g\text{ is in standard guard form}\}$
	\end{figure}
\end{definition} 
The conditions given be the \seg is one rule $r$ , which represents a set $G$  in standard guard form and 
\begin{figure}[H]
	\centering
	$r = \&\&(g_1,( \&\& (\dots,(\&\&(g_{n-1},g_n) )\dots)))$
\end{figure}
The easiest way to retrieve the guards $g_i$ is by using \autoref{algo:decat-guards}.

\begin{algorithm}
	\begin{algorithmic}[1]
		\Function{computeGuardSet}{Rule r} \Comment{r has to be a rule representing a $cond$-term}
			\State Stack $stack \gets r$
			\State Set $guards$
			\While{$!stack.isEmpty()$}
				\State $item \gets stack.pop$
				\If{item is of the form $\&\&(x_1,x_2)$}
					\State add $x_1$ and $x_2$ to $stack$
				\Else
					\State add $item$ to $guards$
				\EndIf				
			\EndWhile			
		\EndFunction		
	\end{algorithmic}
	\caption{Retrieving a set of guards $G$ from a rule $r$ of the form stated in \autoref{sec:derivation-guard}}
	\label{algo:decat-guards}
\end{algorithm}

So we get a set $G=\{ g \mid g \text{ is in standard guard form}\}$. Not only does the \seg normalize the guards to only use $>$, also it provides the guards $g:= \varphi > c \in G$ in the following form:\newline
\begin{figure}[H]
	\centering
	\begin{tikzpicture}
		\node[objDia] (top) {
			\textbf{f1}: RPNFunctionSymbol
			\nodepart{second}arithmeticSymbol: $>$
		};
		\node[rectangle, draw=black, rounded corners, text centered, anchor=north, below left = of top] (left) {
			$\varphi$
		};
		\node[objDia, below right = of top] (right) {
			\textbf{c1}: RPNConstant
			\nodepart{second}value: $c$
		};
	
		\draw[thickarrow] (top.south)  -- ++(0,-0.5) -| (left.north) node [pos = 0.4, above, font=\footnotesize]{left};
		\draw[thickarrow] (top.south)  -- ++(0,-0.5) -| (right.north) node [pos = 0.4, above, font=\footnotesize]{right};
	\end{tikzpicture}
\end{figure}

So to derive the entry's of the \guardmatrix we can simply use \autoref{algo:coefficient} of $\varphi$ and to derive entry's of \guardconstants we have to simply get the \code{right}-child of the root-node.\newline
After doing that, we got for $m=|G|$ the \guardmatrix $G \in \mathbb{Z}^{m\times n}$ and \guardconstants $g \in \mathbb{Z}^m$. \newline
In order to use it within the \iterationmatrix and \iterationconstants and further within the derivation of a \gna we transform it to the following:

\begin{figure}[H]
	\centering
	\begin{tabular}{rcll}
		$G \times \begin{pmatrix} v_1 \\ \vdots \\ v_n \end{pmatrix}$ & $>$ & $g$ &  \\
		$\begin{pmatrix} 
			a_{1,1} & \dots & a_{1,n} \\
			\vdots & \ddots & \vdots \\
			a_{n,1} & \dots & a_{n,n} \\
		\end{pmatrix} \times \begin{pmatrix} v_1 \\ \vdots \\ v_n \end{pmatrix}$&$ >$&$ \begin{pmatrix} c_1 \\ \vdots \\ c_n \end{pmatrix}$ & $| *-1$ \\
		$\begin{pmatrix} 
		-a_{1,1} & \dots & -a_{1,n} \\
		\vdots & \ddots & \vdots \\
		-a_{n,1} & \dots & -a_{n,n} \\
		\end{pmatrix} \times \begin{pmatrix} v_1 \\ \vdots \\ v_n \end{pmatrix}$&$ < $&$\begin{pmatrix} -c_1 \\ \vdots \\ -c_n \end{pmatrix}$ & $| reshape$ \\
		$\underbrace{\begin{pmatrix} 
		-a_{1,1} & \dots & -a_{1,n} \\
		\vdots & \ddots & \vdots \\
		-a_{n,1} & \dots & -a_{n,n} \\
		\end{pmatrix}}_{G} \times \begin{pmatrix} v_1 \\ \vdots \\ v_n \end{pmatrix} $&$\le$&$\underbrace{\begin{pmatrix} -c_1-1 \\ \vdots \\ -c_n-1 \end{pmatrix}}_{g}$ & \\
	\end{tabular}
	
\end{figure}
\subsection{The Iteration Matrix}
The \iterationmatrix and \iterationconstants are a composition of the previously derived \textit{Iteration-} and \guardmatrix respectively \textit{Iteration- } and \guardconstants. \newline
As stated in \autoref{def:iteration} the \iterationmatrix and \iterationconstants can be computed as
\begin{figure}[H]
	\centering
	$A = \begin{pmatrix} G & \textbf{0} \\ M & -I \\ -M & I \end{pmatrix}$ and $b = \begin{pmatrix} g \\ -u \\ u \end{pmatrix}$ \cite{leike2014geometric}
\end{figure}
Given $G, g, U$ and $u$ computing $A$ and $b$ is simply inserting and creating a matrix \textbf{0}$ \in \{0\}^m\times n$ and identity-matrix $I \in \{0,1\}^n\times n$, where $n$ is the number of distinct variables and $m$ the number of guards.

\section{Derivation of the \emph{SMT}-Problem}
\label{sec:derivation-smt}
The existence of a \gna is checked using an \tool{SMT} solver, presented in \autoref{sec:smt-problem}, which will either give us a model satisfying the constraints or proof the non existence by giving an unsatisfiable core. \newline
The constraints the \tool{SMT} solver has to fulfil are the four criteria mentioned within \autoref{def:gna}, which are non-linear. So the satisfiability of these is decidable. Since we derive the deterministic update as \updatematrix we can further compute it's eigenvalues and assign these to $\lambda_1, \dots \lambda_k$, receive linear constraints and thus can decide existence efficiently. \cite{leike2014geometric}.%TODO: show why eigenvalues simplify it
\newline
So the next step in order to proof non termination is to compute the eigenvalues of the \updatematrix. This is done by the \textit{Apache math3} library \footnote{the mentioned method can be found under \cite{ApacheMath3}}  because of performance reasons. Computation of such matrices can be very costly if programmed inefficiently. %TODO: maybe show the derivation principal?
After computing the eigenvalues, we have set values for $x$ and $\lambda_1, \dots \lambda_k$ as constant values.
\newline
Using the \smtfactory, which offers methods to create the within \autoref{sec:smt-problem} and \autoref{ex:assertion-structure} stated structure, we are able to create assertions and add them to the \solver, such that the following holds:
\begin{figure}[H]
	\centering
	If the \solver, with assertions $a^p_1, \dots a^p_n$ created from program $p$,has a model $m$ \newline
	then $m$ defines variables $y_1, \dots, y_k$ and $\mu_1, \mu_{k-1}$ within $\mathbb{Z}$ such that \newline
	$(x, y_1, \dots, y_k, \lambda_1, \dots, \lambda_k, \mu_1, \dots, \mu_{k-1})$ is a \gna.
\end{figure}

\subsection{The Domain Criteria}
\begin{itemize}
	\setlength{\itemindent}{1in}
	\item[(domain)] $x, y_1, \dots, y_k \in \mathbb{R}^n$, $\lambda_1, \dots \lambda_k, \mu_1, \dots \mu_{k-1} \ge 0$
	\item[] (see: \autoref{def:gna})
\end{itemize}
The \domc for $x$ and $y_1, \dots y_k$ are trivial, because at no point of computation we would consider a vector $v\in\mathbb{C}$.
The arity of $x$ is set within the derivation of the \stem (see: \autoref{sec:stem}) and set's the starting values for the $n$-variables. The arity of every $y_i$ is determined within the assertions of the \pointc and the \rayc.\newline
Therefore this criteria adds no further assertions towards the \solver.
\subsection{The Initiation Criteria}
\begin{itemize}
	\setlength{\itemindent}{1in}
	\item[(init)] x represents the \startterm (\stem)
	\item[] (see: \autoref{def:gna})
\end{itemize}

The \initc is quite trivial to mention within the \solver, since we defined the \stem $x$ within \autoref{sec:stem} to be exactly the  \textit{start term}.\newline
So this criteria also adds no further assertions towards the \solver.

\subsection{The Point Criteria}
\label{sec:point-criteria}
\begin{itemize}
	\setlength{\itemindent}{1in}
	\item[(point)] $A\begin{pmatrix} x \\ x + \sum_i y_i \end{pmatrix} \le b$
	\item[] (see: \autoref{def:gna})
\end{itemize}

The \pointc is the first criteria to add assertions towards the \solver.\newline
The point criteria has a special role within the derivation. Since within the \iterationmatrix $A$ the \updatematrix is contained twice with different sign the \iterationmatrix creates through the \pointc exactly opposite signed rules for the last $2n$ rows.
This means that, even though within the \pointc the relation is $\le$, the last $2n$ have to fulfil the equality of the rows.\newline

Let $s\in \mathbb{R}^n$ for $1 \le i \le n$ be $s_i=x_i+\sum_{j} (y_j)_i$, where $(y_j)_i$ denotes the $i$-th entry of $y_j$.
Then the \pointc can be rewritten to: 
\begin{figure}[H]
	\centering
	
	$A\begin{pmatrix} x \\ s \end{pmatrix} \le b$ \newline
	\hspace*{-6em}
	$\Leftrightarrow \begin{pmatrix}
				 & G 		& 			& 0 	 & \dots  & 0 \\
		a_{1,1}  & \dots 	& a_{1,n}	& -1 	 & \dots  & 0 \\
		\vdots   & \ddots 	& \vdots	& \vdots & \ddots & \vdots \\
		a_{n,1}  & \dots 	& a_{n,n}	& 0 	 & \dots  & -1 \\
		-a_{1,1} & \dots 	& -a_{1,n}	& 1 	 & \dots  & 0 \\
		\vdots   & \ddots 	& \vdots	& \vdots & \ddots & \vdots \\
		-a_{n,1} & \dots 	& -a_{n,n}	& 0 	 & \dots  & 1 \\
	\end{pmatrix} \begin{pmatrix} x1 \\ \vdots \\ x_n \\ s_1 \\ \vdots \\ s_n\end{pmatrix} \le \begin{pmatrix} g \\ -u_1 \\ \vdots\\ -u_n \\ u_1 \\ \vdots \\ u_n \end{pmatrix}$\\ %\newline
	$\Rightarrow Gx\le g$, which means that the guards have to hold, and \newline
	\vspace*{1em}
	\hspace*{1em}
	$\begin{pmatrix}
	a_{1,1}  & \dots 	& a_{1,n}	& -1 	 & \dots  & 0 \\
	\vdots   & \ddots 	& \vdots	& \vdots & \ddots & \vdots \\
	a_{n,1}  & \dots 	& a_{n,n}	& 0 	 & \dots  & -1 \\
	-a_{1,1} & \dots 	& -a_{1,n}	& 1 	 & \dots  & 0 \\
	\vdots   & \ddots 	& \vdots	& \vdots & \ddots & \vdots \\
	-a_{n,1} & \dots 	& -a_{n,n}	& 0 	 & \dots  & 1 \\
	\end{pmatrix} \begin{pmatrix} x1 \\ \vdots \\ x_n \\ s_1 \\ \vdots \\ s_n\end{pmatrix} \le \begin{pmatrix} -u_1 \\ \vdots\\ -u_n \\ u_1 \\ \vdots \\ u_n \end{pmatrix}$ \newline
	$= \begin{pmatrix}
	a_{1,1}*x_1  & \dots 	& a_{1,n}*x_n	& -1*s_1 & \dots  & 0*s_n \\
	\vdots   	 & \ddots 	& \vdots		& \vdots & \ddots & \vdots \\
	a_{n,1}*x_1  & \dots 	& a_{n,n}*x_n	& 0*s_1	 & \dots  & -1*s_n \\
	-a_{1,1}*x_1 & \dots 	& -a_{1,n}*x_n	& 1*s_1	 & \dots  & 0*s_n \\
	\vdots   	 & \ddots 	& \vdots		& \vdots & \ddots & \vdots \\
	-a_{n,1}*x_1 & \dots 	& -a_{n,n}*x_n	& 0*s_1	 & \dots  & 1*s_n \\
	\end{pmatrix} \le \begin{pmatrix} -u_1 \\ \vdots\\ -u_n \\ u_1 \\ \vdots \\ u_n \end{pmatrix}$\\
%	\\
	\vspace*{1em}
	By looking closely one can see that for every line $l_i$ $1 \le i \le n$ with 
	\begin{figure}[H]
		\centering
		$l^{\text{left}}_i \le l^{\text{right}}_i$
	\end{figure} 
	there is a rule $l_{i+n}$ with 
	\begin{figure}[H]
		\centering
		$-l^{\text{left}}_i \le -l^{\text{right}}_i$,
	\end{figure}
	which can be rewritten as $n$ rules of the form: 
	\begin{figure}[H]
		\centering
		$l^{\text{left}}_i = l^{\text{right}}_i$
	\end{figure}	
\end{figure}
So using the \smtfactory we create such variables $s_i$ and add the $n$ assertions determined above.\newline
Since variable vectors are represented as a \rpntree we can use a symbol method to calculate the multiplication, normalize the outcome and parse the \rpntree into an assertion all featured by the \smtfactory. \newline
The assertion ensuring that the new variables $s_i$ are the sum of the $i$-th value of the $y_j$ is added within \autoref{sec:additional-assertion}.


\subsection{The Ray Criteria}
\label{sec:ray-criteria}
\begin{itemize}
	\setlength{\itemindent}{1in}
	\item[(ray)] $A\begin{pmatrix} y_i \\ \lambda_i y_i + \mu_{i-1} y_{i-1} \end{pmatrix} \le 0$ for all $1 \le i \le k$
	\item[] (see: \autoref{def:gna})
\end{itemize}
The \rayc is the hardest criteria in terms of asserting, because of it's way of computation. \newline
The computation can be split into two parts on it's own.
\subsubsection{$i=0$:}
For $i=0$ the second addend $\mu_{i-1}y_{i-1}$ is equal to $0$, because of \autoref{def:gna}, that $y_0 = \mu_0 = 0$. So with $\lambda_1$ being the first eigenvalue of the \updatematrix we get that $A\begin{pmatrix} y_1 \\ \lambda_1 y_1 \end{pmatrix} \le 0$. \newline
Through $A$ and the \domc we know, that every $y_i\in \mathbb{R}^n$ so we add $n$ new variables $y_{1,i}$, such that $y_1 = \begin{pmatrix} y_{1,1} \\ \vdots \\ y_{1,n}\end{pmatrix}$, multiply the \updatematrix $A$ with the new vector regarding the substitution and and create an assertion per row using the \smtfactory, the \code{IntegerRelationType} \code{LE} and as the right hand side constant a 0.

\subsubsection{$i \ne 0$:}
Since we don't have any concrete values for any $y_i$ or $\mu_i$ so far the solving of the problem with the term $\mu_{i-1}y_{i-1}$ is not linear and therefore the computation has to be either performed in \textit{quantifier free non-linear integer arithmetic} or iterated over possible entry's for the $\mu$'s.\newline
In the implemented approach the \textit{quantifier free non-linear integer arithmetic}. Even if it's generally undecidable there are implementations over finite domains semi-deciding the problems. \cite{behrmann2014bit} \cite{giesl2016} \newline
Further comment about the usage of QF\_NIA can be found within \autoref{chapter:eval}.
\\
With $\lambda_i$ being the $i$-th eigenvalue of the \updatematrix we can compute the result of the multiplication as in the $i=0$ case, but have to normalize the outcome using the basic distributive property in order to handle it within an \rpntree and correctly generate an assertion from it.\newline
So for every step $i > 0$ we add $n$ new variables $y_{i,n}$ such that $y_i = \begin{pmatrix} y_{i,1} \\ \vdots \\ y_{i,n}\end{pmatrix}$ and a new variable $\mu_{i-1}$ such that 
\begin{figure}[H]
	\centering
	$\lambda_i y_i+\mu_{i-1}y_{i-1} \Leftrightarrow \lambda_i \begin{pmatrix} y_{i,1} \\ \vdots \\ y_{i,n}\end{pmatrix} + \mu_{i-1} \begin{pmatrix} y_{i-1,1} \\ \vdots \\ y_{i-1,n}\end{pmatrix} \Leftrightarrow \begin{pmatrix}
		\lambda_i y_{i,1}+\mu_{i-1}y_{i-1,1} \\ \vdots \\ \lambda_i y_{i,n}+\mu_{i-1}y_{i-1,n}
	\end{pmatrix}$ 
\end{figure}

As in the other case we can simply compute the multiplication with the \updatematrix $A$ , normalize the outcome and analogously create an assertion per row with the \smtfactory. \newline
Note that $y_{i-1,n}$ represent the values of the previous step and therefore not only already exist, but also create a lattice of restrictions for the $y_{i,j}$ since the values depend highly on the previous values. At this point the relation of rewriting the problem as a geometric series, like it is done in the underlying paper \cite{leike2014geometric}, is quite obvious.

\subsection{Additional assertion}
\label{sec:additional-assertion}
The final step of asserting need to be done, because of the restriction of the variables from the \rayc in \autoref{sec:ray-criteria} to the sum from the \pointc in \autoref{sec:point-criteria}.\newline
The assertion, that needs to be added has the following form:
\begin{figure}[H]
	\centering
	$s_i = y_{i,1}+ \dots + y_{i,n}$
\end{figure}
This ensures that the values of $y$ sum up to the values determined in the \pointc.
\\
\\
\\
After the adding the \addass from \autoref{sec:additional-assertion} the \solver contains all the restrictions to compute a \gna or an unsatisfiable core for the given program.\newline
If a \gna is found it is stored as an instance of the corresponding class and given to \aprove as a proof.

\section{Verification of the Geometric Non-Termination Argument}
\label{sec:verification-of-gna}
%TODO: soll das?
An instance of a \gna can be rechecked giving the \iterationmatrix and \iterationconstants if all of the four criteria of \autoref{def:gna} by simply computing and checking if the conditions hold.
	 
