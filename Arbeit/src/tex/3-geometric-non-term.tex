\chapter{Geometric Non-Termination}

Now that all preliminaries are stated we can start looking how the approach works within \aprove. To find a \gna and so prove nontermination we use \aprove to generate an \its of a given program. %TODO: write how? ref: prelim
Based on the calculated \its we derive the \stem, the \loopt and then generate an \tool{SMT}-Problem using \autoref{def:gna} and compute a \gna, which would be a prove of nontermination, or state that no \gna can exist, which does not infer termination nor nontermination.

\section{Derivation of the \emph{STEM}}
\label{sec:stem}
The derivation of the \stem is the first step to do in order to derive a \gna. As described in \autoref{sec:structure} the \stem defines the variables before iterating through the \loopt.  Owned to the fact, that \aprove has to find the a loop within the generated \seg %TODO: cite
one iteration through the \loopt will be calculated. Obviously this does not falsify the result. If it does not terminate i will still not terminate after one iteration and if it terminates after $n$ iterations and we compute one it will still terminate after $n-1$ iterations. \newline
Within the derivation of the \stem we distinguish between two cases discussed in the following sections.

\subsection{Constant \stem}
\label{sec:stem-const}
The constant stem is the easiest case to derive the \stem from. It has the form: 
%\vspace{-1em}
\begin{figure}[H]
	\centering
	$f_x \rightarrow f_y(c_1,\dots c_n) :|: TRUE$
\end{figure} 
%\vspace{-2em}
An example of a constant \stem is shown in \autoref{lst:stem-cons}. 
The values of $x$ can be directly read from the right hand side and need no further calculations.
\newsavebox{\stemexone}% Box to store smallmatrix content
\savebox{\stemexone}{$\begin{pmatrix}10\\2\end{pmatrix}$}
\begin{figure}[H]
	\begin{lstlisting}[escapechar=!]
	!$f_1 \rightarrow f_2(10,2) :|: TRUE $!
	\end{lstlisting}	
	\caption{An example of a constant \its rule to derive the \stem. The \stem in this case would be \usebox{\stemexone} }
	\label{lst:stem-cons}
\end{figure}

\subsection{Variable \stem}
\label{sec:stem-var}
The more complex case is given if the start function symbol has the following form:
\begin{figure}[H]
	\hspace{2cm}
	$f_x \rightarrow f_y(v_1, \dots v_n) :|: cond$
\end{figure}
%\vspace{-1em}
where $v_i$ $1 \le i \le n$ is either a constant term like in \autoref{sec:stem-const} \underline{or} a variable defined by the $cond$ term. An example for such a \stem is shown in \autoref{lst:stem-var}. In order to derive terms in $\mathbb{Z}$ an \tool{SMT}-Problem needs to be solved. We can compute the \guardmatrix, \guardconstants, \updatematrix and \updateconstants of the start function symbol and use the \smtfactory, which we will explain in %TODO: ref
, to create the assertions leading to either an assignment of $x$ to a value or to a unsatisfiable core. Such a core would state, that the \code{while}-Loop would not hold after any assignment and therefore prove termination.
%TODO: im code: wenn unsat dann termination

\newsavebox{\stemextwo}% Box to store smallmatrix content
\savebox{\stemextwo}{$\begin{pmatrix}1+3*3\\2\end{pmatrix}$}
\newsavebox{\stemextwosecond}% Box to store smallmatrix content
\savebox{\stemextwosecond}{$\begin{pmatrix}10\\2\end{pmatrix}$}
\begin{figure}[H]
	\begin{lstlisting}[escapechar=!]
	!$f_1 \rightarrow f_2(1 + 3 * v, 2) :|: v > 2\text{ \&\& }8 < 3 * v $!
	\end{lstlisting}	
	\caption{An example of a variable \its rule to derive the \stem. In order to derive it an $v$ fulfilling the conditions need to be found using an \tool{SMT}-Solver. Since $v=3$ is the first number in $\mathbb{Z}$ that satisfies the guards the \stem would be \usebox{\stemextwo}$=$\usebox{\stemextwosecond} }
	\label{lst:stem-var}
\end{figure}

\section{Derivation of the \emph{LOOP}}

The derivation of the \loopt is pretty straight forward applying \autoref{def:guard}, \autoref{def:update} to a looping rule and then computing \iterationmatrix and \iterationconstants using \autoref{def:iteration}. \newline
Let $f_x$ be the starting function symbol given by the \its and $r_i$ be a rule, with 
\begin{lstlisting}[escapechar=!]
	!$f_x \rightarrow f_y(v_1, \dots, v_n) :|: cond_1$!
\end{lstlisting} 
then we take the in lexicographical order first rule $r_l$ of the form 
\begin{lstlisting}[escapechar=!]
	!$f_y(v_1, \dots, v_n) \rightarrow f_y(v^\prime_1, \dots, v^\prime_n) :|: cond_2$!
\end{lstlisting}
and compute the \iterationmatrix and \iterationconstants according to $r_l$. 

\subsection{The \updatematrix and \updateconstants}
\label{sec:derivation-update}
The derivation of the \updatematrix and \updateconstants can be achieved by applying the \autoref{def:update} to the given rule $r_l$. For that we create $U$ as the coefficient matrix. The size of $U$ can be determined by adding a column per occurring variable and rows per linear equation of every $v^\prime_i$. To derive the entry's of the matrix we use the \rpntree of the given equation and simply perform recursive searching to derive the factor. 
%Also we can use
The procedure works like the following:
\begin{algorithm}[H]
	\caption{Derivation of a coefficient within an \rpntree}
	\label{algo:coefficient}
	\begin{algorithmic}[1]
		\Function{getCoefficient}{$query$}
			\If{node == query}\Comment{query is the tree}
				\State \Return $1$
			\ElsIf{node does \underline{not} contain query} \Comment{tree does not contain query}
				\State \Return $0$
			\EndIf
			\State
			\If{node represents PLUS} \Comment{Choose the subtree containing the query}
				\If{left side contains $query$}
					\State \Return getCoefficient$(query)$
				\Else
					\State \Return getCoefficient$(query)$
				\EndIf
			\EndIf
			\If{node represents TIMES} \Comment{Retrieve value}
				\If{node.right == query}
					\State \Return node.left.value
				\EndIf				
			\EndIf
		\EndFunction
	\end{algorithmic}
\end{algorithm}

Since we can rely on the usage of the \stdLinInt described in \autoref{sec:its} and therefore we can neglect cases for example that the \textit{left}-child of a \textit{RPNFunctionSymbol} with \textit{arithmeticSymbol} \code{TIMES} is the \textit{RPNVariable} and the \textit{right}-child is the \textit{RPNConstant}.
%TODO: Proof?
An example derivation of a factor using \autoref{algo:coefficient} is shown in \autoref{ex:factor-derivation}.

\begin{figure}[H]
	\begin{tikzpicture}
		\node (Plus) at (0,0) [objDia] {
				\textbf{f1}:RPNFunctionSymbol
				\nodepart{second}arithmeticSymbol: PLUS
			};
		\node (Times1) at (-4, -2 ) [objDia] {
				\textbf{f2}:RPNFunctionSymbol
				\nodepart{second}arithmeticSymbol: TIMES	
		};
		\node (Times2) at (4, -2) [objDia] {
			\textbf{f3}:RPNFunctionSymbol
			\nodepart{second}arithmeticSymbol: TIMES	
		};
		\node (cons1) at (-6, -4) [objDia] {
			\textbf{c1}:RPNConstant
			\nodepart{second}value: 2
		};
		\node (var1) at (-2, -4)[objDia] {
			\textbf{v1}:RPNVariable
			\nodepart{second}varName: x
		};
		\node (cons2) at (2, -4) [objDia] {
			\textbf{c2}:RPNConstant
			\nodepart{second}value: 4
		};
		\node (var2) at (6, -4) [objDia] {
			\textbf{v2}:RPNVariable
			\nodepart{second}varName: y
		};
		\draw[neglected] (Plus.south)  -- ++(0,-0.4) -| (Times1.north) node [pos = 0.4, above, font=\footnotesize]{left};
		\draw[considered] (Plus.south)  -- ++(0,-0.4) -| (Times2.north) node [pos = 0.4, above, font=\footnotesize]{right};
		\draw[thickarrow] (Times1.south)  -- ++(0,-0.5) -| (cons1.north) node [pos = 0.4, above, font=\footnotesize]{left};
		\draw[thickarrow] (Times1.south)  -- ++(0,-0.5) -| (var1.north) node [pos = 0.4, above, font=\footnotesize]{right};
		\draw[considered] (Times2.south)  -- ++(0,-0.5) -| (cons2.north) node [pos = 0.4, above, font=\footnotesize]{left};
		\draw[query] (Times2.south)  -- ++(0,-0.5) -| (var2.north) node [pos = 0.4, above, font=\footnotesize]{right};
	\end{tikzpicture}
	\caption{An example of deriving the coefficient of a given formula and a variable as query. This example uses the \rpntree of \autoref{ex:rpntree} and $y$ as the query. }
	\label{ex:factor-derivation}
\end{figure}

The \color{red}red\color{black}-arrow stands for the neglected left subtree of the root node, which can be neglected because the query is not contained. The \color{blue}blue\color{black}-arrows show the path to the subtree further investigated. The \color{green!50!black}green\color{black}-arrow determines, that the right child node is the query so the left child node has to be the coefficient. Since the underlying update is in \stdLinInt the left subtree has to be a \textit{RPNConstant}.

The \updateconstants can be derived by an simplification of \autoref{algo:coefficient}, since we only have to retrieve the constant term within the tree. The corresponding derivation is given by \autoref{algo:constant-term}.

\begin{algorithm}[H]
	\begin{algorithmic}[1]
		\Function{getConstantTerm}{}
			\If{this is a constant}
				\State \Return this.value
			\EndIf
			\State
			\State $flip \gets 1$
			\If{this represents MINUS}\Comment{flip result in case of prev. negation}
				\State $flip \gets -1$
			\EndIf
			\If{this represents sth. $\ne$ TIMES}
				\State $left \gets left.getConstantTerm()$ \Comment{recursive calls}
				\State $right \gets right.getConstantTerm()*flip$
				\State \Return $left+right$
			\EndIf
		\EndFunction
	\end{algorithmic}
	\caption{Derivation of a constant term within an \rpntree}
	\label{algo:constant-term}
\end{algorithm}
Since a constant $c < 0$ can stored in a constellation shown in \autoref{ex:constant-term-minus} we consider a variable $flip$ to store a sign change occurring for a subtraction. Knowing that the \stdLinInt is used all occurs of a multiplication can be neglected.\newline %TODO: small Proof?
Through the \stdLinInt one of the recursive calls has to be $0$ since only one constant term.

\begin{figure}[H]
	\centering
	\begin{tikzpicture}
		\node (minus) [objDia]{
			\textbf{f1}:RPNFunctionSymbol
			\nodepart{second}arithmeticSymbol: MINUS
		};
		\node[below left = of minus] (var) [objDia]{
			\textbf{v1}:RPNVariable
			\nodepart{second}varName: x
		};
		\node[below right = of minus] (cons) [objDia]{
			\textbf{c1}:RPNConstant
			\nodepart{second}value: 3
		};
		
		\draw[thickarrow] (minus.south)  -- ++(0,-0.5) -| (var.north) node [pos = 0.4, above, font=\footnotesize]{left};
		\draw[thickarrow] (minus.south)  -- ++(0,-0.5) -| (cons.north) node [pos = 0.4, above, font=\footnotesize]{right};
	\end{tikzpicture}
	\caption{A scenario, where the $flip$ of \autoref{algo:constant-term} has to be used. This constellation can not be universally neglected.} %TODO: eigentlich schon, da die Methode parseToRPN auf Integer testet und mit PLUS verknüpft
	\label{ex:constant-term-minus}
\end{figure}

Using \autoref{algo:coefficient} and \autoref{algo:constant-term} one can derive the \updatematrix $U \in \mathbb{Z}^{n\times n}$ and \updateconstants $u \in \mathbb{Z}^n$ for a rule $r_j$ of the form
\begin{figure}[H]
	\centering
	$r_j:= f_y(v_1,\dots v_n) \rightarrow f_y(v^\prime_1,\dots v^\prime_n) :|: cond$
\end{figure}  
so that the following holds:
\begin{figure}[H]
	\centering
	$U \times \begin{pmatrix} v_1 \\ \vdots \\ v_n \end{pmatrix} + u = \begin{pmatrix} v^\prime_1 \\ \vdots \\ v^\prime_n \end{pmatrix}$
\end{figure}

\subsection{The \guardmatrix and \guardconstants}
\label{sec:derivation-guard}
The derivation of the \guardmatrix and \guardconstants, whose definition is stated in \autoref{def:guard}, is very similar to \autoref{sec:derivation-update}, but instead of applying the algorithms to the update of the variables the algorithms have to be applied to the guards. The guards are given from the \seg in a standardized form.
\begin{definition}[standard guard form]
	A guard $g$ is in standard guard form iff
		$g := \varphi \circ c$, with
	$\phi$ in \stdLinInt, $a_i,v_i,c \in \mathbb{Z}$, $1 \le i \le n$ and $\circ \in \{ <, >\}$. \newline
	A condition to a rule $cond$ is in standard guard form iff 
	\begin{figure}[H]
		\centering
		$cond = \{ g | g\text{ guard}, g\text{ is in standard guard form}\}$
	\end{figure}
\end{definition} 
The conditions given be the \seg is one rule $r$ , which represents a set $G$  in standard guard form and 
\begin{figure}[H]
	\centering
	$r = \&\&(g_1,( \&\& (\dots,(\&\&(g_{n-1},g_n) )\dots)))$
\end{figure}
The easiest way to retrieve the guards $g_i$ is by using \autoref{algo:decat-guards}.

\begin{algorithm}
	\begin{algorithmic}[1]
		\Function{computeGuardSet}{Rule r} \Comment{r has to be a rule representing a $cond$-term}
			\State Stack $stack \gets r$
			\State Set $guards$
			\While{$!stack.isEmpty()$}
				\State $item \gets stack.pop$
				\If{item is of the form $\&\&(x_1,x_2)$}
					\State add $x_1$ and $x_2$ to $stack$
				\Else
					\State add $item$ to $guards$
				\EndIf				
			\EndWhile			
		\EndFunction		
	\end{algorithmic}
	\caption{Retrieving a set of guards $G$ from a rule $r$ of the form stated in \autoref{sec:derivation-guard}}
	\label{algo:decat-guards}
\end{algorithm}

So we get a set $G=\{ g \mid g \text{ is in standard guard form}\}$. Not only does the \seg normalize the guards to only use $>$, also it provides the guards $g:= \varphi > c \in G$ in the following form:\newline
\begin{figure}[H]
	\centering
	\begin{tikzpicture}
		\node[objDia] (top) {
			\textbf{f1}: RPNFunctionSymbol
			\nodepart{second}arithmeticSymbol: $>$
		};
		\node[rectangle, draw=black, rounded corners, text centered, anchor=north, below left = of top] (left) {
			$\varphi$
		};
		\node[objDia, below right = of top] (right) {
			\textbf{c1}: RPNConstant
			\nodepart{second}value: $c$
		};
	
		\draw[thickarrow] (top.south)  -- ++(0,-0.5) -| (left.north) node [pos = 0.4, above, font=\footnotesize]{left};
		\draw[thickarrow] (top.south)  -- ++(0,-0.5) -| (right.north) node [pos = 0.4, above, font=\footnotesize]{right};
	\end{tikzpicture}
\end{figure}

So to derive the entry's of the \guardmatrix we can simply use \autoref{algo:coefficient} of $\varphi$ and to derive entry's of \guardconstants we have to simply get the \code{right}-child of the root-node.\newline
After doing that, we got for $m=|G|$ the \guardmatrix $G \in \mathbb{Z}^{m\times n}$ and \guardconstants $g \in \mathbb{Z}^m$. \newline
In order to use it within the \iterationmatrix and \iterationconstants and further within the derivation of a \gna we transform it to the following:

\begin{figure}[H]
	\centering
	\begin{tabular}{rcll}
		$G \times \begin{pmatrix} v_1 \\ \vdots \\ v_n \end{pmatrix}$ & $>$ & $g$ &  \\
		$\begin{pmatrix} 
			a_{1,1} & \dots & a_{1,n} \\
			\vdots & \ddots & \vdots \\
			a_{n,1} & \dots & a_{n,n} \\
		\end{pmatrix} \times \begin{pmatrix} v_1 \\ \vdots \\ v_n \end{pmatrix}$&$ >$&$ \begin{pmatrix} c_1 \\ \vdots \\ c_n \end{pmatrix}$ & $| *-1$ \\
		$\begin{pmatrix} 
		-a_{1,1} & \dots & -a_{1,n} \\
		\vdots & \ddots & \vdots \\
		-a_{n,1} & \dots & -a_{n,n} \\
		\end{pmatrix} \times \begin{pmatrix} v_1 \\ \vdots \\ v_n \end{pmatrix}$&$ < $&$\begin{pmatrix} -c_1 \\ \vdots \\ -c_n \end{pmatrix}$ & $| reshape$ \\
		$\underbrace{\begin{pmatrix} 
		-a_{1,1} & \dots & -a_{1,n} \\
		\vdots & \ddots & \vdots \\
		-a_{n,1} & \dots & -a_{n,n} \\
		\end{pmatrix}}_{G} \times \begin{pmatrix} v_1 \\ \vdots \\ v_n \end{pmatrix} $&$\le$&$\underbrace{\begin{pmatrix} -c_1-1 \\ \vdots \\ -c_n-1 \end{pmatrix}}_{g}$ & \\
	\end{tabular}
	
\end{figure}
\subsection{The Iteration Matrix}
The \iterationmatrix and \iterationconstants are a composition of the previously derived \textit{Iteration-} and \guardmatrix respectively \textit{Iteration- } and \guardconstants. \newline
As stated in \autoref{def:iteration} the \iterationmatrix and \iterationconstants can be computed as
\begin{figure}[H]
	\centering
	$A = \begin{pmatrix} G & \textbf{0} \\ M & -I \\ -M & I \end{pmatrix}$ and $b = \begin{pmatrix} g \\ -u \\ u \end{pmatrix}$ \cite{leike2014geometric}
\end{figure}
Given $G, g, U$ and $u$ computing $A$ and $b$ is simply inserting and creating a matrix \textbf{0}$ \in \{0\}^m\times n$ and identity-matrix $I \in \{0,1\}^n\times n$, where $n$ is the number of distinct variables and $m$ the number of guards.

\section{Derivation of the \emph{SMT}-Problem}
The existence of a \gna is checked using an \tool{SMT} solver, which will either give us an argument satisfying the constraints or proof the non existence by given an unsatisfiable core. \newline
The constraints the \tool{SMT} solver has to fulfil are the four criteria mentioned within \autoref{def:gna}, which are non-linear. So the satisfiability of these is decidable. Since we derive the deterministic update as \updatematrix we can further compute it's eigenvalues and assign these to $\lambda_1, \dots \lambda_k$, receive linear constraints and thus can decide existence efficiently. \cite{leike2014geometric}.\newline
So the next step in order to proof non termination is to compute the eigenvalues of the \updatematrix.

\subsection{The Domain Criteria}

\subsection{The Initiation Criteria}

\subsection{The Point Criteria}

\subsection{The Ray Criteria}

\section{Verification of the Geometric Non-Termination Argument}
\label{sec:verification-of-gna}
	 
