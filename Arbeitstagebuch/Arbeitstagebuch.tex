%%Vorlage "Tagebuch", v1.03 %%

\documentclass[a4paper,11pt,DIV=calc]{scrartcl}
\usepackage[T1]{fontenc}
\usepackage[utf8]{inputenc}
\usepackage[ngerman]{babel}
\usepackage[osf]{libertine}
\usepackage{microtype,setspace}
\usepackage[colorlinks=false,pdfborder={0 0 0},bookmarksnumbered]{hyperref} 
\usepackage{amsmath}
\usepackage{amssymb}
\usepackage{color}
\usepackage{listings, lstautogobble}
\usepackage[dvipsnames]{xcolor}
\usepackage{graphicx}
\usepackage{tikz}
\usepackage{pstricks}
\usepackage{float}

\newcommand{\myparagraph}[1]{\paragraph{#1}\mbox{}\\}
\newcommand{\answer}{\color{green}Antwort: \color{black}}
\newcommand{\idea}{\color{red}Überlegung: \color{black}}
\newcommand{\code}{\texttt} 

% DEFINE CUSTOM C 
\lstdefinestyle{customc}{
	belowcaptionskip=1\baselineskip,
	breaklines=true,
	frame=L,
	xleftmargin=\parindent,
	language=C,
	showstringspaces=false,
	basicstyle=\footnotesize\ttfamily,
	keywordstyle=\bfseries\color{green!40!black},
	commentstyle=\itshape\color{purple!40!black},
	identifierstyle=\color{blue},
	stringstyle=\color{orange},
}
% END CUSTOM C
% DEFINE TERMINAL
\lstdefinestyle{BASH}
{
	backgroundcolor=\color{black},
	basicstyle=\scriptsize\color{white}\ttfamily,
	keywordstyle=\color{green!80!black},
	keywordstyle=[2]\color{red!80!white},
	morekeywords={timo@Ubuntu},
	keywords=[2]{ctest}
}

% END TERMINAL
% DEFINE JAVA
\definecolor{pblue}{rgb}{0.13,0.13,1}
\definecolor{pgreen}{rgb}{0,0.5,0}
\definecolor{pred}{rgb}{0.9,0,0}
\definecolor{pgrey}{rgb}{0.46,0.45,0.48}
\lstset{language=Java,
	showspaces=false,
	showtabs=false,
	breaklines=true,
	showstringspaces=false,
	breakatwhitespace=true,
	commentstyle=\color{pgreen},
	keywordstyle=\color{pblue},
	stringstyle=\color{pred},
	basicstyle=\tiny,
	moredelim=[il][\textcolor{pgrey}]{$$},
	moredelim=[is][\textcolor{pgrey}]{\%\%}{\%\%}
}
% END JAVA

\begin{document}


\hypersetup{
	pdftitle={Arbeitstagebuch},
	pdfauthor={Timo Bergerbusch},
	pdfsubject={Arbeitstagebuch}
	}
	
\title{Arbeitstagebuch}
\subtitle{Bachelorarbeit 2017}
\author{Timo Bergerbusch}
\date{}
\maketitle

\onehalfspace % 1,5-facher Zeilenabstand

%Vorwort zum Tagebuch
Die ist ein Arbeitstagebuch um den Überblick über bereits geleistete Arbeit zu behalten und Probleme und Änderungen zu protokollieren.
Dabei werden die verschiedenen Tage unterteilt in die Bereiche \emph{Allgemein} und \emph{Probleme/Offene Fragen}. \emph{Allgemein} beschreit was ich an dem Tag getan habe und womit ich mich beschäftigt habe und \emph{Probleme/Offene Fragen} beschreibt alle Probleme welche in Folge der Arbeit auftraten. Fragen, welche beantwortet werden sollen dann als Frage mit zugehöriger Erklärung im \emph{Allgemein}-Teil aufgegriffen werden.

%Alle Tage
%\section*{20.04.2017}

\paragraph{Allgemein}
\begin{itemize}
	\item Einlesen in die Paper \textit{main} und das Paper \textit{non-term}
	\item Installieren der Software auf dem Laptop und in der VM
\end{itemize}



\myparagraph{Probleme/Offene Fragen}
Zu \textit{main}:
\begin{enumerate}
	\item Seite 2, Preliminaries \newline
		Die Summe startet bei $0$, jedoch passt dies nicht. Wenn man $k=0$ setzt sollte $\begin{pmatrix}3\ 1\end{pmatrix}^T$ raus kommen, da keine Schleifeniteration durchgeführt wird also nur der \textbf{STEM}-Teil relevant ist. allerdings kommt dann $\begin{pmatrix}10\ 2\end{pmatrix}^T$ raus, was der Wert nach der 1. Iteration ist.
	
	\item Seite 3, Definition 2.2 \newline
		$Gx<g \land MX+m=x'$, was sind $G,g,M$ und $m$?
\end{enumerate}

zu \textit{non-term}:
\begin{enumerate}
	\item Seite 4, Definition 1: \newline
		$x,x' \in \mathbb{R}^n$, also $x=\begin{pmatrix}x_1,\ ... \ x_n\end{pmatrix}^T$ \newline
%		$A \in \mathbb{R}^{n \times m} $ \newline
%		Wenn $A\begin{pmatrix}x\ x'\end{pmatrix}$ korrekt sein soll, dann muss $\begin{pmatrix}x\\ x'\end{pmatrix} = \begin{pmatrix}
%			\begin{pmatrix}x_1,\ ... \ x_n\end{pmatrix} \\
%			\begin{pmatrix}x_1',\ ... \ x_n'\end{pmatrix}
%			\end{pmatrix}$ sein 
		Welche Dimension hat dann $\begin{pmatrix}x\\ x'\end{pmatrix}$?\newline
		$\begin{pmatrix}
					\begin{pmatrix}x_1,\ ... \ x_n\end{pmatrix} \\
					\begin{pmatrix}x_1',\ ... \ x_n'\end{pmatrix}
		\end{pmatrix} \in R^{2\times n}$?
\end{enumerate}
\section*{20.04.2017}

\paragraph{Allgemein}
\begin{itemize}
	\item Einlesen in die Paper \textit{main} und das Paper \textit{non-term}
	\item Installieren der Software auf dem Laptop und in der VM
\end{itemize}



\myparagraph{Probleme/Offene Fragen}
Zu \textit{main}:
\begin{enumerate}
	\item Seite 2, Preliminaries \newline
		Die Summe startet bei $0$, jedoch passt dies nicht. Wenn man $k=0$ setzt sollte $\begin{pmatrix}3\ 1\end{pmatrix}^T$ raus kommen, da keine Schleifeniteration durchgeführt wird also nur der \textbf{STEM}-Teil relevant ist. allerdings kommt dann $\begin{pmatrix}10\ 2\end{pmatrix}^T$ raus, was der Wert nach der 1. Iteration ist.
	
	\item Seite 3, Definition 2.2 \newline
		$Gx<g \land MX+m=x'$, was sind $G,g,M$ und $m$?
\end{enumerate}

zu \textit{non-term}:
\begin{enumerate}
	\item Seite 4, Definition 1: \newline
		$x,x' \in \mathbb{R}^n$, also $x=\begin{pmatrix}x_1,\ ... \ x_n\end{pmatrix}^T$ \newline
%		$A \in \mathbb{R}^{n \times m} $ \newline
%		Wenn $A\begin{pmatrix}x\ x'\end{pmatrix}$ korrekt sein soll, dann muss $\begin{pmatrix}x\\ x'\end{pmatrix} = \begin{pmatrix}
%			\begin{pmatrix}x_1,\ ... \ x_n\end{pmatrix} \\
%			\begin{pmatrix}x_1',\ ... \ x_n'\end{pmatrix}
%			\end{pmatrix}$ sein 
		Welche Dimension hat dann $\begin{pmatrix}x\\ x'\end{pmatrix}$?\newline
		$\begin{pmatrix}
					\begin{pmatrix}x_1,\ ... \ x_n\end{pmatrix} \\
					\begin{pmatrix}x_1',\ ... \ x_n'\end{pmatrix}
		\end{pmatrix} \in R^{2\times n}$?
\end{enumerate} 
\section*{26.04.2017}

\paragraph{Allgemein}
\begin{itemize}
	\item Script für das Tagebuch erstellt
	\item Versucht das Git-Repository in der VM zu installieren. Vergeblich
\end{itemize}

\myparagraph{Probleme/Offene Fragen}
\begin{enumerate}
	\item Vm - Git \newline
		Erstellen der Projekte wirft sofort Fehler. Wieso ist einfaches Clonen nicht ausreichend? \answer sollte eigentlich ausreichen
\end{enumerate} 
\section*{27.04.2017}

\paragraph{Allgemein}
\begin{itemize}
	\item Script abgeändert
	\item Repositorys in VM geklont und danach die Projekte angelegt. Jedoch nun 4000+ Errors. Code bis auf weiteres verschoben.
	\item weiteres einlesen in \textit{main} und \textit{non-term}
\end{itemize}

\myparagraph{Probleme/Offene Fragen}
Zum Ecplise-projekt:
\begin{enumerate}
	\item \emph{ant grammars} hat kein passendes \emph{build.xml}. Selbst mit \emph{build-aprove.xml} ist \emph{grammars} nicht definiert
\end{enumerate}

Zu \textit{main}:
\begin{enumerate}
	\item Seite3, Definition 2.2: Was sind G und M? M ist die \textit{\glqq actual update matrix \grqq} aber was soll das sein? $direction \times speed^i$ von der Introduction?
	\item Seite 4, Definition 2.6: defekt einer Matrix nur noch schleierhaft.\newline
	$def(A) = dim(ker(A))$ und	$ ker(A) = \{v \in \mathbb{R}| Av=0\}$
\end{enumerate} 
\section*{16.05.2017}

\paragraph{Allgemein}
\begin{itemize}
	\item David eine E-Mail geschrieben \answer Jera kommt Mittwoch (17.05.2017) wieder. David könnte mit bei der Installation am Freitag (19.05.2017) helfen
	\item erste Gedanken über den Ablauf:
	\begin{enumerate}
		\item Syntaxcheck: 
			\begin{enumerate}
				\item Teste auf erlaubte Elemente
					\begin{itemize}
						\item keine \textit{for}-Schleifen
						\item keine \textit{GOTO}'s oder ähnliches
					\end{itemize}
				\item Teste auf Unterteilung in \emph{STEM}
					\begin{enumerate}
						\item Anfangswerte für Variablen
						\item nicht aufgeführte Variablen werden mitgeschrieben
					\end{enumerate}
				\item Teste auf Unterteilung von \emph{LOOP}
					\begin{itemize}
						\item \textit{Guard} identifizieren 
						\item \underline{ausschließlich} lineare Updates
					\end{itemize}
			\end{enumerate}
		\item Simple Fälle abfangen
			\begin{itemize}
				\item eine Variable wird immer auf \code{nondet()} gesetzt
			\end{itemize}
	\end{enumerate}
\end{itemize}

\myparagraph{Probleme/Offene Fragen}
Zu non-term:
\begin{enumerate}
	\item non-term, Seite 2: Die Ausführung von \textit{Figure 1a}: Wieso $ \left( 2,0 \right)^T$ und $ \left( 2,1 \right)^T$ ? Angenommen die Reihenfolge ist $ \left( a,b \right)^T$ dann müsste es doch mit so etwas wie $ \left( undef/0,1 \right)^T$ starten. $b$ wird immer wieder auf \code{nondet} gesetzt was jede Ausführung sein kann. Sind also 2 und (immer) 1 zufällig gewählt? \idea Die ersten Einträge sind vor dem \emph{STEM} und somit beide \code{nondet}. Dann kommt der \emph{STEM} und dann die \emph{LOOP}. Zudem muss $a$ so gesetzt sein, dass die \textit{Guard} "passt"
	\item die Geometrische Reihe passt für \textit{Figure 1a} und \textit{Figure 1b} nicht
	\item zu \textit{Figure 1c}: $\mu$ Faktor von $b$?
\end{enumerate}

Allgemein:
\begin{enumerate}	
	\item Wo wird das Programm angesetzt? Als einzelner Thread nebenher oder an einer bestimmten Stelle?
	\item Gibt es dann Elemente auf die ich bereits zurückgreifen kann?
	\item Nur für Java?
\end{enumerate} 
\section*{17.05.2017}

\paragraph{Allgemein}
\begin{itemize}
	\item Weiteres Einlesen in \emph{non-term} und \emph{main}
	\item Vielleicht erst eine Methode um für geg. \emph{GNA} zu testen ob die 4 Bedingungen(\emph{non-term}, Seite 5) halten
	\item Verwendung von SMT-Solver:
		\begin{enumerate}
			\item Berechnen der Eigenwerte als $\lambda$'s der Updatematrix
			\item Berechnen der Eigenvektoren zu den Eigenwerten
		\end{enumerate}
\end{itemize}

\myparagraph{Probleme/Offene Fragen}
zu \emph{main}:
\begin{enumerate}
	\item Seite 5, Definition 3.1: Wenn wir $k$ $\lambda$'s haben aber nur $k-1$ $\mu$'s, wie sind dann Programme wie: 
		\begin{lstlisting}
			b=1;
			while a+b >= 4 do
				a = 3*a+b;
				b = 2*b +a;
			end while 
		\end{lstlisting}
		möglich? Das eine $\mu$ wird für die Beziehung von $a$ zu $b$ gebraucht aber dann ex. kein weiteres $\mu$ für die Beziehung von $b$ zu $a$.
	
\end{enumerate} 
\section*{28.05.2017}

\paragraph{Allgemein}
\begin{itemize}
	\item Ecplise versucht wieder ans laufen zu bekommen
	\item ITRS eingelesen
\end{itemize}

\myparagraph{Probleme/Offene Fragen}
Generell:
\begin{enumerate}
	\item Bis wann muss die Bachelorarbeit angemeldet sein?
	\item Kann das Projekt mit Eclipse nicht mehr öffnen
	\item Toolbar verschwindet ständig: \answer \emph{workbench.xmi} löschen und Neustarten
	\item Wie soll man aus einem \emph{ITRS} \emph{STEM} und \emph{LOOP} ablesen können?
\end{enumerate} 
\section*{08.06.2017}

\paragraph{Allgemein}
\begin{itemize}
	\item Compilieren von \emph{.c}-Dateien in \emph{.llvm}-Dateien
	\item Reproduzieren von verschiedenen Beweisen um mehr Verständnis zu erhalten wie ein ITRS abgelesen werden könnte
	\item Eingearbeitet in \emph{sat4j} und Erstellung einer Basis, welche eine Datein in CNF auf Erfüllbarkeit prüft: 
	\lstinputlisting[language=Java]{listings/sat4j-CNFSolver.java}
	Das Problem als Input
	\lstinputlisting{listings/CNF-InputFile.txt}
	
	
\end{itemize}

\myparagraph{Probleme/Offene Fragen}
\begin{enumerate}
	\item Wenn ich 
		\lstset{autogobble=true}
		\begin{lstlisting}[style=BASH]
			timo@Ubuntu:~/Downloads$ clang -S -emit-llvm ctest.c
		\end{lstlisting}
		für meine Testdatei \emph{ctest.c} eingebe erstellt mir \emph{Clang} eine \emph{.ll}-Datei, welche in Eclipse zu dem (bekannten) \emph{computePointerSizeFromDataLayout}-NullPointer führt. \newline
		\answer Benutze den Befehl: 
		\begin{lstlisting}[style=BASH]
			timo@Ubuntu:~/Downloads$ clang ctest.c -S -emit-llvm -o ctest.llvm
		\end{lstlisting}
	\item die \emph{clang} -Kompilierung verwirft die \emph{query}-Angabe und schreibt jedes mal \emph{source\_filename = "ctest.c"} hinzu, was Eclipse Probleme bereitet
	\item wenn ich einen \emph{.llvm}-Code erstellen lasse und die Anpassungen für \emph{query} und \emph{source\_filename} mache erhalte ich immer eine \emph{InconsistentStateException} im Graphen 
	\item Auch beim Web Interface bekomme ich die selben Probleme (nicht ausführen des Beweises: \emph{Analyze Termination of all function calls matching the pattern: main()}) . Z.B. für mein Test-File
	\newpage
		% DAS LISTING
		\lstset{escapechar=@, style=customc,xleftmargin=.2\textwidth, xrightmargin=.2\textwidth, autogobble=true}
		\lstinputlisting[language=C]{listings/simple-While.c}	
	\item Wo genau kann ich das ITRS herbekommen um darauf zu arbeiten? An welcher Stelle/Klasse o."a. \answer Bekomme das Problem als IRSwTProblem-Instanz. Siehe \hyperref[IRSwTProblemMethoden]{14.06.2017}
\end{enumerate}

 
\section*{09.06.2017}

\paragraph{Allgemein}
\begin{itemize}
	\item Änderung von den "Bounded" Variablen von \emph{True} in \emph{False}  in: 
		\begin{itemize}
			\item src/aprove/Testing/Manual/LLVMTermGraphTester.java
			\item src/aprove/PredefinedStrategies/Auto/current.strategy
			\item src/aprove/PredefinedStrategies/Auto/noT2.strategy
		\end{itemize}
	\item Git-Repo in Ubuntu geklont und verbunden
	\item Einlesen in die \emph{IRSwTPRoblem}-Art (src/aprove/Framework/IntTRS/IRSwTProblem.java)
	\item 
\end{itemize}

\myparagraph{Probleme/Offene Fragen}
\begin{enumerate}
	\item Zum \emph{IRSwTProblem}: im Beispiel des Javadoc's ist die 17 eine zufällige Zahl? \newline
	Zitat: "Variables not occurring in $\phi$ or in arithmetical operations may be instantiated by terms consisting of constructor symbols."
\end{enumerate} 


\end{document}