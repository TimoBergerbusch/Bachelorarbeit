%%Vorlage "Tagebuch", v1.03 %%

\documentclass[a4paper,11pt,DIV=calc]{scrartcl}
\usepackage[T1]{fontenc}
\usepackage[utf8]{inputenc}
\usepackage[ngerman]{babel}
\usepackage[osf]{libertine}
\usepackage{microtype,setspace}
\usepackage[colorlinks=false,pdfborder={0 0 0},bookmarksnumbered]{hyperref} 
\usepackage{amsmath}
\usepackage{amssymb}
\usepackage{color}
\usepackage{listings, lstautogobble}
\usepackage[dvipsnames]{xcolor}

\newcommand{\myparagraph}[1]{\paragraph{#1}\mbox{}\\}
\newcommand{\answer}{\color{green}Antwort: \color{black}}
\newcommand{\idea}{\color{red}Antwort: \color{black}}
\newcommand{\code}{\texttt} 

% DEFINE CUSTOM C 
\lstdefinestyle{customc}{
	belowcaptionskip=1\baselineskip,
	breaklines=true,
	frame=L,
	xleftmargin=\parindent,
	language=C,
	showstringspaces=false,
	basicstyle=\footnotesize\ttfamily,
	keywordstyle=\bfseries\color{green!40!black},
	commentstyle=\itshape\color{purple!40!black},
	identifierstyle=\color{blue},
	stringstyle=\color{orange},
}
% END CUSTOM C
% DEFINE TERMINAL
\lstdefinestyle{BASH}
{
	backgroundcolor=\color{black},
	basicstyle=\scriptsize\color{white}\ttfamily,
	keywordstyle=\color{green!80!black},
	keywordstyle=[2]\color{red!80!white},
	morekeywords={timo@Ubuntu},
	keywords=[2]{ctest}
}

% END TERMINAL
% DEFINE JAVA
\definecolor{pblue}{rgb}{0.13,0.13,1}
\definecolor{pgreen}{rgb}{0,0.5,0}
\definecolor{pred}{rgb}{0.9,0,0}
\definecolor{pgrey}{rgb}{0.46,0.45,0.48}
\lstset{language=Java,
	showspaces=false,
	showtabs=false,
	breaklines=true,
	showstringspaces=false,
	breakatwhitespace=true,
	commentstyle=\color{pgreen},
	keywordstyle=\color{pblue},
	stringstyle=\color{pred},
	basicstyle=\tiny,
	moredelim=[il][\textcolor{pgrey}]{$$},
	moredelim=[is][\textcolor{pgrey}]{\%\%}{\%\%}
}
% END JAVA

\begin{document}


\hypersetup{
	pdftitle={Arbeitstagebuch},
	pdfauthor={Timo Bergerbusch},
	pdfsubject={Arbeitstagebuch}
	}
	
\title{Arbeitstagebuch}
\subtitle{Bachelorarbeit 2017}
\author{Timo Bergerbusch}
\date{}
\maketitle

\onehalfspace % 1,5-facher Zeilenabstand

%Vorwort zum Tagebuch
Die ist ein Arbeitstagebuch um den Überblick über bereits geleistete Arbeit zu behalten und Probleme und Änderungen zu protokollieren.
Dabei werden die verschiedenen Tage unterteilt in die Bereiche \emph{Allgemein} und \emph{Probleme/Offene Fragen}. \emph{Allgemein} beschreit was ich an dem Tag getan habe und womit ich mich beschäftigt habe und \emph{Probleme/Offene Fragen} beschreibt alle Probleme welche in Folge der Arbeit auftraten. Fragen, welche beantwortet werden sollen dann als Frage mit zugehöriger Erklärung im \emph{Allgemein}-Teil aufgegriffen werden.

%Alle Tage
%\section*{20.04.2017}

\paragraph{Allgemein}
\begin{itemize}
	\item Einlesen in die Paper \textit{main} und das Paper \textit{non-term}
	\item Installieren der Software auf dem Laptop und in der VM
\end{itemize}



\myparagraph{Probleme/Offene Fragen}
Zu \textit{main}:
\begin{enumerate}
	\item Seite 2, Preliminaries \newline
		Die Summe startet bei $0$, jedoch passt dies nicht. Wenn man $k=0$ setzt sollte $\begin{pmatrix}3\ 1\end{pmatrix}^T$ raus kommen, da keine Schleifeniteration durchgeführt wird also nur der \textbf{STEM}-Teil relevant ist. allerdings kommt dann $\begin{pmatrix}10\ 2\end{pmatrix}^T$ raus, was der Wert nach der 1. Iteration ist.
	
	\item Seite 3, Definition 2.2 \newline
		$Gx<g \land MX+m=x'$, was sind $G,g,M$ und $m$?
\end{enumerate}

zu \textit{non-term}:
\begin{enumerate}
	\item Seite 4, Definition 1: \newline
		$x,x' \in \mathbb{R}^n$, also $x=\begin{pmatrix}x_1,\ ... \ x_n\end{pmatrix}^T$ \newline
%		$A \in \mathbb{R}^{n \times m} $ \newline
%		Wenn $A\begin{pmatrix}x\ x'\end{pmatrix}$ korrekt sein soll, dann muss $\begin{pmatrix}x\\ x'\end{pmatrix} = \begin{pmatrix}
%			\begin{pmatrix}x_1,\ ... \ x_n\end{pmatrix} \\
%			\begin{pmatrix}x_1',\ ... \ x_n'\end{pmatrix}
%			\end{pmatrix}$ sein 
		Welche Dimension hat dann $\begin{pmatrix}x\\ x'\end{pmatrix}$?\newline
		$\begin{pmatrix}
					\begin{pmatrix}x_1,\ ... \ x_n\end{pmatrix} \\
					\begin{pmatrix}x_1',\ ... \ x_n'\end{pmatrix}
		\end{pmatrix} \in R^{2\times n}$?
\end{enumerate}
\section*{20.04.2017}

\paragraph{Allgemein}
\begin{itemize}
	\item Einlesen in die Paper \textit{main} und das Paper \textit{non-term}
	\item Installieren der Software auf dem Laptop und in der VM
\end{itemize}



\myparagraph{Probleme/Offene Fragen}
Zu \textit{main}:
\begin{enumerate}
	\item Seite 2, Preliminaries \newline
		Die Summe startet bei $0$, jedoch passt dies nicht. Wenn man $k=0$ setzt sollte $\begin{pmatrix}3\ 1\end{pmatrix}^T$ raus kommen, da keine Schleifeniteration durchgeführt wird also nur der \textbf{STEM}-Teil relevant ist. allerdings kommt dann $\begin{pmatrix}10\ 2\end{pmatrix}^T$ raus, was der Wert nach der 1. Iteration ist.
	
	\item Seite 3, Definition 2.2 \newline
		$Gx<g \land MX+m=x'$, was sind $G,g,M$ und $m$?
\end{enumerate}

zu \textit{non-term}:
\begin{enumerate}
	\item Seite 4, Definition 1: \newline
		$x,x' \in \mathbb{R}^n$, also $x=\begin{pmatrix}x_1,\ ... \ x_n\end{pmatrix}^T$ \newline
%		$A \in \mathbb{R}^{n \times m} $ \newline
%		Wenn $A\begin{pmatrix}x\ x'\end{pmatrix}$ korrekt sein soll, dann muss $\begin{pmatrix}x\\ x'\end{pmatrix} = \begin{pmatrix}
%			\begin{pmatrix}x_1,\ ... \ x_n\end{pmatrix} \\
%			\begin{pmatrix}x_1',\ ... \ x_n'\end{pmatrix}
%			\end{pmatrix}$ sein 
		Welche Dimension hat dann $\begin{pmatrix}x\\ x'\end{pmatrix}$?\newline
		$\begin{pmatrix}
					\begin{pmatrix}x_1,\ ... \ x_n\end{pmatrix} \\
					\begin{pmatrix}x_1',\ ... \ x_n'\end{pmatrix}
		\end{pmatrix} \in R^{2\times n}$?
\end{enumerate} 
\section*{26.04.2017}

\paragraph{Allgemein}
\begin{itemize}
	\item Script für das Tagebuch erstellt
	\item Versucht das Git-Repository in der VM zu installieren. Vergeblich
\end{itemize}

\myparagraph{Probleme/Offene Fragen}
\begin{enumerate}
	\item Vm - Git \newline
		Erstellen der Projekte wirft sofort Fehler. Wieso ist einfaches Clonen nicht ausreichend? \answer sollte eigentlich ausreichen
\end{enumerate} 
\section*{27.04.2017}

\paragraph{Allgemein}
\begin{itemize}
	\item Script abgeändert
	\item Repositorys in VM geklont und danach die Projekte angelegt. Jedoch nun 4000+ Errors. Code bis auf weiteres verschoben.
	\item weiteres einlesen in \textit{main} und \textit{non-term}
\end{itemize}

\myparagraph{Probleme/Offene Fragen}
Zum Ecplise-projekt:
\begin{enumerate}
	\item \emph{ant grammars} hat kein passendes \emph{build.xml}. Selbst mit \emph{build-aprove.xml} ist \emph{grammars} nicht definiert
\end{enumerate}

Zu \textit{main}:
\begin{enumerate}
	\item Seite3, Definition 2.2: Was sind G und M? M ist die \textit{\glqq actual update matrix \grqq} aber was soll das sein? $direction \times speed^i$ von der Introduction?
	\item Seite 4, Definition 2.6: defekt einer Matrix nur noch schleierhaft.\newline
	$def(A) = dim(ker(A))$ und	$ ker(A) = \{v \in \mathbb{R}| Av=0\}$
\end{enumerate} 


\end{document}