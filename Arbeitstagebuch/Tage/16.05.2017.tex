\section*{16.05.2017}

\paragraph{Allgemein}
\begin{itemize}
	\item David eine E-Mail geschrieben \answer Jera kommt Mittwoch (17.05.2017) wieder. David könnte mit bei der Installation am Freitag (19.05.2017) helfen
	\item erste Gedanken über den Ablauf:
	\begin{enumerate}
		\item Syntaxcheck: 
			\begin{enumerate}
				\item Teste auf erlaubte Elemente
					\begin{itemize}
						\item keine \textit{for}-Schleifen
						\item keine \textit{GOTO}'s oder ähnliches
					\end{itemize}
				\item Teste auf Unterteilung in \emph{STEM}
					\begin{enumerate}
						\item Anfangswerte für Variablen
						\item nicht aufgeführte Variablen werden mitgeschrieben
					\end{enumerate}
				\item Teste auf Unterteilung von \emph{LOOP}
					\begin{itemize}
						\item \textit{Guard} identifizieren 
						\item \underline{ausschließlich} lineare Updates
					\end{itemize}
			\end{enumerate}
		\item Simple Fälle abfangen
			\begin{itemize}
				\item eine Variable wird immer auf \code{nondet()} gesetzt
			\end{itemize}
	\end{enumerate}
\end{itemize}

\myparagraph{Probleme/Offene Fragen}
Zu non-term:
\begin{enumerate}
	\item non-term, Seite 2: Die Ausführung von \textit{Figure 1a}: Wieso $ \left( 2,0 \right)^T$ und $ \left( 2,1 \right)^T$ ? Angenommen die Reihenfolge ist $ \left( a,b \right)^T$ dann müsste es doch mit so etwas wie $ \left( undef/0,1 \right)^T$ starten. $b$ wird immer wieder auf \code{nondet} gesetzt was jede Ausführung sein kann. Sind also 2 und (immer) 1 zufällig gewählt? \idea Die ersten Einträge sind vor dem \emph{STEM} und somit beide \code{nondet}. Dann kommt der \emph{STEM} und dann die \emph{LOOP}. Zudem muss $a$ so gesetzt sein, dass die \textit{Guard} "passt"
	\item die Geometrische Reihe passt für \textit{Figure 1a} und \textit{Figure 1b} nicht
	\item zu \textit{Figure 1c}: $\mu$ Faktor von $b$?
\end{enumerate}

Allgemein:
\begin{enumerate}	
	\item Wo wird das Programm angesetzt? Als einzelner Thread nebenher oder an einer bestimmten Stelle?
	\item Gibt es dann Elemente auf die ich bereits zurückgreifen kann?
	\item Nur für Java?
\end{enumerate}