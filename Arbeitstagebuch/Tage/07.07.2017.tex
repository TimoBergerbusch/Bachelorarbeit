\section*{07.07.2017}

\paragraph{Allgemein}
\begin{itemize}
	\item Fehlerbeseitigung: \newline
		Das Beispiel-Argument aus dem Paper funktioniert so leider nicht, da das \emph{ITRS} bereits wenige Schritte der \emph{LOOP} ausführt.\newline
		Demnach müssen die korrespondierenden \code{Y} angepasst werden. 
		$ A \times \begin{pmatrix} x_1 \\ x_1 + \sum_{i}y_i \end{pmatrix} = A \times \begin{pmatrix} 10 \\ 2 \\ a \\ b \end{pmatrix} =
		\begin{pmatrix}
			0 & -2 & 0 & 0 \\
			-3 & -1 & 0 & 0 \\
			-1 & -1 & 0 & 0 \\
			-3 & 0 & 0 & 0 \\
			3 & 1 & -1 & 0 \\
			0 & 2 & 0 & -1 \\
			-3 & -1 & 1 & 0 \\
			0 & -2 & 0 & 1
		\end{pmatrix} \times  \begin{pmatrix} 10 \\ 2 \\ a \\ b \end{pmatrix} =  \begin{pmatrix} -4 \\ -32 \\ -12 \\ -30 \\ 32-a \\ 4-b \\ -32+a \\ -4+b \end{pmatrix} \le  \begin{pmatrix} -3 \\ -31 \\ -11 \\ -29 \\ 0 \\ 0 \\ 0 \\ 0 \end{pmatrix}$
		\newline
		Offensichtliche Lösung für die Ungleichung ist $a=32, b=4$, somit muss 
			\begin{figure}[H] 
				\centering 
				$10 + \sum_{i}y_i^1=32 \leftrightarrow \sum_{i}y_i^1=22 \leftrightarrow y_1^1+y_2^1=22$ und \\
				$2 + \sum_{i}y_i^2 = 4 \leftrightarrow \sum_{i}y_i^2=2 \leftrightarrow y_1^2+y_2^1=2$			
			\end{figure}
		Problem: \newline
		 Es ex. keine $y_1=\begin{pmatrix} y_1^1 \\ y_1^2 \end{pmatrix}, y_2=\begin{pmatrix} y_2^1 \\ y_2^2 \end{pmatrix}$, sodass
			\begin{figure}[H]
				\centering
				$A*y_1<=0$, $ A*y_2<=0$ and $y_1+y_2=\begin{pmatrix} 22 \\ 2 \end{pmatrix}$
			\end{figure}
		Erklärung:\newline
		Die Eigenvektoren sind $\begin{pmatrix} 1 \\ 0 \end{pmatrix}$ mit Eigenwert $\lambda_1=3$ und $\begin{pmatrix} -1 \\ 1 \end{pmatrix}$ mit Eigenwert $\lambda_2=2 $. Wenn $y_1=\begin{pmatrix} y_1^1 \\ y_1^2 \end{pmatrix}$ und $y_2=\begin{pmatrix} y_2^1 \\ y_2^2 \end{pmatrix}$, dann müsste für die erste Ungleichung gelten, dass $A*\begin{pmatrix} y_1^1 \\ y_1^2 \\ 3*y_1^1 \\ 3*y_1^2 \end{pmatrix}\le 0 \leftrightarrow y_1^1 \ge 0, y_1^2 = 0$, da $\mu_0 = 0$ und $y_0 = $ nach Definition.\newline
		Jedoch für $y_2$ müsste gelten, dass $A*\begin{pmatrix} y_2^1 \\ y_2^2 \\ 2*y_2^1 + \mu_1*y_1^1\\ 2*y_2^2 + \mu_1*y_2^2 \end{pmatrix}\le 0$. \newline
		Dies hat nur die Lösungen: $y_1^1=0, y_2^1=0$ und $y_2^2=0$ für ein beliebiges $\mu$.\newline
		Dann wäre $y_1+y_2= \begin{pmatrix} 0 \\ 0 \end{pmatrix} + \begin{pmatrix} 0 \\ 0 \end{pmatrix}= \begin{pmatrix} 0 \\ 0 \end{pmatrix} \ne \begin{pmatrix} 22 \\ 2 \end{pmatrix}$
	\item
		Per Hand gerechnet geht es für das Beispiel im Skript auf ($a=4,c=3,b=0,d=1$).
		wolframalpha-Query's:
		\begin{itemize}
			\item Point: \{\{-1,-1,0,0\},\{3,1,-1,0\},\{0,2,0,-1\},\{-3,-1,1,0\},\{0,-2,0,1\}\}*\{3,1,10,2\}<=\{-4,0,0,0,0\}
			\item Ray 1: \{\{-1,-1,0,0\},\{3,1,-1,0\},\{0,2,0,-1\},\{-3,-1,1,0\},\{0,-2,0,1\}\}*\{a,b,3*a,3*b\}<=\{0,0,0,0,0\}
			\item Ray 2: \{\{-1,-1,0,0\},\{3,1,-1,0\},\{0,2,0,-1\},\{-3,-1,1,0\},\{0,-2,0,1\}\}*\{3,1,2*3+4,2*1\}<=\{0,0,0,0,0\}
		\end{itemize}
	\item Vielleicht ist es weil die Regeln: $x > y$ zu $-x < -y$ wurden und nicht zu $-x <= -y-1$
	\item WICHTIG: Im main-Paper ist ein gravierender Fehler! Die Eigenwerte wurden mit $\lambda_1 = 3 $ und  $\lambda_2 =1$ angegeben jedoch ist der zweite Eigenwert der Matrix $ \begin{pmatrix} 3 & 1 \\ 0 & 2 \end{pmatrix}$ ist $ \lambda_2 = 2$
	\item vielleicht: da alle Guards die Form $a_ix_1+b_ix_2+...>c_i$  haben müsste die Summe größer bleiben, also $\sum_{i} a_ix_1+b_ix_2+...$ $> \sum_{i} c_i$
\end{itemize}

\myparagraph{Probleme/Offene Fragen}
\begin{enumerate}
	\item Frage
\end{enumerate}