%\section{Beispiel}
%\frame{\tableofcontents[currentsection]}

\begin{frame}[fragile] %%Eine Folie
  \frametitle{Beispiel \tool{C}-Programm} %%Folientitel
  Ein Beispiel f\"ur den gesamten Vortrag:
  \begin{columns}
  	\begin{column}{6cm}
  		\begin{lstlisting}[language = java,escapechar = !]
int main(){
  		
	int a;!\tikz[remember picture] \node [] (a) {};!
	int b=1;!\tikz[remember picture] \node [] (b) {};!
  		
	while(a+b>=4){! \tikz[remember picture] \node [] (c) {}; !
    a=3*a+b;!\tikz[remember picture] \node [] (d) {}; !
    b=2*b-5;!\tikz[remember picture] \node [] (e) {}; !
	}
}		
  		\end{lstlisting}
%  		\begin{tikzpicture}[remember picture, overlay, 
%  		every edge/.append style = {dashedarrow},
%  		every node/.append style = {explNode},
%  		text width = 2cm ]
%  		\node[above right = .2cm and 2.5 cm of a] (A) {the \textit{STEM}};
%  		\draw[rounded corners=5pt] (A.west) edge (a.east);
%  		\draw (A.west) edge (b.east);
%  		
%  		\node[below = 1.5cm of A] (B) {the guard};
%  		\draw (B.west) edge (c.east);
%  		
%  		\node[below = .8cm of B] (C) {the linear update};
%  		\draw (C.west) edge (d.east);
%  		\draw (C.west) edge (e.east);
%  		\end{tikzpicture}
  	\end{column}
	\begin{column}{5cm}
		\begin{itemize}			
			\item einfaches \tool{C}-Programm
			\item terminiert es?
			\item[]<2-> $\Rightarrow$ \color{red}Nein!\color{black}
			\item[]<2-> wie kann man das beweisen?
		\end{itemize}
	\end{column}
  \end{columns}
\end{frame}