\section{N{\"o}tiges Vorwissen}

\subsection{Integer Transitionssystems (ITS) }
\frame{\tableofcontents[currentsection]}
\begin{frame}[fragile] %%Eine Folie
	\frametitle{Integer Transition Systems (ITS)}
	Hier betrachtete Programme:
	\begin{lstlisting}[escapechar=!]
	!$\overbrace{f_x}^{(1)} \qquad\quad\>\>\>\> \rightarrow \overbrace{f_y}^{(2)} (v_1, \dots v_n) :|: cond_1$!
	!$\>\>f_y(\underbrace{v_1, \dots v_n}_{(3)}) \> \rightarrow \>\> f_y \>\>\>(\underbrace{v^\prime_1,\dots v^\prime_n}_{(3)})  :|: \underbrace{cond_2}_{(4)}$!
	\end{lstlisting}
	
	\begin{enumerate}
		\item[(1)] Startsymbol (keine Variablen)
		\item[(2)] Funktionssymbol
		\item[(3)] Variablen $v^\prime_i$ als lineare Updates der Variablen $v_j$
		\item[(4)] eine Menge von (Un-)Gleichungen \"uber $v_j$
	\end{enumerate}
\end{frame}


\begin{frame}[fragile]
	\begin{itemize}
		\item Idee: Teilen des Programms in \blue{zwei} Teile:
			\begin{itemize}
				\item \stem: Variablen Deklaration und ggf. Initialisierung
					\begin{lstlisting}[language = java]
	int a;
	int b=1;
					\end{lstlisting}
				\item \loopt: \code{while}-Bedingung und lineare Updates
				\begin{lstlisting}[language = java]
	while(a+b>=4){
		a=3*a+b;
		b=2*b-5;
	}
				\end{lstlisting}
			\end{itemize}
		\item suche eines GNAs (J. Leike und M. Heizmann)
	\end{itemize}
\end{frame}

\begin{frame}[fragile]
	\vspace*{-.5em}
	\begin{variableblock}{Erinnerung: \its}{bg=orange!50!white,fg=black}{bg=orange, fg=white}
		\begin{lstlisting}[language = java,escapechar = !]
	int a;
	int b=1;!\tikz[remember picture] \node [] (b) {};!	
	while(a+b>=4){
		a=3*a+b; !\tikz[remember picture] \node [] (a) {};!
		b=2*b-5;!\tikz[remember picture] \node [] (bu) {};!
	}
		\end{lstlisting}
		\begin{tikzpicture}[remember picture, overlay,
		  	every node/.append style = {stdNode}]
		  	\node[right = 2cm of b] (txt) {$b^\prime=2*1-5=-3$};
		  	
		  	\node[right = 2cm of a] (txt2) {\text{hier: }$b=1$ !};
		  		
		  	\draw[dashedarrow] (txt.west) |- (b.east);
			\draw[dashedarrow] (txt.west) -- +(-0.5,0) |- (bu.east);
			
			\draw[dashedarrow, color = red] (txt2.west) |- (a.east);
		\end{tikzpicture}
	\end{variableblock}
	\vspace*{-.5em}
	\begin{exampleblock}{Beispiel}
		Das \its zum Beispielprogramm:
		\begin{lstlisting}[linewidth=10.5cm, escapechar = !]
!$f_1 \qquad\> \rightarrow f_2(1+3\cdot a,-3)   :|: a>2 \text{ \&\& } 8<3\cdot a$!
!$f_2(a,b) \rightarrow f_2(3\cdot a+b,z) \>\ :|: a + b > 3 \text{ \&\& } a > 6 \text{ \&\& } $!
				!$ 3 \cdot  a > 20 \text{ \&\& } 5 + z = 2 \cdot  b \text{ \&\& } z < -10 $!
		\end{lstlisting}
	\end{exampleblock}	
\end{frame}

\subsection{Definitionen}


\begin{frame}[fragile]
	\begin{variableblock}{Erinnerung: \its}{bg=orange!50!white,fg=black}{bg=orange, fg=white}
		\begin{lstlisting}[linewidth=10.5cm, escapechar = !]
!$f_1 \qquad\> \rightarrow f_2(1+3\cdot a,-3)   :|: a>2 \text{ \&\& } 8<3\cdot a$!
!$f_2(a,b) \rightarrow f_2(3\cdot a+b,z) \>\ :|: a + b > 3 \text{ \&\& } a > 6 \text{ \&\& } $!
				!$ 3 \cdot  a > 20 \text{ \&\& } 5 + z = 2 \cdot  b \text{ \&\& } z < -10 $!
		\end{lstlisting}
	\end{variableblock}
	\begin{exampleblock}{Guard-Matrix \& Konstanten}
		Vorher: Normalisieren 
		\begin{center}
			\only<1>{$a+b > 3 \Leftrightarrow -a-b < -3 \Leftrightarrow -a-b \le -4$}
			\only<2>{$a+b > 3 \Leftrightarrow -a-b < -3 \Leftrightarrow \green{-a}\blue{-b} \le \red{-4}$}
		\end{center}
		\uncover<2->{$\Rightarrow$ Guard-Matrix $G$ und Konstanten $g$:\newline
		\begin{center}
			\vspace{-2em}
			$G = \begin{pmatrix} \green{-1} & \blue{-1} \\ -1 & 0 \\ -3 & 0 \\ 0 & 2 \end{pmatrix}$ und $g= \begin{pmatrix} \red{-4} \\ -7 \\ -21 \\ -6 \end{pmatrix}$
		\end{center}}		
	\end{exampleblock}
\end{frame}

\begin{frame}[fragile]
	\begin{variableblock}{Erinnerung: \its}{bg=orange!50!white,fg=black}{bg=orange, fg=white}
		\begin{lstlisting}[linewidth=10.5cm, escapechar = !]
!$f_1 \qquad\> \rightarrow f_2(1+3\cdot a,-3)   :|: a>2 \text{ \&\& } 8<3\cdot a$!
!$f_2(a,b) \rightarrow f_2(\green{3\cdot a}+\blue{b},z) \>\ :|: a + b > 3 \text{ \&\& } a > 6 \text{ \&\& } $!
				!$ 3 \cdot  a > 20 \text{ \&\& } 5 + z = 2 \cdot  b \text{ \&\& } z < -10 $!
		\end{lstlisting}
	\end{variableblock}
		\begin{exampleblock}{Update-Matrix \& Konstanten}
			Vorher: Ersetzen
			\begin{center}
				$5+z=2\cdot b \Leftrightarrow z=2\cdot b - 5$
			\end{center}
			$\Rightarrow$ Update-Matrix $U$ und Konstanten $u$:\newline
			\begin{center}
				\vspace{-2em}
				$U = \begin{pmatrix} \green{3} & \blue{1} \\ 0 & 2 \end{pmatrix}$ und $u = \begin{pmatrix} \red{0} \\ -5 \end{pmatrix}$
			\end{center}		
		\end{exampleblock}
\end{frame}

\begin{frame}[fragile]
	\begin{definition}[Iteration-Matrix \& Konstanten]
		$\Rightarrow$ Iteration-Matrix $A$ und Konstanten $b$ wie folgt:
		\vspace*{-1em}
		\begin{figure}[H]
			\centering
			$A = \begin{pmatrix} \green{G} & \textbf{0} \\ \blue{U} & -I \\ -U & I \end{pmatrix}$ und $b = \begin{pmatrix} \green{g} \\ -u \\ \blue{u} \end{pmatrix}$
		\end{figure}
	\end{definition}
	\begin{exampleblock}{Beispiel: Iteration-Matrix \& Konstanten}
		\centering
		\(
			A = \left(\begin{array}{*{4}{c}}
				\tikzmark{left1}-1 		& -1 		&  0		& 0		 \\
				-1 		& 0 		&  0		& 0		 \\
				-3 		& 0 		&  0		& 0		 \\
				0 		& 2 \tikzmark{right1}		&  0		& 0		 \\
				\tikzmark{left2}3 		& 1 		&  -1		& 0		 \\
				0 		& 2 \tikzmark{right2} 		&  0		& -1	 \\
				-3 		& -1 		&  1		& 0		 \\
				0 		& -2 		&  0		& 1	 	 \\           
			\end{array}\right)
		\)
		\(
		b = \left(\begin{array}{*{1}{c}}
		\tikzmark{sleft1}-4 \\ -7 \\ -21 \\ -6\tikzmark{sright1} \\ 0 \\ 5 \\ \tikzmark{sleft2}0 \\ -5 \tikzmark{sright2}	          
		\end{array}\right)
		\)
		
		\DrawFirstBox[thick]
		\DrawFirstSmallBox[thick]
		\DrawSecondBox[thick]
		\DrawSecondSmallBox[thick]
	\end{exampleblock}
\end{frame}


\subsection{Geometrische Nicht-Terminierungs Argumente (GNA)}
\begin{frame}
	\begin{definition}[Geometrische Nicht-Terminierungs Argumente]
		\label{def:gna}
		Ein Tupel der Form:
		\vspace{-1em}
		\begin{figure}
			\centering
			$(x, y_1, \dots, y_k, \lambda_1, \dots, \lambda_k, \mu_1, \dots, \mu_{k-1})$
		\end{figure}  
		\vspace{-1em}
		ist ein GNA der Gr\"o\ss e $k$ mit $n$ Variablen g.d.w.:
		\begin{itemize}
			\setlength{\itemindent}{1in}
			\item[(domain)] $x, y_1, \dots, y_k \in \mathbb{R}^n$, $\lambda_1, \dots \lambda_k, \mu_1, \dots \mu_{k-1} \ge 0$
			\item[(init)] x repr\"asentiert den \stem
			\item[(point)] $A\begin{pmatrix} x \\ x + \sum_i y_i \end{pmatrix} \le b$
			\item[(ray)] $A\begin{pmatrix} y_i \\ \lambda_i y_i + \mu_{i-1} y_{i-1} \end{pmatrix} \le 0$ for all $1 \le i \le k$
		\end{itemize}
		Anmerkung:
		\begin{itemize}
			\item Definiere $y_0 = \mu_0 = 0$
			\item $\lambda_i$ ist der $i$-te Eigenwert von $U$.
		\end{itemize}
	\end{definition}
\end{frame}

\subsection{SMT-Solving}

\begin{frame}
	\frametitle{\color{white}SMT-Solving}
	
	\begin{itemize}
		\item Grundlegende Idee: 
			\begin{itemize}
				\item[] Menge von Regeln: (Un)-Gleichungen mit Variablen
				\item[] \qquad$\>\>$ $\xrightarrow{\text{\solver}}$ Modell \underline{oder} unerf\"ullbarer Kern
			\end{itemize}		
		\item \blue{Modell}: einen Wert f\"ur jede Variable
		\item \color{blue}unerf\"ullbarer Kern\color{black}: eine (minimale) unerf\"ullbare Menge von Bedingungen
	\end{itemize}
	\begin{exampleblock}{Beispiel}
		\begin{figure}[H]
			\centering
			\begin{tabular}{cccc}
				$x > 5 $ & $x \le y$ & $ x+ y \le 20$ &$y \neq 10$ \\
			\end{tabular}
		\end{figure}
		M\"ogliches Modell $m_1 = \{x=6, y=6\}$.
		\begin{figure}[H]
			\centering
			\begin{tabular}{cccc}
				$x > 5 $ & $x \le y$ & $ x+ y \le 10$ &$y \neq 10$ \\
			\end{tabular}
		\end{figure}
		unerf\"ullbarer Kern $\{x > 5$, $x \le y$,  $x+ y \le 10 \}$
	\end{exampleblock}
\end{frame}