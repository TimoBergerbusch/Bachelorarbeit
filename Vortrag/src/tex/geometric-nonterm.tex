\section{Geometrische Nichtterminierung}

\frame{\tableofcontents[currentsection]}

\subsection{Herleitung: STEM}


\begin{frame}[fragile]
	\frametitle{Herleitung: \stem}
	Unterscheiden von \blue{zwei} M\"oglichkeiten:\newline
	
	\begin{tabular}{rl}
		\blue{konstanter \stem}: & $f_1 \rightarrow f_2(10,-3) :|: TRUE$ \\
							  & $\Rightarrow$ Werte ablesen:  \\
							  & \quad$\>$ \stem = $ \begin{pmatrix} 10 \\ -3 \end{pmatrix}$ \\
								  
	\end{tabular}
	
	\uncover<2>{\begin{tabular}{rl}
								& \\
		\blue{variabler \stem}: & $f_1 \rightarrow f_2(1 + 3a, -3) :|: a > 2\text{ \&\& }8 < 3a $ \\
								& $\Rightarrow$ Bedingungen einem \solver geben \\
								& $\Rightarrow$ Modell $m_1=\{a=3\}$ \\
								& $\Rightarrow$ \stem = $ \begin{pmatrix} 1+3\cdot3 \\ -3 \end{pmatrix}=\begin{pmatrix} 10 \\ -3 \end{pmatrix}$
	\end{tabular}
	}
\end{frame}

\subsection{Herleitung: Guard-Matrix \& Konstanten}
\begin{frame}
	\frametitle{Herleitung: Guard-Matrix \& Konstanten}
	\begin{itemize}
		\item Bedingungen: $G=\{g \mid g \text{ ist eine (Un-)Gleichung}\}$
		\item<2-> \red{Problem}: $g$ muss nicht die Form $a_1 v_1+\dots +a_n v_n \le c$ haben
		\item<3-> \red{Schlimmer}: $g$ k\"onnte mittels \glqq$=$\grqq\xspace neue Variablen einf\"uhren
		\item<4-> \green{L\"osung}: Umformen von $g$ in die gew\"unschte Form: 
			\begin{tabular}{cll}
				1. & Gleichungen suchen:   & Ersetzen \glqq neuer\grqq\xspace Variablen \\
				2. & Normalisierung ($\le$): & Umschreiben $<,>,\ge$ zu $\le$ \\
				3. & Normalisierung ($c$):   & Umschreiben der Konstanten\\
			\end{tabular}
	\end{itemize}	
\end{frame}

\begin{frame}[fragile]
	\begin{enumerate}
		\item filtern, welche Variable nicht vorkommen soll
		\item Suchen einer Gleichung mit dieser Variable
		\item umformen und substituieren
	\end{enumerate}
	\begin{exampleblock}{Beispiel}
		Geg. Bedingungen:
		$\{a + b > 3\text{, } a > 6 \text{, } 3 \cdot  a > 20 \text{, } 5 + z = 2 \cdot  b \text{, } z < -10\}$
		\begin{enumerate}
			\item $z \in V_{sub}=V_{links}-V_{rechts}$
			\item $5+z=2\cdot b \Leftrightarrow z=2\cdot b - 5$
			\item neue Bedingungen
			\item[] $\{a + b > 3\text{, } a > 6 \text{, } 3 \cdot  a > 20 \text{, } 2\cdot b-5 < -10\}$
		\end{enumerate}
	\end{exampleblock}
	\begin{itemize}
		\item[] Zum Schluss: 
		\item[] Schreibe alle \"ubrigen Gleichungen als Ungleichungen
	\end{itemize}
\end{frame}

\begin{frame}
	\begin{enumerate}
		\item Normalisierung($\le$)
			\begin{itemize}
				\setlength{\itemindent}{.5in}
				\item[bei $\ge$/$>$:] Multipliziere mit $-1$ zu $\le/<$ 
				\item[bei $<$:] Subtrahiere 1 auf der rechten Seite und nutze $\le$
			\end{itemize}
		\item Normalisierung($c$)
			\begin{itemize}
				\setlength{\itemindent}{.5in}
				\item[] Subtrahiere linken konstanten Term  
			\end{itemize}
	\end{enumerate}
	\begin{exampleblock}{Beispiel}
		F\"ur $g \Leftrightarrow 3a+5>20$:
		\begin{tabular}{cll}
			Schritt 1: & Multipliziere mit $-1$ & $\Rightarrow -3a-5 < -20$ \\
			Schritt 2: & Subtrahiere 1 & $\Rightarrow -3a-5\le -21$ \\
			Schritt 3: & Subtrahiere konstanten Term & $\Rightarrow -3a \le -16$ \\
		\end{tabular}
	\end{exampleblock}
\end{frame}


\begin{frame} % conclus
	\begin{itemize}
		\item Jede Ungleichung hat die Form $a_1 v_1 + \dots + a_n v_n \le c$
		\item Herleiten der Guard-Konstanten ist nun trivial
		\item Herleiten der Guard-Matrix ist einfaches Ablesen
		\hspace*{1.5cm}(genauer unter der Update-Matrix)
		\item[$\Rightarrow$] Guard-Matrix \& Konstanten hergeleitet \checkmark
	\end{itemize}
\end{frame}

\subsection{Herleitung: Update-Matrix \& Konstanten}
\begin{frame}
	\frametitle{Herleitung: Update-Matrix \& Konstanten}
	\begin{itemize}
		\item beinhalten \underline{keine} (Un-)Gleichheiten
		\item haben die Form: $a_1 v_1 + \dots + a_n v_n + c$ (\underline{nicht} $v_i a_i$)
		\item \red{Problem}: k\"onnte \glqq neue\grqq\xspace Variablen aus den Guards beinhalten
		\item[]<2-> \green{L\"osung}: Substitutionen auf die Updates anwenden
		\item<2-> \red{Problem}: keine Struktureigenschaft wie bei den Guards
		\item[]<3-> \green{L\"osung}: rekursive Suche mit zwei Eigenschaften:
			\begin{enumerate}
				\item es existiert maximal eine Konstante
				\item die Konstante wird \underline{nicht} multipliziert
			\end{enumerate}
	\end{itemize}
\end{frame}

\begin{frame}
	\begin{columns}		
		\begin{column}{0.48\textwidth}
			\begin{exampleblock}{Beispiel: query=$a$, Result = 1}
				\begin{figure}[H]
					\centering
					\begin{tikzpicture}[scale=0.8, every node/.style={scale=0.8}]
					\node (var1) [stdNode, green!50!black, thick] {
						$a$
					};
					\end{tikzpicture}
				\end{figure}
			\end{exampleblock}
		\end{column}
		\begin{column}{0.48\textwidth}
			\begin{exampleblock}{Beispiel: query=$a$, Result=0}
				\begin{figure}[H]
					\centering
					\begin{tikzpicture}[scale=0.8, every node/.style={scale=0.8}]
					\node (Times1) [stdNode] {
						$\times$
					};
					\node (cons1) [stdNode, below left = 2em of Times1] {
						5
					};
					\node (var1) [stdNode, below right = 2em of Times1] {
						$z$
					};
					\draw[thickarrow] (Times1.south)  -- ++(0,-0.25) -| (cons1.north) node [pos = 0.4, above, font=\footnotesize]{left};
					\draw[neglected] (Times1.south)  -- ++(0,-0.25) -| (var1.north) node [pos = 0.4, above, font=\footnotesize]{right};
					\end{tikzpicture}
				\end{figure}
			\end{exampleblock}
		\end{column}
	\end{columns}
	\begin{exampleblock}{Beispiel: query=$a$, Result=3}
		\begin{figure}[H]
			\centering
			\begin{tikzpicture}[scale=0.8, every node/.style={scale=0.8}]
			\node (Plus) at (0,0) [stdNode] {
				$+$
			};
			\node (Times1) [stdNode, below left = 2em of Plus] {
				$\times$
			};
			\node (cons1) [stdNode, below left = 2em of Times1] {
				3
			};
			\node (var1) [stdNode, below right = 2em of Times1] {
				$a$
			};
			\node (var2) [stdNode, below right = 2em of Plus] {
				$b$
			};
			\draw[considered] (Plus.south)  -- ++(0,-0.4) -| (Times1.north) node [pos = 0.4, above, font=\footnotesize]{left};
			\draw[neglected] (Plus.south)  -- ++(0,-0.4) -| (var2.north) node [pos = 0.4, above, font=\footnotesize]{right};
			\draw[considered] (Times1.south)  -- ++(0,-0.5) -| (cons1.north) node [pos = 0.4, above, font=\footnotesize]{left};
			\draw[query] (Times1.south)  -- ++(0,-0.5) -| (var1.north) node [pos = 0.4, above, font=\footnotesize]{right};
			\end{tikzpicture}
		\end{figure}
	\end{exampleblock}
\end{frame}


\subsection{Herleitung: Iteration-Matrix \& Konstanten}
\begin{frame} % iterationmatrix
	\frametitle{\color{white}Iteration-Matrix \& Konstanten}
	\begin{itemize}
		\item Koeffizient f\"ur jede Variable und Konstante pro Update
		\item[$\Rightarrow$] Update-Matrix \& Konstanten \checkmark
		\uncover<2>{\item Iteration-Matrix \& Konstanten gegeben durch: 
			\begin{figure}[H]
				\centering
				$A = \begin{pmatrix} G & \textbf{0} \\ M & -I \\ -M & I \end{pmatrix}$ und $b = \begin{pmatrix} g \\ -u \\ u \end{pmatrix}$
			\end{figure}
		 k\"onnen berechnet werden
		 \item[$\Rightarrow$] Iteration-Matrix \& Konstanten \checkmark}
	\end{itemize}
\end{frame}

\subsection{Herleitung: SMT-Problem}
\begin{frame}
	\frametitle{SMT-Problem}
	\begin{itemize}
		\item Geg. Iteration-Matrix A und Iteration-Konstante b
		\item nutze SMT-Solver zum Beweis der (Nicht-)Existenz
		\item \red{Problem}: Teile des Ray-Kriteriums nicht linear
		\uncover<2>{\item[] zwei m\"ogliche L\"osungsans\"atze:
			\begin{enumerate}
				\item nutze \qfnia
				\item iteriere \"uber Unbekannte
			\end{enumerate}}
	\end{itemize}
\end{frame}

\begin{frame}
	\begin{variableblock}{Erinnerung: }{bg=orange!50!white,fg=black}{bg=orange, fg=white}
		\begin{itemize}
			\setlength{\itemindent}{1cm}
			\item[(domain)] $x, y_1, \dots, y_k \in \mathbb{R}^n$, $\lambda_1, \dots \lambda_k, \mu_1, \dots \mu_{k-1} \ge 0$
		\end{itemize}
	\end{variableblock}
	\begin{itemize}
		\item[] $\Rightarrow$ Bedingungen f\"ur  $\lambda_i$ und $\mu_i$ hinzuf\"ugen
	\end{itemize}
	\uncover<2>{\begin{variableblock}{Erinnerung:}{bg=orange!50!white,fg=black}{bg=orange, fg=white}
		\begin{itemize}
			\setlength{\itemindent}{1cm}
			\item[(init)] x repr\"asentiert den \stem
		\end{itemize}
	\end{variableblock}
	\begin{itemize}
		\item[] $\Rightarrow$ keine weiteren Bedingungen
	\end{itemize}}
	
\end{frame}

\begin{frame}
	\begin{variableblock}{Erinnerung: }{bg=orange!50!white,fg=black}{bg=orange, fg=white}
		\begin{itemize}
			\setlength{\itemindent}{1cm}
			\item[(point)] $A\begin{pmatrix} x \\ x + \sum_i y_i \end{pmatrix} \le b$
		\end{itemize}
	\end{variableblock}

	\begin{figure}[H]
		\centering		
		\only<1>{$\begin{pmatrix}G 	& \textbf{0} \\	U	& -I \\	-U & I \\\end{pmatrix} 
			\begin{pmatrix}	x \\ s \end{pmatrix} \le
			\begin{pmatrix}	g\\	-u\\ u\\ \end{pmatrix}$}
		\only<2>{$\begin{pmatrix}\blue{G} 	& \red{\textbf{0}} \\	U	& -I \\	-U & I \\\end{pmatrix} 
			\begin{pmatrix}	\blue{x} \\ \red{s} \end{pmatrix} \le
			\begin{pmatrix}	\blue{g}\\	-u\\ u\\ \end{pmatrix}$}
		\only<3>{$\begin{pmatrix}G 	& \textbf{0} \\	\blue{U}	& \blue{-I} \\	\red{-U} & \red{I} \\\end{pmatrix} 
			\begin{pmatrix}	x \\ s \end{pmatrix} \le
			\begin{pmatrix}	g\\	\blue{-u}\\ \red{u}\\ \end{pmatrix}$}
	\end{figure}
	
	\begin{itemize}
		\item[$\Rightarrow$]<2-> pr\"ufe Guard-Bedingungen f\"ur den \stem
		\item[]<3> eine Bedingung pro $s_i$ und
		\item[]<3> $n$ Bedingungen mit Gleichheit
		
	\end{itemize}
\end{frame}

\begin{frame}
	\begin{variableblock}{Erinnerung: Hergeleitete Matrix und Werte}{bg=orange!50!white,fg=black}{bg=orange, fg=white}
		\centering
		\begin{tabular}{rll}
			\multirow{2}{*}{$\begin{pmatrix}
				-1 		& -1 		&  0		& 0		 \\
				-1 		& 0 		&  0		& 0		 \\
				-3 		& 0 		&  0		& 0		 \\
				0 		& 2 		&  0		& 0		 \\
				3 		& 1 		&  -1		& 0		 \\
				0 		& 2 		&  0		& -1	 \\
				-3 		& -1 		&  1		& 0		 \\
				0 		& -2 		&  0		& 1	 	 \\
				\end{pmatrix}$}& &\multirow{2}{*}{$ \le \begin{pmatrix}
				-4 \\ -7 \\ -21 \\ -6 \\ 0 \\ 5 \\ 0 \\ -5
				\end{pmatrix} $}\\
			& & \\
			& \multirow{2}{*}{$\begin{pmatrix}
				x_{1,1} \\ x_{1,2} \\ s_{1} \\ s_{2}
				\end{pmatrix} $} & \\
			& & \\
			& & \\
			& & \\
			& & \\
			& & \\
		\end{tabular}
	\end{variableblock}
	\vspace*{-.5em}
	\begin{exampleblock}{Beispiel: Point Kriterium ($x=(10, -3)^T$ )}
		\begin{itemize}
			\item Guards: $-10-(-3)\le-4$
			\item Gleichheitsbedingung: $30-3-s_1=0$
			\item Summenbedingungen: $s_1= x_1+y_{1,1}+y_{2,1}$ und $s_2=x_2+y_{1,2}+y_{2,2}$
		\end{itemize}
	\end{exampleblock}	
\end{frame}

\begin{frame}
	\begin{variableblock}{Erinnerung: Ray Kriterium}{bg=orange!50!white,fg=black}{bg=orange, fg=white}
		\begin{itemize}
			\setlength{\itemindent}{1cm}
			\item[(ray)] $A\begin{pmatrix} y_i \\ \lambda_i y_i + \mu_{i-1} y_{i-1} \end{pmatrix} \le 0$ for all $1 \le i \le k$
		\end{itemize}
	\end{variableblock}
	F\"uge Bedingungen hinzu:
	\begin{itemize}
		\setlength{\itemindent}{1cm}
		\item[$i=1$:] $\Rightarrow \mu_{i-1}y_{i-1}=0 \Rightarrow $ $A\begin{pmatrix} y_1 \\ \lambda_1 y_1 \end{pmatrix} \le 0$
		\item[$i>1$:] mit $\lambda_i$ als den $i$-ten Eigenwert
		\item[]
		\item[] $\Rightarrow$ alle n\"otigen Bedingungen  hinzugef\"ugt\checkmark
		\item[] lasse den \solver ein GNA herleiten
	\end{itemize}	
\end{frame}

\begin{frame}
	\begin{variableblock}{Erinnerung: Hergeleitete Matrix und Werte}{bg=orange!50!white,fg=black}{bg=orange, fg=white}
	\begin{tabular}{rll}
		\multirow{2}{*}{$\begin{pmatrix}
			-1 		& -1 		&  0		& 0		 \\
			-1 		& 0 		&  0		& 0		 \\
			-3 		& 0 		&  0		& 0		 \\
			0 		& 2 		&  0		& 0		 \\
			3 		& 1 		&  -1		& 0		 \\
			0 		& 2 		&  0		& -1	 \\
			-3 		& -1 		&  1		& 0		 \\
			0 		& -2 		&  0		& 1	 	 \\
			\end{pmatrix}$}& &\multirow{2}{*}{$ \le \begin{pmatrix}
			0 \\ 0 \\ 0 \\ 0 \\ 0 \\ 0 \\ 0 \\ 0 \\ 
			\end{pmatrix} $}\\
		& & \\
		& \multirow{2}{*}{$\begin{pmatrix}
			y_{i,1} \\ y_{i,2} \\ \lambda_iy_{i,1}+\mu_{i-1}y_{i-1,1} \\ \lambda_iy_{i,2}+\mu_{i-1}y_{i-1,2}
			\end{pmatrix} $} & \\
		& & \\
		& & \\
		& & \\
		& & \\
		& & \\
	\end{tabular}
	\end{variableblock}
	\vspace*{-.5em}
	\begin{exampleblock}{Beispiel: Ray Kriterium ($\lambda_1 = 3,\lambda_2 = 2$) }
		\begin{itemize}
			\setlength{\itemindent}{0.25cm}
			\item[i=1:] Beispielbedingung: $-y_{1,1}-y_{1,2} \le -4$, $3\cdot y_{1,1}+y_{1,2}-3\cdot y_{1,1}=0$
			\item[i$=$2:] Beispielbedingung: $-y_{2,1}-y_{2,2} \le -4$, 
			\item[] $3\cdot y_{2,1}+y_{2,2}-1\cdot (2\cdot y_{2,1}+\mu\cdot y_{1,1})=0$
		\end{itemize}
	\end{exampleblock}
\end{frame}




\section{Beispiel  eines GNAs}
\frame{\tableofcontents[currentsection]}
\begin{frame}
	\frametitle{\color{white}Beispiel  eines GNAs}
	\begin{exampleblock}{Beispiel eines GNAs \rom{1}}
		Vom \solver: $y_1=\begin{pmatrix} 9 \\ 0 \end{pmatrix}$, $y_2=\begin{pmatrix} 8 \\ -8 \end{pmatrix}$, $\mu_1=0$\newline
		\vspace*{-.8em}
		\begin{itemize}
			\setlength{\itemindent}{1.5cm}
			\item[(domain)] offensichtlich wahr \checkmark
			\item[(init)] Berechnung des \stem (\checkmark)
			\item[(point)]  $A\begin{pmatrix} 10 \\ -3 \\ 10+9+8 \\ -3 + 0 + (-8) \end{pmatrix} \le b \Leftrightarrow A\begin{pmatrix} 10 \\ -3 \\ 27 \\ -11 \end{pmatrix} \le b$ \checkmark
		\end{itemize}
	\end{exampleblock}
\end{frame}

\begin{frame}
	\begin{exampleblock}{Beispiel eines GNAs \rom{2}}
		Vom \solver: $y_1=\begin{pmatrix} 9 \\ 0 \end{pmatrix}$, $y_2=\begin{pmatrix} 8 \\ -8 \end{pmatrix}$, $\mu_1=0$\newline
		\begin{itemize}
			\setlength{\itemindent}{0.5cm}
			\item[(ray)]
			\item[$i=1$:]  $A\begin{pmatrix} 9 \\ 0 \\ 3\cdot 9 \\ 3\cdot 0 \end{pmatrix} \le 0 \Leftrightarrow A\begin{pmatrix} 9 \\ 0 \\ 27 \\ 0 \end{pmatrix} \le 0$ \checkmark
			\item[$i>1$:] $A\begin{pmatrix} 8 \\ -8 \\ 2\cdot 8+0\cdot 9 \\ 2\cdot (-8)+0\cdot 0 \end{pmatrix} \le 0 \Leftrightarrow A\begin{pmatrix} 8 \\ -8 \\ 16 \\ -16 \end{pmatrix} \le 0$ \checkmark
		\end{itemize}
		$\Rightarrow$ das GNA erf\"ullt alle Kriterien\newline
		$\Rightarrow$ Nichtterminierung ist bewiesen
	\end{exampleblock}
\end{frame}






