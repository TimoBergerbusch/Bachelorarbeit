\section{Geometrische Nicht-Terminierung}

\frame{\tableofcontents[currentsection]}

%\begin{frame} %%Eine Folie
%  \frametitle{Geometrische Nicht-Terminierung} 
%  Notwendige Schritte zur Herleitung eines GNA:
%  \begin{enumerate}
%  	\item den \stem herleiten
%  	\item die Guard Matrix \& Konstanten herleiten
%  	\item die Update Matrix \& Konstanten herleiten
%  	\item die Iteration Matrix \& Konstanten herleiten
%  	\item alle Kriterien eines GNA einem \solver zuweisen
%  	\item GNA ablesen (wenn eins ex.)
%  \end{enumerate}
%\end{frame}

\subsection{Herleitung: STEM}


\begin{frame}[fragile]
	\frametitle{Herleitung: \stem}
	unterscheiden von \blue{zwei} M\"oglichkeiten:
	\begin{tabular}{rl}
		\blue{konstanter \stem}: & $f_x \rightarrow f_y(c_1,\dots,c_n) :|: TRUE$ \\
							  & $\Rightarrow$ Werte ablesen \\
	\end{tabular}
	\begin{exampleblock}{Beispiel}
		\centering
		$f_1 \rightarrow f_2(10,-3)$
		$\Rightarrow$ \stem = $ (10, -3)^T$
	\end{exampleblock}

	\uncover<2>{\begin{tabular}{rl}
		\blue{variabler \stem}: & $f_x \rightarrow f_y(c_1+\sum_{i=1}^{n}a_{1,i}v_i, \dots, c_n+\sum_{i=1}^{n}a_{n,i}v_i) :|: $ \\
							  & $\bigwedge_{\text{guard }g} \sum_{i=1}^{n}g_{n,i}v_i \le c_m$ \\
							  & $\Rightarrow$ Bedingungen einem \solver geben und  \\
							  & \quad$\>$ Modell ablesen \\
	\end{tabular}
	\begin{exampleblock}{Beispiel}
		\centering
		$f_1 \rightarrow f_2(1 + 3v_1, -3) :|: v_1 > 2\text{ \&\& }8 < 3v_1 $ \newline
		$\Rightarrow$ Modell $m_1=\{v_1=3\}$ $\Rightarrow$\stem = $ (10, -3)^T$
	\end{exampleblock}}
\end{frame}


\subsection{Herleitung: Guard Matrix \& Konstanten}
\begin{frame}
	\frametitle{Herleitung: Guard Matrix \& Konstanten}
	\begin{itemize}
		\item (Un-)Gleichungen vom \seg der Form
		\item[] \hspace{1cm} $r = \&\&(g_1,( \&\& (\dots,(\&\&(g_{n-1},g_n) )\dots)))$
%	\begin{algorithm}[H]
%		\begin{algorithmic}[1]
%			\Function{computeGuardSet}{Rule r}
%			\State Stack $stack \gets r$
%			\State Set $guards$
%			\While{$!stack.isEmpty()$}
%			\State $item \gets stack.pop$
%			\If{item is of the form $\&\&(x_1,x_2)$}
%			\State add $x_1$ and $x_2$ to $stack$
%			\Else
%			\State add $item$ to $guards$
%			\EndIf				
%			\EndWhile	
%			\State \Return $guards$
%			\EndFunction
%		\end{algorithmic}
%		\caption{derive set of guards}
%	\end{algorithm}
%\end{frame}	
%	
%\begin{frame}
	
		\item Aufspalten zu $G=\{g \mid g \text{ ist eine (Un-)Gleichung}\}$
		\item<2-> \red{Problem}: $g$ k\"onnte nicht die Form $\varphi \le c$ haben
		\item<3-> \red{Schlimmer}: $g$ k\"onnte mittels "$=$" neue Variablen einf\"uhren
		\item<4-> \green{L\"osung}: umformen von $g$ in die gew\"unschte Form: 
			\begin{tabular}{cll}
				1. & Gleichungen suchen   & ersetzen "neuer" Variablen \\
				2. & Normalisierung ($\le$) & umschreiben $<,>,\ge$ to $\le$ \\
				3. & Normalisierung ($c$)   & umschreiben der Konstanten\\
			\end{tabular}
	\end{itemize}	
\end{frame}

\begin{frame} % filter eq
	\vspace*{-.25cm}
	\begin{algorithm}[H]
		\begin{algorithmic}[1]
			\Function{filterEqualities}{$G$}
			\State $V_{left} = \{v \mid $ die linke Seite beinhaltet $ v\}$
			\State $V_{right} = \{v \mid $ die rechte Seite beinhaltet $ v\}$
			\State $V_{sub} = V_{right} - V_{left}$
			\State definiere Substitutionen $\theta=\{\}$
			\While{$V_{sub} \neq \emptyset}$
			\State w\"ahle $ s \in V_{sub}$
			\State w\"ahle $g_s \in \{g \in G\mid g \text{ beinhaltet } "=" \text{ und } s\}$
			\State entferne $g_s$ aus $G$
			\State umschreiben von $g_s$ zur Form $s = \psi$
			\State $\theta=\theta\{s/\psi\}$
			\ForAll{$g\in G$}
			\State $g = \theta g$
			\EndFor	
			\State entferne $s$ aus $V_{sub}$
			\EndWhile
			\State \Return $G$
			\EndFunction
		\end{algorithmic}
	\end{algorithm}
\end{frame}

\begin{frame} % bsp: filter eq.
	\begin{exampleblock}{Beispiel}
		F\"ur das Beispiel \its erhalten wir:\newline
		$\{v_1 + v_2 > 3\text{, } v_1 > 6 \text{, } 3 * v_1 > 20 \text{, } 5 + v_3 = 2 * v_2 \text{, } v_3 < -10\}$
		\begin{enumerate}
%			\setlength{\itemindent}{1cm}
			\item berechne $V_{left}=\{v_1, v_2\}$, $V_{right}=\{v_1,v_2,v_3\}$ so $V_{sub}=\{v_3\}$
			\item starte mit $\theta=\{\}$ 
			\item Da $V_{sub} \neq \emptyset$  w\"ahle $s=v_3$ und w\"ahle $g_s \Leftrightarrow 5+v_3=2*v_2$
			\item $g_s$ umschreiben zur Form $s=\psi$ f\"uhrt zu $v_3=2*v_2-5$
			\item $\theta = \theta\{v_3/2*v_2-5\} = \{v_3/2*v_2-5\}$
			\item $G=\{v_1 + v_2 > 3$, $ v_1 > 6 $, $ 3 * v_1 > 20 $, $ 2*v_2-5 < -10\}$
			\item Da $V_{sub}=\emptyset$ gebe $G$ zur\"uck
		\end{enumerate}
	\end{exampleblock}
\end{frame}

\begin{frame} % normal. <=
	\frametitle{Normalisierung ($\le$)}
	Umschreiben von $g_i$ der Form $g_i \Leftrightarrow \psi + c_{\psi} \circ c$, mit $\circ \in \{<,>,\le,\ge\}$ zur Form $ \eta*\psi + \eta*c_{\psi} \le \eta*c-\tau$ abh\"angig von $\circ$.\newline
	\begin{figure}[H]
		\vspace*{-1cm}
		\centering
		\begin{tabular}{|l|r|l|l|}
			\hline
			$\circ$ 	& $\eta$ 	& $\tau$ 	&  $ \eta*\psi + \eta*c_{\psi} \le \eta*c-\tau$ \\ 
			\hline \hline
			$<$ 		& $1$ 		&  $1$ 		& $\psi + c_{\psi} \le c - 1$ \\ \hline
			$>$ 		& $-1$		&  1 		& $-\psi - c_{\psi} \le -c -1 $ \\ \hline
			$\le$ 		& $1$ 		&  0 		& $\psi + c_{\psi} \le c$ \\ \hline
			$\ge$ 		& $-1$ 		&  0 		& $-\psi - c_{\psi} \le -c$ \\ \hline
		\end{tabular}
	\end{figure}
	\footnotesize
	$\eta$ beschreibt die Umwandlung von $\ge$ ($>$) zu $\le$ ($<$)\newline
	$\tau$ steht f\"ur m\"oglich Subtraktion um $\le$ statt $<$ zu erhalten
\end{frame}

%\begin{frame} % normal. c
%	\frametitle{Normalisierung ($c$)}
%	Subtrahiere $\eta*c_\psi$ auf beiden Seiten: \newline
%	endg\"ultige Form:  $\eta*\psi \le \underbrace{\eta*c -\tau -1*\eta*c_{\psi}}_{\text{constant term}}$
%	
%	\begin{variableblock}{Erinnerung: \its Struktur}{bg=orange!50!white,fg=black}{bg=orange, fg=white}
%		Herleiten des Konstanten Terms unter Ausnutzung einer Struktureigenschaft:
%		\begin{figure}[H]
%			\centering
%			\begin{tikzpicture}[scale=0.8, every node/.style={scale=0.8}]
%			\node[objDia] (top) {
%				\textbf{f1}: RPNFunctionSymbol
%				\nodepart{second}arithmeticSymbol: $\le$
%			};
%			\node[rectangle, draw=black, rounded corners, text centered, anchor=north, below left = of top] (left) {
%				$\eta*\psi$
%			};
%			\node[objDia, below right = of top] (right) {
%				\textbf{c1}: RPNConstant
%				\nodepart{second}value: $\eta*c -\tau -1*\eta*c_{\psi}$
%			};
%			
%			\draw[thickarrow] (top.south)  -- ++(0,-0.5) -| (left.north) node [pos = 0.4, above, font=\footnotesize]{left};
%			\draw[thickarrow] (top.south)  -- ++(0,-0.5) -| (right.north) node [pos = 0.4, above, font=\footnotesize]{right};
%			\end{tikzpicture}
%		\end{figure}
%	\end{variableblock}
%\end{frame}
%
%\begin{frame} % bsp: norm
%	\begin{exampleblock}{Beispiel}
%		Normalisierung der Ungleichung $g \Leftrightarrow 3*v_1>20 \Leftrightarrow \underbrace{3*v_1}_{\psi}+\underbrace{0}_{c_\psi} >\underbrace{20}_{c}$\newline
%		
%		Nachschlagen des Wertes von $\circ \Leftrightarrow >$: 
%		\hspace*{1.5cm}
%		\begin{tabular}{|l|r|l|l|}
%			\hline
%			$\circ$ 	& $\eta$ 	& $\tau$ 	&  $ \eta*\psi + \eta*c_{\psi} \le \eta*c-\tau$ \\ 
%			\hline \hline
%			\vdots 		& \vdots 	&  \vdots 		& \vdots \\ \hline
%			$>$ 		& $-1$		&  1 		& $-\psi - c_{\psi} \le -c -1 $ \\ \hline
%		\end{tabular} \newline
%	
%		Endergebnis mit $\eta=-1$, $\tau = 1$ in: \newline
%		\hspace*{1.5cm}$-(3*v_1)-(0)\le-20-1 \Leftrightarrow -3*v_1 \le -21$
%	\end{exampleblock}
%\end{frame}

\begin{frame} % normal. c
	\frametitle{Normalisierung ($c$)}
	Subtrahiere $\eta*c_\psi$ auf beiden Seiten: \newline
%	endg\"ultige Form:  $\eta*\psi \le \underbrace{\eta*c -\tau -1*\eta*c_{\psi}}_{\text{constant term}}$
	endg\"ultige Form:  $\eta*\psi \le \eta*c -\tau -1*\eta*c_{\psi}$
	
	\begin{exampleblock}{Beispiel}
		Normalisierung der Ungleichung $g \Leftrightarrow 3*v_1>20 \Leftrightarrow \underbrace{3*v_1}_{\psi}+\underbrace{0}_{c_\psi} > \underbrace{20}_{c}$\newline
		
		Nachschlagen des Wertes von $\circ \Leftrightarrow >$: 
		\hspace*{1.5cm}
		\begin{tabular}{|l|r|l|l|}
			\hline
			$\circ$ 	& $\eta$ 	& $\tau$ 	&  $ \eta*\psi + \eta*c_{\psi} \le \eta*c-\tau$ \\ 
			\hline \hline
			\vdots 		& \vdots 	&  \vdots 		& \vdots \\ \hline
			$>$ 		& $-1$		&  1 		& $-\psi - c_{\psi} \le -c -1 $ \\ \hline
		\end{tabular} \newline
	
		Endergebnis mit $\eta=-1$, $\tau = 1$ in: \newline
		\hspace*{1.5cm}$-(3*v_1)-(0)\le-20-1 \Leftrightarrow -3*v_1 \le -21$
	\end{exampleblock}
\end{frame}

\begin{frame} % conclus
	\begin{itemize}
		\item Jede Ungleichung hat die Form $\varphi \le c$
		\item Herleiten der Guard Konstanten ist nun trivial
		\item Herleiten der Guard Matrix ist einfaches Ablesen
		\hspace*{1.5cm}(genauer unter der Update Matrix)
		\item[$\Rightarrow$] Guard Matrix \& Konstanten hergeleitet \checkmark
	\end{itemize}
\end{frame}

\subsection{Herleitung: Update Matrix \& Konstanten}
\begin{frame}
	\frametitle{Herleitung: Update Matrix \& Konstanten}
	\begin{itemize}
		\item beinhalten \underline{keine} (Un-)Gleichheiten
		\item haben die Form: $c+\sum_{i=1}^{n}a_i v_i$ (\underline{nicht} $v_i a_i$)
		\item \red{Problem}: k\"onnte "neue" Variablen aus den Guard beinhalten
		\item[]<2-> \green{L\"osung}: Substitutionen auf die Updates anwenden
		\item<2-> \red{Problem}: keine Struktureigenschaft wie bei den Guards
		\item[]<3-> \green{L\"osung}: rekursive Suche mit zwei Eigenschaften:
			\begin{enumerate}
				\item es ex. maximal eine Konstante
				\item die Konstante wird \underline{nicht} multipliziert
			\end{enumerate}
	\end{itemize}
\end{frame}

\begin{frame} % algo: coefficient deriv.
	\begin{algorithm}[H]
		\caption{Derivation of a coefficient}
		\begin{algorithmic}[1]
			\Function{getCoefficient}{$query$}
			\If{this == query}
			\State \Return $1$
			\ElsIf{beinhaltet die query \underline{nicht}}
			\State \Return $0$
			\EndIf
			\State
			\If{this repr\"asentiert PLUS}
			\If{linke Seite beinhaltet $query$}
			\State \Return getCoefficient$(query)$
			\Else
			\State \Return getCoefficient$(query)$
			\EndIf
			\EndIf
			\If{this repr\"asentiert TIMES}
			\If{this.right == query}
			\State \Return this.left.value
			\EndIf				
			\EndIf
			\EndFunction
		\end{algorithmic}
	\end{algorithm}
\end{frame}

%\begin{frame}
%	code of constant derivation
%\end{frame}

\begin{frame}
	\begin{exampleblock}{Beispiel}
		Beispielherleitung des Koeffizienten von $v_1$:
		\begin{tikzpicture}[scale=0.8, every node/.style={scale=0.8}]
		\node (Plus) at (0,0) [objDia] {
			\textbf{f1}:RPNFunctionSymbol
			\nodepart{second}arithmeticSymbol: PLUS
		};
		\node (Times1) at (-4, -2 ) [objDia] {
			\textbf{f2}:RPNFunctionSymbol
			\nodepart{second}arithmeticSymbol: TIMES	
		};
		\node (cons1) at (-6, -4) [objDia] {
			\textbf{c1}:RPNConstant
			\nodepart{second}value: 3
		};
		\node (var1) at (-2, -4)[objDia] {
			\textbf{v1}:RPNVariable
			\nodepart{second}varName: $v_1$
		};
		\node (var2) at (4, -2) [objDia] {
			\textbf{v2}:RPNVariable
			\nodepart{second}varName: $v_2$
		};
		\draw[considered] (Plus.south)  -- ++(0,-0.4) -| (Times1.north) node [pos = 0.4, above, font=\footnotesize]{left};
		\draw[neglected] (Plus.south)  -- ++(0,-0.4) -| (var2.north) node [pos = 0.4, above, font=\footnotesize]{right};
		\draw[considered] (Times1.south)  -- ++(0,-0.5) -| (cons1.north) node [pos = 0.4, above, font=\footnotesize]{left};
		\draw[query] (Times1.south)  -- ++(0,-0.5) -| (var1.north) node [pos = 0.4, above, font=\footnotesize]{right};
		\end{tikzpicture}
	\end{exampleblock}
\end{frame}


\subsection{Herleitung: Iteration Matrix \& Konstanten}
\begin{frame} % iterationmatrix
	\frametitle{Update Matrix \& Konstanten, Iteration Matrix \& Konstanten}
	\begin{itemize}
		\item Koeffizient f\"ur jede Variable und Konstante pro Update
		\item[$\Rightarrow$] Update Matrix \& Konstanten \checkmark
		\uncover<2>{\item Iteration Matrix \& Konstanten gegeben durch: 
			\begin{figure}[H]
				\centering
				$A = \begin{pmatrix} G & \textbf{0} \\ M & -I \\ -M & I \end{pmatrix}$ und $b = \begin{pmatrix} g \\ -u \\ u \end{pmatrix}$
			\end{figure}
		 können berechnet werden
		 \item[$\Rightarrow$] Iteration Matrix \& Konstanten \checkmark}
	\end{itemize}
\end{frame}

\subsection{Herleitung: SMT-Problem}
\begin{frame}
	\frametitle{SMT-Problem}
	\begin{itemize}
		\item geg. $A$ und $b$ nutze \solver zum Beweisen der (nicht) ex. eines GNA
		\item in Anlehnung an den Beweis der GNA-Korrektheit:
		\item[]	\begin{block}{Satz}
					$\lambda_i$ ist der $i$-te Eigenwert von $U$.
				\end{block}
		\item \red{Problem}: $\mu_i*y_i$ ist nicht-linear
		\uncover<2>{\item[] zwei m\"ogliche L\"osungsans\"atze:
			\begin{enumerate}
				\item nutze \qfnia
				\item iterieren \"uber alle $\mu$'s
			\end{enumerate}}
	\end{itemize}
\end{frame}

\begin{frame}
	\begin{variableblock}{Erinnerung: Domain Kriterium}{bg=orange!50!white,fg=black}{bg=orange, fg=white}
		\begin{itemize}
			\setlength{\itemindent}{1cm}
			\item[(domain)] $x, y_1, \dots, y_k \in \mathbb{R}^n$, $\lambda_1, \dots \lambda_k, \mu_1, \dots \mu_{k-1} \ge 0$
		\end{itemize}
	\end{variableblock}
	\begin{itemize}
		\item[] $\Rightarrow$ Bedingungen f\"ur  $\lambda$'s und $\mu$'s hinzuf\"ugen
	\end{itemize}
	\uncover<2>{\begin{variableblock}{Erinnerung: Initiation Kriterium}{bg=orange!50!white,fg=black}{bg=orange, fg=white}
		\begin{itemize}
			\setlength{\itemindent}{1cm}
			\item[(init)] x repr\"asentiert den \stem
		\end{itemize}
	\end{variableblock}
	\begin{itemize}
		\item[] $\Rightarrow$ keine weiteren Bedingungen
	\end{itemize}}
	
\end{frame}

\begin{frame}
	\begin{variableblock}{Erinnerung: Point Kriterium}{bg=orange!50!white,fg=black}{bg=orange, fg=white}
		\begin{itemize}
			\setlength{\itemindent}{1cm}
%			\item[(domain)] $x, y_1, \dots, y_k \in \mathbb{R}^n$, $\lambda_1, \dots \lambda_k, \mu_1, \dots \mu_{k-1} \ge 0$
%			\item[(init)] x represents the \startterm (\stem)
			\item[(point)] $A\begin{pmatrix} x \\ x + \sum_i y_i \end{pmatrix} \le b$
			%			\item[(ray)] $A\begin{pmatrix} y_i \\ \lambda_i y_i + \mu_{i-1} y_{i-1} \end{pmatrix} \le 0$ for all $1 \le i \le k$
		\end{itemize}
	\end{variableblock}
	\begin{itemize}
		\item $y_i$ unbekannt $\Rightarrow$ erstelle $s_i=x_i+\sum_{j=1}^n y_{j,i}$
		\item $A\begin{pmatrix} x \\ s \end{pmatrix} \le b$ \newline
	\end{itemize}
	$\Leftrightarrow \begin{pmatrix}
	& G 		& 			& 0 	 & \dots  & 0 \\
	a_{1,1}  & \dots 	& a_{1,n}	& -1 	 & \dots  & 0 \\
	\vdots   & \ddots 	& \vdots	& \vdots & \ddots & \vdots \\
	a_{n,1}  & \dots 	& a_{n,n}	& 0 	 & \dots  & -1 \\
	-a_{1,1} & \dots 	& -a_{1,n}	& 1 	 & \dots  & 0 \\
	\vdots   & \ddots 	& \vdots	& \vdots & \ddots & \vdots \\
	-a_{n,1} & \dots 	& -a_{n,n}	& 0 	 & \dots  & 1 \\
	\end{pmatrix} \begin{pmatrix} x1 \\ \vdots \\ x_n \\ s_1 \\ \vdots \\ s_n\end{pmatrix} \le \begin{pmatrix} g \\ -u_1 \\ \vdots\\ -u_n \\ u_1 \\ \vdots \\ u_n \end{pmatrix}$\\ %\newline
\end{frame}

\begin{frame}[fragile]
	\begin{itemize}
%		\item $y_i$ unknown $\Rightarrow$ create $s_i=x_i+\sum_{j=1}^n y_{j,i}$
		\vspace*{-1em} \item[] \hspace*{-1cm} 
			\begin{tikzpicture}
				\matrix [matrix of math nodes,left delimiter=(,right delimiter=)] (m)
				{
					a_{1,1}*x_1  & \dots 	& a_{1,n}*x_n	& -1*s_1 & \dots  & 0*s_n \\
					\vdots   	 & \ddots 	& \vdots		& \vdots & \ddots & \vdots \\
					a_{n,1}*x_1  & \dots 	& a_{n,n}*x_n	& 0*s_1	 & \dots  & -1*s_n \\
					-a_{1,1}*x_1 & \dots 	& -a_{1,n}*x_n	& 1*s_1	 & \dots  & 0*s_n \\
					\vdots   	 & \ddots 	& \vdots		& \vdots & \ddots & \vdots \\
					-a_{n,1}*x_1 & \dots 	& -a_{n,n}*x_n	& 0*s_1	 & \dots  & 1*s_n \\           
				};  
				\node[right = 0.25 cm of m] (le) {$\le$}; 
				\matrix [right = 0.25cm of le, matrix of math nodes,left delimiter=(,right delimiter=)] (c)
				{
					-u_1 \\ 
					\vdots\\ 
					-u_n \\ 
					u_1 \\ 
					\vdots \\ 
					u_n \\           
				};
				\draw[color=red, thick] (m-1-1.north west) -- (m-1-6.north east) -- (m-1-6.south east) -- (m-1-1.south west) -- (m-1-1.north west);	
				\draw[color=red, thick] (m-4-1.north west) -- (m-4-6.north east) -- (m-4-6.south east) -- (m-4-1.south west) -- (m-4-1.north west);
				\draw[color=red, thick] (c-1-1.north west) -- (c-1-1.north east) -- (c-1-1.south east) -- (c-1-1.south west) -- (c-1-1.north west);
				\draw[color=red, thick] (c-4-1.north west) -- (c-4-1.north east) -- (c-4-1.south east) -- (c-4-1.south west) -- (c-4-1.north west);
			\end{tikzpicture}
		\item[$\Rightarrow$] f\"uge Guard Bedingungen f\"ur den \stem hinzu, $n$ Bedingungen mit Gleichheit
		\item[] und eine Bedingung pro $s_i$
	\end{itemize}
\end{frame}

\begin{frame}
	\begin{variableblock}{Erinnerung: Hergeleitete Matrix und Werte}{bg=orange!50!white,fg=black}{bg=orange, fg=white}
		\centering
		\begin{tabular}{rll}
			\multirow{2}{*}{$\begin{pmatrix}
				-1 		& -1 		&  0		& 0		 \\
				-1 		& 0 		&  0		& 0		 \\
				-3 		& 0 		&  0		& 0		 \\
				0 		& 2 		&  0		& 0		 \\
				3 		& 1 		&  -1		& 0		 \\
				0 		& 2 		&  0		& -1	 \\
				-3 		& -1 		&  1		& 0		 \\
				0 		& -2 		&  0		& 1	 	 \\
				\end{pmatrix}$}& &\multirow{2}{*}{$ \le \begin{pmatrix}
				-4 \\ -7 \\ -21 \\ -6 \\ 0 \\ 5 \\ 0 \\ -5
				\end{pmatrix} $}\\
			& & \\
			& \multirow{2}{*}{$\begin{pmatrix}
				x_{1,1} \\ x_{1,2} \\ s_{1} \\ s_{2}
				\end{pmatrix} $} & \\
			& & \\
			& & \\
			& & \\
			& & \\
			& & \\
		\end{tabular}
	\end{variableblock}
	\vspace*{-.5em}
	\begin{exampleblock}{Beispiel: Bedingungen \rom{1}: Point Krit.($x=(10, -3)^T$ )}
		\begin{itemize}
			\item Guards: $-10-(-3)\le-4$
			\item Gleichheitsbedingung: $30-3-s_1=0$
			\item Summenbedingungen: $s_1= x_1+y_{1,1}+y_{2,1}$ und $s_2=x_2+y_{1,2}+y_{2,2}$
		\end{itemize}
	\end{exampleblock}	
\end{frame}

\begin{frame}
	\begin{variableblock}{Erinnerung: Ray Kriterium}{bg=orange!50!white,fg=black}{bg=orange, fg=white}
		\begin{itemize}
			\setlength{\itemindent}{1cm}
			\item[(ray)] $A\begin{pmatrix} y_i \\ \lambda_i y_i + \mu_{i-1} y_{i-1} \end{pmatrix} \le 0$ for all $1 \le i \le k$
		\end{itemize}
	\end{variableblock}
	\begin{itemize}
		\setlength{\itemindent}{1cm}
		\item[$i=1$:] $\Rightarrow \mu_{i-1}y_{i-1}=0 \Rightarrow $ $A\begin{pmatrix} y_1 \\ \lambda_1 y_1 \end{pmatrix} \le 0$
		\item[]	f\"uge Bedingung hinzu mit $y_{1,j}$ $1\le j \le n$ als neue 
		\item[] Variablen
		\item[$i>1$:] mit $\lambda_i$ als den $i$-ten Eigenwert
		\item[] f\"uge Bedingung hinzu mit $y_{i,j}$ $1 \le j \le n$ 
		\item[] und $\mu_i$ als neue Variablen
		\item[]
		\item[] $\Rightarrow$ alle n\"otigen Bedingungen  hinzugef\"ugt\checkmark
		\item[] lasse den \solver ein GNA herleiten (wenn eins ex.)
	\end{itemize}	
\end{frame}

\begin{frame}
	\begin{variableblock}{Erinnerung: Hergeleitete Matrix und Werte}{bg=orange!50!white,fg=black}{bg=orange, fg=white}
	\begin{tabular}{rll}
		\multirow{2}{*}{$\begin{pmatrix}
			-1 		& -1 		&  0		& 0		 \\
			-1 		& 0 		&  0		& 0		 \\
			-3 		& 0 		&  0		& 0		 \\
			0 		& 2 		&  0		& 0		 \\
			3 		& 1 		&  -1		& 0		 \\
			0 		& 2 		&  0		& -1	 \\
			-3 		& -1 		&  1		& 0		 \\
			0 		& -2 		&  0		& 1	 	 \\
			\end{pmatrix}$}& &\multirow{2}{*}{$ \le \begin{pmatrix}
			-4 \\ -7 \\ -21 \\ -6 \\ 0 \\ 5 \\ 0 \\ -5
			\end{pmatrix} $}\\
		& & \\
		& \multirow{2}{*}{$\begin{pmatrix}
			y_{i,1} \\ y_{i,2} \\ \lambda_iy_{i,1}+\mu_{i-1}y_{i-1,1} \\ \lambda_iy_{i,2}+\mu_{i-1}y_{i-1,2}
			\end{pmatrix} $} & \\
		& & \\
		& & \\
		& & \\
		& & \\
		& & \\
	\end{tabular}
	\end{variableblock}
	\vspace*{-.5em}
	\begin{exampleblock}{Beispiel: Bedingungen \rom{2}: Ray Krit.($\lambda_1 = 3,\lambda_2 = 2$) }
		\begin{itemize}
			\setlength{\itemindent}{0.25cm}
			\item[i=1:] Beispielbedingung: $-y_{1,1}-y_{1,2} \le -4$, $3*y_{1,1}+y_{1,2}-3*y_{1,1}=0$
			\item[i$=$2:] Beispielbedingung: $-y_{2,1}-y_{2,2} \le -4$, 
			\item[] $3*y_{2,1}+y_{2,2}-1*(2*y_{2,1}+\mu*y_{1,1})=0$
		\end{itemize}
	\end{exampleblock}
\end{frame}




\section{Verifizierung eines GNA}
\frame{\tableofcontents[currentsection]}
\begin{frame}
	\frametitle{\color{white}Verifizierung eines GNA}
	\begin{itemize}
		\item bekommen ein GNA vom \solver
		\item wollen testen ob dieses wirklich korrekt ist
		\item[$\Rightarrow$] Nachberechnen des GNA mit geg. Matrizen und Werten
	\end{itemize}
	\begin{exampleblock}{Beispiel: Verifizierung eines GNA \rom{1}}
		Vom \solver: $y_1=\begin{pmatrix} 9 \\ 0 \end{pmatrix}$, $y_2=\begin{pmatrix} 8 \\ -8 \end{pmatrix}$, $\mu_1=0$\newline
		\vspace*{-.8em}
		\begin{itemize}
			\setlength{\itemindent}{1.5cm}
			\item[(domain)] offensichtlich wahr \checkmark
			\item[(init)] neu berechnen des \stem (\checkmark)
			\item[(point)]  $A\begin{pmatrix} 10 \\ -3 \\ 10+9+8 \\ -3 + 0 + (-8) \end{pmatrix} \le b \Leftrightarrow A\begin{pmatrix} 10 \\ -3 \\ 27 \\ -11 \end{pmatrix} \le b$ \checkmark
		\end{itemize}
	\end{exampleblock}
\end{frame}

\begin{frame}
	\begin{exampleblock}{Beispiel: Verifizierung eines GNA \rom{2}}
		Vom \solver: $y_1=\begin{pmatrix} 9 \\ 0 \end{pmatrix}$, $y_2=\begin{pmatrix} 8 \\ -8 \end{pmatrix}$, $\mu_1=0$\newline
		\begin{itemize}
			\setlength{\itemindent}{0.5cm}
			\item[(ray)]
			\item[$i=1$:]  $A\begin{pmatrix} 9 \\ 0 \\ 3*9 \\ 3*0 \end{pmatrix} \le 0 \Leftrightarrow A\begin{pmatrix} 9 \\ 0 \\ 27 \\ 0 \end{pmatrix} \le 0$ \checkmark
			\item[$i>1$:] $A\begin{pmatrix} 8 \\ -8 \\ 2*8+0*9 \\ 2*(-8)+0*0 \end{pmatrix} \le 0 \Leftrightarrow A\begin{pmatrix} 8 \\ -8 \\ 16 \\ -16 \end{pmatrix} \le 0$ \checkmark
		\end{itemize}
		$\Rightarrow$ das GNA erf\"ullt alle Kriterien\newline
		$\Rightarrow$ Nicht-Terminierung ist bewiesen
	\end{exampleblock}
\end{frame}






