\documentclass{article}
\usepackage{latexsym}
\usepackage{amssymb}
\usepackage{latexsym}
\usepackage{amsthm}
\newtheorem{definition}[figure]{Definition}
\newtheorem{theorem}[figure]{Theorem}
\renewcommand{\P}{\mathcal{P}}
\newcommand{\R}{\mathcal{R}}
\newcommand{\Q}{\mathcal{Q}}
\renewcommand{\S}{\mathcal{S}}
\newcommand{\T}{\mathcal{T}}
\newcommand{\N}{\mathbb{N}}
\title{A transformation from DP problems to SRSs}
\date{}
\begin{document}
\maketitle
\ \\[-10ex]
A string $w = acbd$ applied to a variable $x$ (denoted as $w(x)$) yields
a term $a(c(b(d(x))))$.
\begin{definition}[Pseudo-SRS DP Problem]
Let $\Sigma$ be a signature with $\Sigma = \{F\} \uplus \Sigma_{1}$
where $F$ is a binary function symbol and $\Sigma_{1} \neq \emptyset$ contains
only unary function symbols.

A DP problem $(\P,\Q,\emptyset,f)$ over the signature $\Sigma$ and the
variables $\{x,y\}$ is a \emph{Pseudo-SRS DP problem} if, and only if,
$\P = \P_{1} \uplus \P_{2}$ where
\begin{itemize}
\item $\P_{1} \subseteq \{ F(x, u(y)) \to F(v(x), y) \mid u,v \in \Sigma_{1}^{+} \}$
\item $\P_{2} \subseteq \{ F(a(x), y) \to F(x, a(y)) \mid a \in \Sigma_{1} \}$
\end{itemize}
\end{definition}
The reversal of a string from $\Sigma^{*}$ is defined inductively as $\varepsilon^{-1} = \varepsilon$
and $(aw')^{-1} = w'^{-1}a$ for $a \in \Sigma$ and $w' \in \Sigma^{*}$.
\begin{theorem}
A Pseudo-SRS DP problem $(\P,\Q,\emptyset,f)$ with $P_{1} = \{F(x, u_{i}(y)) \to F(v_{i}(x), y) \mid 1 \leq i \leq n\}$
is finite if the SRS $\S = \{u_{i} \to v_{i}^{-1} \mid 1 \leq i \leq n\}$ is strongly normalizing.
\end{theorem}
\begin{proof}
We define a partial mapping $\tau : \T(\Sigma,\{x,y\}) \to \Sigma_{1}^{*}$ by $\tau(F(s(x),t(y))) = s^{-1}t$ for all $s,t \in \Sigma_{1}^{*}$ to map terms to strings.

Let $s_{1} \to t_{1}, s_{2} \to t_{2}, \ldots$ be an infinite chain. Then, there is a substitution
$\sigma$ such that $t_{i}\sigma = s_{i+1}\sigma$ for all $i \in \N$. W.l.o.g.\ we can assume
that $\sigma(z) \in \T(\Sigma,\{x,y\})$ for all variables $z \in Dom(\sigma)$.

Thus, there is an infinite reduction $s_{1}\sigma \to_{\P} s_{2}\sigma \to_{\P} \ldots$. We distinguish
the following two cases:
\begin{enumerate}
\item \label{case1}$s_{i}\sigma \to_{F(a(x),y) \to F(x,a(y))} s_{i+1}\sigma$:\\[2ex]
  Then there are $p_{i},r_{i} \in \Sigma_{1}^{*}$ such that $s_{i}\sigma = F(a(p_{i}(x)),r_{i}(y))$ and
  $s_{i+1}\sigma = F(p_{i}(x),a(r_{i}(y)))$. Thus, $\tau(s_{i}\sigma) = (ap_{i})^{-1}r_{i} = p_{i}^{-1}ar_{i} = \tau(s_{i+1}\sigma)$.
\item \label{case2}$s_{i}\sigma \to_{F(x,u(y)) \to F(v(x),y)} s_{i+1}\sigma$:\\[2ex]
  Then there are $p_{i},r_{i} \in \Sigma_{1}^{*}$ such that $s_{i}\sigma = F(p_{i}(x),u(r_{i}(y)))$ and
  $s_{i+1}\sigma = F(v(p_{i}(x)),r_{i}(y))$. Thus, $\tau(s_{i}\sigma) = p_{i}^{-1}ur_{i} \to_{u \to v^{-1}} p_{i}^{-1}v^{-1}r_{i} = (vp_{i})^{-1}r_{i} = \tau(s_{i+1}\sigma)$.
\end{enumerate}
The DP problem $(\P_{2}, \Q,\emptyset,f)$ is clearly finite, and, consequently, there is no
infinite chain consisting just of dependency pairs from $\P_{2}$. Thus, our chain contains infinitely many
dependency pairs from $\P_{1}$. By $\tau$ and Cases (\ref{case1}) and (\ref{case2}) we can
map the chain to an infinite derivation w.r.t.\ $\S$. This implies that $\S$ is not strongly normalizing.
\end{proof}
\end{document}